The goal of event selection is to separate signal from background. Separate selection is applied to detector level MC events and generator level MC events. Simulated events are tagged as passing only the reconstruction-based, only the particle-based or both selections. The selection for data is that of the detector level MC events.

The discussion of this section is adapted from~\cite{CMS-AN-2017-159}, which uses a similar event selection.

\section{Detector level}
\label{sec:detector_level}

The event selection is based on the \ttbar$\to$lepton+jets decay topology where one of the \PW bosons decays to a charged lepton ($\ell=e, \mu$) and a corresponding neutrino, while the other \PW boson decays to quarks yielding jets.

The particle flow PF algorithm is used for reconstruction of final state objects~\cite{Sirunyan:2017ulk}. This algorithm combines signals from all sub-detectors to enhance the reconstruction performance and it allows to identify muons, electrons, photons, charged hadrons and neutral hadrons produced after a \Pp\Pp collision.

Data samples are collected using the single lepton trigger paths of the High Level Trigger summarised in Table~\ref{tab:triggers}.

\begin{table}[htp]
\centering
\caption{Trigger paths used for online selection in the analysis.}
\label{tab:triggers}
\begin{tabularx}{\linewidth}{lllXX}\hline
Final state                 & Path                                & Run range & Function & L1 seed\\\hline
e+jets                      & \small HLT\_Ele32\_eta2p1\_WPTight\_Gsf\_v & all       & \small Select $e$ with $\left|\eta\right|<2.1$ and $\pt>32$ with the tight working point and using the GSF to reconstruct tracks
                                                                                         & \small L1\_SingleEG40\newline OR\newline L1\_SingleIsoEG22er\newline OR\newline L1\_SingleIsoEG24er\newline OR\newline L1\_SingleIsoEG24\newline OR\newline L1\_SingleIsoEG26\\\hline
\multirow[t]{2}{*}{$\mu$+jets}
                            & \small HLT\_IsoMu24\_v                     & all       & \small Select isolated $\mu$ with $\pt>20$~\GeV using L3 tracker algorithm
                                                                                         & \multirow[t]{2}{*}{\small L1\_SingleMu18}\\
                            & \small HLT\_IsoTkMu24\_v                   & all       & \small Select isolated $\mu$ with $\pt>20$~\GeV using HLT tracker muon algorithm
                            & \\\hline
\end{tabularx}
\end{table}

Offline, we require exactly one tight electron/muon with $\pt>34/26\GeV$ and $|\eta|<2.1/2.4$. The tight working point allows to identify an electron/muon when it is really an electron/muon, important in a high background environment.
The event is vetoed in the presence of a second loose lepton with $\pt>15\GeV$ and $|\eta|<2.4$.

The events are required to have in addition four jets clustered with the anti-$k_{T}$ algorithm with jet separation $R=0.4$ and charged hadron subtraction (we use shorthand AK4PFchs) with $\pt>30\GeV$  and $|\eta|<2.4$. The motivation for selecting high \pt physics objects is that the detector efficiency drops at low \pt.

At least two jets are required to be \cPqb-tagged by the Combined Secondary Vertex algorithm (CSVv2) medium working point. 

At least two untagged (light) jets are required to yield a \PW boson candidate with an invariant mass $\left|m_{jj}-80.4\right|<15~\GeV$.

The event yields at different selection stages are shown in Fig.~\ref{fig:_reco_selection} and Table~\ref{tab:yields}. Table~\ref{tab:yields_cflip} shows the event yields for the colour octet \PW sample. The estimated signal fraction of the signal increases from 0.1~\%in the initial selection stage to 94.2~\% at the final selection stage - this is a measure of efficiency of our selection.

\figureEML{/reco/}{_reco_selection}{Event yields at different stages of selection: $1 \ell$, $1 \ell+\geq 4j$, $1 \ell+\geq 4j (2b)$, $1 \ell+\geq 4j (2b, 2lj)$.}

\input{tables/event_yields_tables/event_yields_table.txt}

\input{tables/event_yields_tables/event_yields_table_cflip.txt}


\section{Generator level}
\label{sec:generator_level}

In the simulation, the offline selection is mimicked at particle level using the \PSEUDOTOPPRODUCER tool~\cite{code:pseudotop}, using a common lepton selection for both electrons and muons of $\pt>26\GeV$ and $|\eta|<2.4$, and otherwise jet $\pt/\eta$ ($\pt>30~\GeV$, $|\eta|<2.4$) and W mass requirements ($\left|m_{jj}-80.4\right|<15~\GeV$) identical to the offline selection.

Charged leptons stemming from the hard process are dressed with nearby photons in a $R=0.1$ cone, and jets are clustered with the anti-$k_T$ algorithm with  $R=0.4$ cone after removing the dressed leptons as well as all neutrinos. In order to identify the flavour of the jet at particle level, ``ghost'' B-hadrons are included in the clustering after scaling their momentum by $10^{-20}$ in order that they do not change significantly the jet energy scale at particle level.
