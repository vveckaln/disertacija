We look for experimental signatures of colour connection between hadron jets resulting from the decay of a top quark pair. The colour connection is enforced by the decay of a colour singlet \PW boson. The top quark pair is produced in \Pp\Pp collisions at a center of momentum energy $\sqrt{s}=13\TeV$. Observations are conducted at the CMS experiment of the CERN LHC. The key observable is the pull angle \cite{Gallicchio:2010sw}. Also used is an adaptation of a methodology used at LEP wherein jet constituents are projected onto inter-jet planes \cite{Abbiendi:2005es}, \cite{Abdallah:2006uq}, \cite{Achard:2003pe}. When compared to non-colour connected jets, a distinctive experimental signature of colour connection is revealed. We also experimentally study the observables of colour connection between jets resulting from the decay of a theoretical colour octet \PW boson. 

This thesis shows results from a research activity undertaken by the Top Quark group of the CMS experiment. The results at various stages have been presented in the Top Modelling and Generator physics meetings - on 19 January 2016, 29 March 2016, 7 June 2016, 30 August 2016, 13 February 2018 and 17 October 2018.

The results shown in this thesis to this date have not followed the approval procedure of the CMS experiment \cite{twiki:PhysicsApprovals}. Therefore they cannot be regarded as a CMS public result and plots are marked as private work. The CMS approval is envisioned as a subsequent step in this analysis.

When work referenced in this thesis was in full progress we in April, 2018 celebrated the adhesion of Rīgas Tehniskā universitāte to a full membership of the CMS experiment. This work is the first contribution of Latvia to the experimental programme of the CERN LHC.
