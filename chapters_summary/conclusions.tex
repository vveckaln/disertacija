We have been convinced that the method of pull angle based on good track reconstruction is sensitive to identify colour-connected jets. In the distribution of the pull angle there is a disernible peak centred on 0 rad for colour connected jets while the distribution is flat for jets not connected in colour.

Convincing results have also been obtained applying the ``LEP method''. The density of particles is higher between colour-connected jets than in colour-free regions.

We were able to test the results with \PW colour octet samples in which the colour-connection between the hadronic decay products of the \PW boson was removed. Hence, these jets appeared as jets not connected in colour in the pull angle method and the ``LEP method''.

We did the exercise of unfolding the pull angle as it is a valid model to identify the true value of the observable before the reconstruction at detector. Unfolding did not bring any change into our conclusions.

We noticed that the \POWHEG + \PYTHIA MC simulation overemphasises colour connection compared to detector observations of real world events. This is represented in a more prominent central peak in the distribution of the pull angle in MC simulations. \HERWIGpp and several \PYTHIA tunes turn out to be better modellers of colour connection in hadronisation.

Overall, the fit between data and MC results is not particularly good. A combination of $\sim\frac{2}{3}$ \ttbar results and $\sim\frac{1}{3}$ \ttbar cflip results best fit the detector observations. This result was obtained in the hypothesis testing exercise.
