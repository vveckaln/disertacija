The present study is conducted using arguably the most complex and largest experimental setup in the history of humanity, involving one of the most global collaborations in research. The LHC and its experiments were designed and built to answer some of the most fundamental questions in physics:

\begin{itemize}
\item Study electroweak symmetry breaking and search for the Higgs boson. Predicted in 1964~\cite{Higgs:1964ia} \cite{Englert:1964et} the Higgs boson had been the missing piece of the Standard Model. If discovered, it would confirm fundamental concepts of our understanding of the subatomic world. The relevant discovery was announced simultaneously by CMS and ATLAS in 2012~\cite{Chatrchyan:2012xdj} \cite{Aad:2012tfa} after almost 50 years of search.
\item Study Standard Model physics to unprecedented detail with state-of-the-art detectors, high integrated luminosity and high centre-of-mass energy. One of the most interesting areas is studying the newly discovered top quark. Due to its high mass the top quark is predicted to couple well with the Higgs boson.
\item Create the conditions for the primordial Quark-Gluon Plasma thus answering fundamental questions about the evolution of our Universe.
\item Search for the Dark Matter, exotic particles, supersymmetric partners, extra dimensions and other puzzling and hypothetical topics beyond the Standard Model. These questions are still elusive and are motivations behind the High-Luminosity LHC, Future Circular Collider and other experimental concepts on a grand scale.
\end {itemize}

The CMS experiment is one of the flagship experiments of the Large Hadron Collider. Hence, in the present discussion the LHC will be presented first followed by a description of the CMS apparatus.

\section{The LHC}

The LHC is a two-ring superconducting hadron accelerator and collider installed in a 26.7~km tunnel 45\textendash170~m underground traversing the Franco-Swiss border in Geneva area (Fig.~\ref{fig:LHC_underground}). The hadrons circulate in the LHC with a constant radius but variable frequency. Hence, the LHC is a synchrotron. It reuses the tunnel and injection chain of the Large Electron-Positron collider (LEP).

Initially the LHC project faced severe competition from the more powerful Superconducting Super Collider in the USA. C. Rubbia argued that the luminosity higher by a factor of 10 at the LHC would compensate its lower energy vis-à-vis the SSC. Eventually, the SSC project was cancelled in 1993. Cost overruns played a role. The CERN Council approved the LHC project in 1994. It started data taking in 2008.

\begin{figure}[H]
  \centering
  \includegraphics[width=0.8\textwidth]{fig/LHC_underground.png}
  \caption{The Large Hadron Collider situated underground on the French-Swiss border in Geneva area~\cite{cds:LHCunderground}.}
  \label{fig:LHC_underground}
\end{figure}

The protons at the LHC circulate at nearly the speed of light. The per proton energy is 7~\TeV, the $\gamma$ factor being 7461. It is not practical to accelerate a proton from zero velocity to such an energy in one accelerator. Therefore before reaching this energy the protons undergo a sequence of accelerations in the CERN accelerator complex (Fig.~\ref{fig:CERN_accelerator_complex}):

\begin{itemize}
\item up to 50~\MeV in Linac2
\item up to 1.4~\GeV in PS Booster
\item up to 26~\GeV in the Proton Sinchrotron (PS)
\item up to 450~\GeV in the Superproton sinchrotron (SPS)
\end{itemize}

\begin{figure}[H]
  \centering
  \includegraphics[width=1\textwidth]{fig/CERNacceleratorcomplex.jpg}
  \caption{The CERN accelerator complex~\cite{espace:CERNacceleratorcomplex}.}
  \label{fig:CERN_accelerator_complex}
\end{figure}

After the protons are fully accelerated they are allowed to circulate in the LHC \textendash the LHC is a storage ring. There are $1.15\times10^{11}$ protons in each bunch and 2808 bunches in circulation. The revolution frequency is 11.245~kHz~\cite{Bruning:2004ej}. Each bunch crossing lasts 25~ns. There is an ultrahigh vacuum maintained in the beam pipes.

The LHC uses superconducting magnet systems. Particularly, the dipole magnets bend the beam in a circular arc, and quadrupole magnets squeeze the beam near the collision points. Magnets of higher orders provide steering and correction to the beam. The magnet systems rely on the NbTi Rutherford cable, that is cooled by helium to below 2~K - below the lambda point of helium. Thus unlike other large accelerators that use NbTi but operate above the lambda point of helium (Tevatron-FNAL, HERA-DESY and RHIC-BNL) a much higher field of 8~T can be achieved in the dipole magnets at the LHC. A special two-in-one dipole magnet was designed for the LHC that uses the same yoke but fields of different polarities for the two proton beams circulating in opposite directions. Cooling the magnets requires the largest cryogenic system on Earth~\cite{MYERS:2013hra}, \cite{Evans:2008zzb}.

The design COM of the LHC is 14~\TeV. In its first data taking period from 2010\textendash2013 it operated at \sqrts=7\textendash8~\TeV. This period is referred to as Run I. In its second data taking period from 2015\textendash2018 referred to as Run II it operated at \sqrts=13\textendash14~\TeV. The present study is conducted with Run II data.

The LHC houses two high-luminosity experimental insertions \textendash CMS and ATLAS each targeting a luminosity above $10\~{\nicefrac{1}{\text{pb}\cdot \text{s}}}$, one \cPqb physics experiment LHCb targeting a luminosity of $0.1~\nicefrac{1}{\text{pb}\cdot \text{s}}$ and one dedicated ion collision experiment \textendash ALICE. 

\section{The CMS detector}

The CMS detector is located at Point~5 of the LHC, close to the French village of Cessy, between Lake Geneva and the Jura mountains. It is placed in underground caverns about 100~m deep that were excavated to house the detector complex.

The CMS detector is designed to operate in diverse physics programmes in the \TeV range. It is an onion-type detector covering $4\pi$ of solid angle around the collision point. The CMS detector is composed of the following layers starting from the beam axis \textendash a silicon pixel and strip tracker, a lead tungstate electromagnetic calorimeter, a brass and a plastic scintillator hadron calorimeter, a superconducting magnet producing 3.8\textendash4.0~T of magnetic field, and a gas-ionisation muon spectrometer~\cite{Chatrchyan:2008aa}. The shape of the CMS detector is a cylinder. It has endcaps on both ends while the cental part is called the barrel. The length of the CMS detector is 21.6~m, diameter 14.6~m and total weight \SI{12500}{t}. A cut-away view of the CMS detector is presented in Fig.~\ref{fig:CMS_detector}.

\begin{figure}[hbtp]
\centering
\def\twidth{1}
\includegraphics[width=\twidth\textwidth]{fig/cms_120918_03}
\caption{A cut-away view of the CMS detector~\cite{Sakuma:2013jqa}.}
\label{fig:CMS_detector}
\end{figure}

Starting from the beam interaction region, particles first enter a tracker, in which charged-particle trajectories (tracks) and origins (vertices) are reconstructed from signals (hits) in the sensitive layers.  The tracker is immersed in a magnetic field that bends the trajectories and allows the electric charges and momenta of charged particles to be measured. Electrons and photons are then absorbed in an electromagnetic calorimeter (ECAL). The corresponding electromagnetic showers are detected as clusters of energy recorded in neighbouring cells, from which the energy and direction of the particles can be determined.  Charged and neutral hadrons may initiate a hadronic shower in the ECAL as well, which is subsequently fully absorbed in the hadron calorimeter (HCAL). The corresponding clusters are used to estimate their energies and directions.  Muons and neutrinos traverse the calorimeters with little or no interactions. While neutrinos escape undetected, muons produce hits in additional tracking layers called muon detectors, located outside the calorimeters. This simplified view is graphically summarised in Fig.~\ref{fig:CMSpf}, which displays a sketch of a transverse slice of the CMS detector.

A significantly improved event description can be achieved by correlating the basic elements from all detector layers (tracks and clusters) to identify each final-state particle, and by combining the corresponding measurements to reconstruct the particle properties on the basis of this identification. This holistic approach is called particle-flow (PF) reconstruction~\cite{Sirunyan:2017ulk}.

\begin{figure}[h]
  \centering
  \includegraphics[width=1\textwidth]{fig/CMSpf.png}
  \caption{A sketch of the specific particle interactions in a transverse slice of the CMS detector, from the beam interaction region to the muon detector~\cite{Sirunyan:2017ulk}.}
  \label{fig:CMSpf}
\end{figure}

The fine-granularity and fast response tracker~\cite{Karimaki:368412}, \cite{tracker_addendum} is an important segment in resolving the fine jet constituents. It is closely aligned to the beam axis and has a length of 5.8~m and radius of 2.5~m. The CMS solenoid provides a homogeneous and coaxial magnetic field of 3.8\textendash4.0~T over the full volume of the tracker. At radius below 10~cm a hit rate at the order of $100~\nicefrac{\text{kHz}}{\text{mm}^2}$ is encountered. In order to achieve the desired resolution \SI{100}{\um}~$\times$~\SI{150}{\um} pixel detectors are used. At a higher radius the reduced particle flux allows the use of silicon micro-strip detectors with a typical size of 10~cm~$\times$~\SI{80}{\um} to 25~cm~$\times$~\SI{150}{\um}, the size increasing with an increasing radius. There are 66 million pixels with 1~$\text{m}^{2}$ active area in the pixel detector and 9.3~million strips and 193~${\text{m}}^2$ active area in the strip detector.

The electromagnetic calorimeter of CMS (ECAL) is a hermetic homogeneous calorimeter made of \num{61200} lead tungstate ($\text{PbWO}_{4}$) crystals mounted in the central barrel part, closed by \num{7324} crystals in each of the two endcaps. The barrel part covers the pseudorapidity range $\left|\eta\right|<1.479$ while the endcaps cover the pseudrapidity range $1.479<\left|\eta\right|<3.0$. A preshower detector is placed in front of the endcap crystals. Avalanche photodiodes (APDs) are used as photodetectors in the barrel and vacuum phototriodes (VPTs) in the endcaps. The $\text{PbWO}_{4}$ crystals exhibit characteristics that make them an appropriate choice for an electromagnetic calorimeter at the LHC. The high density 8.28~$\nicefrac{\text{g}}{\text{cm}^3}$, short radiation length (0.89~cm) and small Molière radius (2.2~cm) result in a fine granularity and a compact calorimeter. The scintillation decay time of $\text{PbWO}_{4}$ is of the same order of magnitude as the LHC bunch crossing time: about 80~\% of the light is emitted in 25~ns. In the barrel the crystal cross-section corresponds to approximately 0.0174~$\times$~0.0174 in $\eta-\phi$, corresponding to a front cross-section 22~$\times$22~$\text{mm}^2$ and a rear cross-section 26~$\times$26~$\text{mm}^2$. The crystal length is 230~mm, corresponding to 25.8~$X_{0}$. There are \num{61200} crystals in the barrel. In the endcaps the crystals have a rear face cross section 30~$\times$30~$\text{mm}^2$, a front face cross section 28.62~$\times$~28.62~$\text{mm}^2$ and a length of 220~mm (24.7~$X_{0}$). Additionally in the fiducial region $1.653<\left|\eta\right|<2.6$ there is a preshower detector whose principal aim is to identify neutral pions in the endcaps. The energy resolution of the barrel electromagnetic calorimeter depends on the incident energy and is measured from 0.94~\% ($\nicefrac{\sigma}{E}$) at 20~\GeV to 0.34~\% at 250~\GeV~\cite{Adzic:2007mi}. The preshower detector consists of a lead radiator where electromagnetic showers from incoming electrons/photons are initiated. Behind the lead radiator there are silicon strips to measure the deposited energy and transverse shower profiles.

The hadronic calorimeter~\cite{HCAL_report} consists of a barrel ($\left|\eta\right|<1.3$) and two endcap disks ($1.3<\left|\eta\right|<3.0 $). The space of the hadron calorimeter in the central pseudorapidity region is constrained. Therefore, an outer tail catcher layer behind the soleonoid is used. The solenoid is used as an additional absorber for the tail catcher. The absorber consists of a 40~mm thick front steel plate, followed by eight 50.5~mm thick brass plates, six 56.5~mm thick brass plates, and a 75~mm thick steel back plate. The total absorber thickness at 90~$^{\circ}$ is 5.82 interaction lengths ($\lambda_{\text{I}}$). As the active material plastic scintillator arranged in tiles is used. Wavelength shifting fibres are used to bring out the light. The hadronic calorimeter is read out in individual towers with a cross section $\Delta\eta\times\Delta\phi=0.087~\times~0.087$ for $\left|\eta\right|<1.6$ and $0.17~\times~0.17$ at larger pseudorapidities. The hadronic calorimeter at  $\left|\eta\right|$ extending up to $\simeq5.0$ where particle flux and radiation damage is highest is complemented by hadron forward calorimeters. The hadron forward calorimeter consists of a steel absorber composed of grooved plates. Radiation-hard quartz fibres are inserted in the grooves along the beam direction and are read out by photomultipliers. The signals are grouped so as to define calorimeter towers with a cross section $\Delta\eta~\times~\Delta\phi=0.175\times0.175$ over most of the pseudorapidity range. 

The magnet is located behind the calorimeters and the tracker to ensure that as less material as possible is situated between these subdetectors and the interaction point. The length of the magnet is 12.5~m and the free-bore radius is 3.15~m. The coil delivers a 3.8\textendash4.0~T uniform and axial magnetic field to the tracker and the calorimeters. The magnet operates at 4.45~K and uses a NbTi superconducting coil. The magnet is characterised by a high stored-energy/mass ratio 11.6~$\nicefrac{\text{kJ}}{\text{kg}}$.

The muon channel is a very powerful tool for studying interesting HEP processes and has been very important for CMS since the experiment's inception. This is because of the relative ease of detecting muons and because they are minimally affected by radiative losses in the tracker material. Four muon detector planes are located outside the solenoid coil interleaved with three layers of steel yoke~\cite{muon_tech_rep}. In the barrel region - $\left|\eta\right|<1.2$ where the muon rate is low, and the 4~T magnetic field is uniform and mostly contained in the steel yoke, drift chambers are used. In the endcaps $0.9<\left|\eta\right|<2.4$ where the muon rates and background levels are high and the magnetic field is large and non-uniform, the muon system uses cathode strip chambers (CSC). Because of the uncertainty in the eventual background rates and in the ability of the muon system to measure the correct beam-crossing time when the LHC reaches full luminosity, a complementary, dedicated trigger system consisting of resistive plate chambers (RPC) is added in both the barrel and endcap regions. The RPCs provide a fast, independent, and highly-segmented trigger with a sharp \pt threshold over a large portion of the rapidity range ($\left|\eta\right|<1.6$) of the muon system. The particle flow reconstruction involves a global trajectory fit across the muon detectors and the inner tracker. 

Jets are reconstructed using the anti-$k_{\text{T}}$ algorithm~\cite{Cacciari:2008gp} with radius parameter $R=0.4$ as implemented by the \FASTJET~\cite{Cacciari:2011ma} package. Distance $d_{ij}$ between jets is determined by using $p=-1$ in the general formula:

\begin{equation}
d_{ij}=\text{min}(k_{\text{T}i}^{2p}, k_{\text{T}j}^{2p})\frac{\Delta_{ij}^{2}}{R^{2}},
\end{equation}

\noindent where $\Delta_{ij}^{2}=(y_{i}-y_{j})^{2}+(\phi_{i}-\phi_{j})^{2}$ and $k_{\text{T}i}$, $y_{i}$, $\phi_{i}$ are respectively the transverse momentum, rapidity and azimuth of particle $i$. 

The key feature of this algorithm is that soft particles do not modify the shape of the jet. Given separation between jets $\Delta_{ij}\leq2R$ the jets have conical shapes. 
