\documentclass[titlepage, a4paper, LV, SHORT]{mythesis}
\usepackage[bindingoffset=0cm, left=2.5cm, right=2.5cm, top=2.5cm, bottom=2.5cm]{geometry}
%\usepackage[utf8]{inputenc}
\usepackage[backend=bibtex, style=numeric, sorting=none]{biblatex}
\usepackage{amsmath}
\usepackage{graphicx}
\usepackage{makecell}
\usepackage{multirow}
\usepackage{longtable}
%\usepackage{setspace}
\usepackage{listings}
\usepackage{lscape}
\usepackage{amssymb}
\usepackage{tabularx}
\usepackage{enumerate}
\usepackage[T1]{fontenc}
\renewcommand{\baselinestretch}{1.5}
\setlength{\skip\footins}{1.25 cm}
\linespread{1.2}
\usepackage{fontspec}
\setmainfont{TeX Gyre Termes}
\renewcommand{\normalsize}{\fontsize{12pt}{13.4pt}\selectfont}
\title{Virsotnes kvarku pāra sabrukšanas ceļā radušos krāsu plūsmu pētījumi ar 13 TeV CERN LHP KMS eksperimentā} 
\author{Viesturs Veckalns}
\institution{Rīgas Tehniskā universitāte}
\supervisor{Leonīds Ribickis}
\date{2019}
\newcommand{\DeltaR}{\ensuremath{\Delta R^{}}\xspace}
\newcommand{\leadingjet}{\ensuremath{j_{1}^{W}}\xspace}%
\newcommand{\scndleadingjet}{\ensuremath{j_{2}^{W}}\xspace}%
\newcommand{\leadingb}{\ensuremath{j_{1}^{b}}\xspace}%
\newcommand{\scndleadingb}{\ensuremath{j_{2}^{b}}\xspace}%
\newcommand{\hadronicb}{\ensuremath{j_{\text{h}}^{b}}\xspace}%
\newcommand{\pullangle}{\ensuremath{\theta_{\text{p}}}\xspace}%
\newcommand{\pullvector}{\ensuremath{\vec{P}}\xspace}%
\newcommand{\pvmag}{\ensuremath{\left|\vec{P}\right|}\xspace}%
\newcommand{\pval}{$p$-value}%

\newcommand{\jettitle}[1]{%
\ifthenelse{\equal{#1}{leading_jet}}{\leadingjet}{}%
\ifthenelse{\equal{#1}{scnd_leading_jet}}{\scndleadingjet}{}%
\ifthenelse{\equal{#1}{leading_b}}{\leadingb}{}%
\ifthenelse{\equal{#1}{scnd_leading_b}}{\scndleadingb}{}%
}%

\newcommand{\countertitle}[1]{%
\ifthenelse{\equal{#1}{leading_jet}}{\scndleadingjet}{}%
\ifthenelse{\equal{#1}{scnd_leading_jet}}{\leadingjet}{}%
\ifthenelse{\equal{#1}{leading_b}}{\scndleadingb}{}%
\ifthenelse{\equal{#1}{scnd_leading_b}}{\leadingb}{}%
}%


\newcommand{\observabletitle}[1]{%
\ifthenelse{\equal{#1}{pull_angle}}{pull angle \pullangle}{}%
\ifthenelse{\equal{#1}{pvmag}}{magnitude of the pull vector \pvmag}{}%
}%

\newcommand{\chargetitle}[1]{%
\ifthenelse{\equal{#1}{allconst}}{all jet constituents}{}%
\ifthenelse{\equal{#1}{chconst}}{only charged jet constituents}{}%
}%

\newcommand{\methodtitle}[1]{%
\ifthenelse{\equal{#1}{nominal}}{\ensuremath{t\overline{t}}\xspace}{}%
\ifthenelse{\equal{#1}{cflip}}{\ensuremath{t\overline{t}\ \text{cflip}}\xspace}{}%
}%

\newcommand{\binningtitle}[1]{% 
  \ifthenelse{\equal{#1}{ORIG}}
             {the original binnning}
             {}%    
  \ifthenelse{\equal{#1}{ATLAS3}}{3 regularly sized bins}{}%
  \ifthenelse{\equal{#1}{SIGMA_0p1}}{the optimised binning with a $\sigma$ factor of 0.1}{}%
  \ifthenelse{\equal{#1}{SIGMA_0p6}}{the optimised binning with a $\sigma$ factor of 0.6}{}%
}%           

\newcommand{\flowtitle}[1]{%
\ifthenelse{\equal{#1}{N}}{particle}{}%
\ifthenelse{\equal{#1}{E}}{energy}{}%
\ifthenelse{\equal{#1}{Pt}}{\ensuremath{p_{T}}\xspace}{}%
}%

\newcommand{\recoleveltitle}[1]{%
\ifthenelse{\equal{#1}{gen}}{generator}{}%
\ifthenelse{\equal{#1}{reco}}{reconstruction}{}%
}%

\newcommand{\modeltitle}[1]{%
\ifthenelse{\equal{#1}{nominal}}{SM}{}%
\ifthenelse{\equal{#1}{cflip}}{colour octet $W$}{}%
}%

\def\customwidth{\textwidth}

%% \ExplSyntaxOn
%% \NewExpandableDocumentCommand{\capitalise}{m}{\tl_mixed_case:n{#1}}
%% \ExplSyntaxOff

\newcommand{\figureratiographslv}[3]{
  \def\twidth{0.45}
  \def\chargetag{#1}
  \def\flowtag{#2}
  \def\tagl{chirg_#1_#2}
  \begin{figure}[hbtp]
    \centering
  \subfloat[\leadingb, \scndleadingb]{%
    \includegraphics[width=\twidth\textwidth]{fig/ratiographs_merged_self/L_blb2l_\flowtag_\chargetag_reco.png}%
    \label{fig:\tagl_a}
  }\hfil
  \subfloat[\hadronicb, $j_{\text{f}}^{W}$]{%
    \includegraphics[width=\twidth\textwidth]{fig/ratiographs_merged_self/L_qfhb_\flowtag_\chargetag_reco.png}%
    \label{fig:\tagl_b}
  }\hfil
 \subfloat[$j_{\text{c}}^{W}$, \hadronicb]{%
    \includegraphics[width=\twidth\textwidth]{fig/ratiographs_merged_self/L_hbqc_\flowtag_\chargetag_reco.png}%
    \label{fig:\tagl_c}
  }
  \caption{#3}
  \label{fig:\tagl}
\end{figure}
}


\newcommand{\figureChilv}[3]{
  \def\twidth{0.45}
  \def\chargetag{#1}
  \def\flowtag{#2}
  \def\tagl{chi_\chargetag_\flowtag}
  \begin{figure}[hbtp]
  \centering
  \subfloat[\leadingb, \scndleadingb]{%
    \includegraphics[width=\twidth\textwidth]{fig/histos/L/reco/chi/charge/\chargetag/L_chiblb2l_\flowtag_\chargetag_reco_jetprt.png}%
    \label{fig:\tagl_a}
  }\hfil
  \subfloat[\hadronicb, $j_{f}^{W}$]{%
    \includegraphics[width=\twidth\textwidth]{fig/histos/L/reco/chi/charge/\chargetag/L_chiqfhb_\flowtag_\chargetag_reco_jetprt.png}%
    \label{fig:\tagl_b}
  }\\
 \subfloat[$j_{c}^{W}$, \hadronicb]{%
    \includegraphics[width=\twidth\textwidth]{fig/histos/L/reco/chi/charge/\chargetag/L_chihbqc_\flowtag_\chargetag_reco_jetprt.png}%
    \label{fig:\tagl_c}
  }\hfil
 \subfloat[\leadingjet, \scndleadingjet]{%
    \includegraphics[width=\twidth\textwidth]{fig/histos/L/reco/chi/charge/\chargetag/L_chiqlq2l_\flowtag_\chargetag_reco_jetprt.png}%
    \label{fig:\tagl_d}
  }
  \caption{#3}
  \label{fig:\tagl}

\end{figure}
}


\newcommand{\figunfoldinglv}[7]{%
  \def\method{#1}%
  \def\jet{#2}%
  \def\chargetag{#3}%
  \def\observable{#4}%
  \def\binningscheme{#5}%
%  \newcommand{\optlevel}{}%
  \ifthenelse{\equal{\binningscheme}{ORIG}}{\def\optlevel{ORIG}}{\def\optlevel{OPT}}%
  %\newcommand{\bintag}{}
  \ifthenelse{\equal{\binningscheme}{ORIG}}{\def\bintag{}}{\def\bintag{_\binningscheme}}
  \def\sampletag{#6}
  \def\commonplottag{\jet_\chargetag_\observable_\optlevel}%
  \def\directory{\jet_\chargetag_\sampletag_\observable_\optlevel_\method\bintag}%
  \def\subdir{\observable/\binningscheme}
  \def\taglabel{unfolding_\method_\jet_\chargetag_\observable_\optlevel_\sampletag_\binningscheme}%
  \begin{figure}
  \centering
    \subfloat[Migrācijas matrica, kuras $x$ ass atbilst rekonstrukcijas līmenim, bet $y$ ass - ģenerācijas līmenim.]{%
      \includegraphics[width=0.45\textwidth]{fig/unfolding_\method/\subdir/\directory/migrationmatrix.png}%
      \label{fig:\taglabel_a}
    }\hfil
    \subfloat[Dati un Monte Karlo, kuri tiek izmantoti kā ievade.]{%
      \includegraphics[width=0.45\textwidth]{fig/common_plots_\method/\subdir/\commonplottag_reco_in.png}%
      \label{fig:\taglabel_b}
    }\\
    \subfloat[Atlocītais rezultāts.]{%
      \includegraphics[width=0.45\textwidth]{fig/common_plots_\method/\subdir/\commonplottag_gen_out.png}%
      \label{fig:\taglabel_c}
    }\hfil
    \subfloat[Atpakaļ locītais rezultāts.]{%
      \includegraphics[width=0.45\textwidth]{fig/common_plots_\method/\subdir/\commonplottag_reco_out.png}%
      \label{fig:\taglabel_d}
    }\\  
    \subfloat[Stabilitāte un tīrība katrā vērtību intervālā.]{%
      \includegraphics[width=0.45\textwidth]{fig/unfolding_\method/\subdir/\directory/stabpur.png}%
      \label{fig:\taglabel_e}
    }\hfil
    \subfloat[Vilkme.]{%
      \includegraphics[width=0.45\textwidth]{fig/unfolding_\method/\subdir/\directory/pull.png}%
      \label{fig:\taglabel_f}
    }
    \caption{#7}
    \label{fig:\taglabel}
  \end{figure}
}

\newcommand{\figureEML}[4][]{
  \def\twidth{0.33}
  \begin{figure}[hbtp]
  \subfloat[$e$ + jets channel.]{
    \includegraphics[width=\twidth\textwidth]{fig/histos#1/E/#2/E#3.png}
    \label{fig:#3_a}
  }%
  \subfloat[$\mu$ + jets channel.]{
    \includegraphics[width=\twidth\textwidth]{fig/histos#1/M/#2/M#3.png}
    \label{fig:#3_b}
  }%
 \subfloat[Combined lepton + jets channel.]{
    \includegraphics[width=\twidth\textwidth]{fig/histos#1/L/#2/L#3.png}
    \label{fig:#3_c}
  }
   \caption{#4}
  \label{fig:#3}
\end{figure}
}

%% \newcommand{\figureEML}[4][]{
%%   \def\twidth{0.33}
%%   \begin{figure}[hbtp]
%%   \subfloat[$e$ + jets channel.]{
%%     \includegraphics[width=\twidth\textwidth]{fig/histos#1/E/#2/E#3.png}
%%     \label{fig:#3_a}
%%   }%
%%   \subfloat[$\mu$ + jets channel.]{
%%     \includegraphics[width=\twidth\textwidth]{fig/histos#1/M/#2/M#3.png}
%%     \label{fig:#3_b}
%%   }%
%%  \subfloat[Combined lepton + jets channel.]{
%%     \includegraphics[width=\twidth\textwidth]{fig/histos#1/L/#2/L#3.png}
%%     \label{fig:#3_c}
%%   }
%%    \caption{#4}
%%   \label{fig:#3}
%% \end{figure}
%% }


\newcommand{\figureEMLcontrol}[4]{
  \def\twidth{0.33}
  \begin{figure}[hbtp]
  \subfloat[$e$ + jets channel.]{
    \includegraphics[width=\twidth\textwidth]{fig/histos/E/#2/#3.png}
    \label{fig:#2_a}
  }%
  \subfloat[$\mu$ + jets channel.]{
    \includegraphics[width=\twidth\textwidth]{#1/M/#2/#3.png}
    \label{fig:#2_b}
  }%
 \subfloat[Combined lepton + jets channel.]{
    \includegraphics[width=\twidth\textwidth]{#1/L/#2/#3.png}
    \label{fig:#2_c}
  }
   \caption{#3}
  \label{fig:#2_#3}
\end{figure}
}

\usepackage{ptdr-definitions}
\usepackage{siunitx}
\usepackage{float}
\usepackage[latvian]{babel}
\usepackage{caption}
\usepackage{subfig}
\captionsetup[figure]{labelsep=space}
\DeclareCaptionFormat{hfillstart}{\hfill#1#2#3\par}
\captionsetup[table]{
  format=hfillstart,
  labelsep=newline,
  justification=centering
}
\addto\captionslatvian{\renewcommand{\figurename}{att.}}
%\renewcommand\part{\secdef\@part\@spart}
\def\startnewpart{FALSE}
\def\partname{}
\makeatletter
\def\@part[#1]#2{%
    %% \ifnum \c@secnumdepth >-2\relax
    %%   \refstepcounter{part}%
    %%   \addcontentsline{toc}{part}{\thepart\hspace{1em}#1}%
    %% \else
      \addcontentsline{toc}{part}{#1}%
%    \fi
    \markboth{}{}%
    {\centering
     \interlinepenalty \@M
     \normalfont
     %% \ifnum \c@secnumdepth >-2\relax
     %%   \huge\bfseries \partname\nobreakspace\thepart
     %%   \par
     %%   \vskip 20\p@
     %% \fi
     \Huge \bfseries #2\par}%
    \@endpart}

\renewcommand\chapter{%\if@openright\cleardoublepage\else\clearpage\fi
  \setcounter{figure}{0}
  \addtocounter{totfigures}{\value{figure}}%
  \ifthenelse{\equal{\startnewpart}{FALSE}}{
    \clearpage
  }{
  }
                    \def\startnewpart{FALSE}
                    \thispagestyle{plain}%
                    \global\@topnum\z@
                    \@afterindentfalse
                    \secdef\@chapter\@schapter
%                    \def\startnewpart{FALSE}
}

  \renewcommand\part{%
  \def\startnewpart{TRUE}
\if@openright
    \cleardoublepage
  \else
    \clearpage
  \fi
  %% \thispagestyle{plain}%
  \if@twocolumn
    \onecolumn
    \@tempswatrue
  \else
    \@tempswafalse
  \fi
  \null\vfil
  \secdef\@part\@spart}
  %%   \renewcommand\part{%
  %% \def\startnewpart{TRUE}
  %% \clearpage
  %% \secdef\@part\@spart}
\def\@endpart{%% \vfil\newpage
              %% \if@twoside
              %%  \if@openright
              %%   \null
              %%   \thispagestyle{empty}%
              %%   \newpage
              %%  \fi
              %% \fi
              \if@tempswa
                \twocolumn
              \fi}
\makeatother


\addbibresource{disertacija.bib}
\renewcommand{\contentsname}{Satura rādītājs}
\usepackage[toc]{glossaries}
\renewcommand{\glossarypreamble}{Nepieciešams ieviest augstas enerģijas fizikā lietoto terminu tulkojumus latviešu valodā. Zemāk sniegti šajā darbā lietotie nozares termini latviešu valodā, norādot atbilstošo terminu angļu valodā. Saīsinājumiem papildus ir sniegts to atšifrējums latviešu valodā. Tiek norādīta lapaspuse, kurā termins sastopams.}
\newglossaryentry{strūklām}
{
  name=strūkla,
  description={jet}
}

\newglossaryentry{virsotnes kvarku}
{
  name=virsotnes kvarks,
  description={top quark}
}

\newglossaryentry{singletam}
{
  name=singlets,
  description={singlet}
}

\newglossaryentry{okteta}
{
  name=oktets,
  description={octet}
}

\newglossaryentry{LHP}
{
  name=LHP,
  description={LHC\\Lielais hadronu paātrinātājs}
}

\newglossaryentry{KMS}
{
  name=KMS,
  description={CMS\\Kompaktais mionu solenoīds}
}

\newglossaryentry{LEP}
{
  name=LEP,
  description={LEP\\Lielais elektronu pozitronu kolaiders}
}

\newglossaryentry{vilkmes leņķis}
{
  name=vilkmes leņķis,
  description={pull angle}
}

\newglossaryentry{otrās pakāpes}
{
  name=otrā pakāpe,
  description={next-to-next leading order}
}


\newglossaryentry{spīduma}
{
  name=spīdums,
  description={luminosity}
}

\newglossaryentry{gruvešos}
{
        name=gruveši,
        description={debris}
}

\newglossaryentry{smaržas}
{
        name=smarža,
        description={flavour}
}

\newglossaryentry{norobežojuma barjeras}
{
        name=norobežojuma barjera,
        description={confinement barrier}
}

\newglossaryentry{trekeris}
{
        name=trekeris,
        description={tracker}
}

\newglossaryentry{lietus}
{
        name=lietus,
        description={shower}
}

\newglossaryentry{puduri}
{
        name=puduris,
        description={cluster}
}

\newglossaryentry{papildus dimensijas}
{
        name=papildus dimensijas,
        description={extra dimensions}
}

\newglossaryentry{Augsta spīduma LHP}
{
        name=Augsta Spīduma LHP,
        description={High-Luminosity LHP}
}

\newglossaryentry{Nākotnes apļveida kolaideri}
{
        name=Nākotnes apļveida kolaiders,
        description={Future Circular Collider}
}

\newglossaryentry{ievades ķēdi}
{
        name=ievades ķēdi,
        description={injection chain}
}

\newglossaryentry{pastiprinātājā}
{
        name=pastiprinātājs,
        description={booster}
}

\newglossaryentry{uzkrāšanas aplis}
{
        name=uzkrāšanas aplis,
        description={storage ring}
}

\newglossaryentry{pūlī}
{
        name=pūlis,
        description={bunch}
}

\newglossaryentry{darba periodu}
{
        name=darba periods,
        description={Run}
}

\newglossaryentry{strēmeļu}
{
        name=strēmele,
        description={strip}
}

\newglossaryentry{gala segumi}
{
        name=gala segums,
        description={endcap}
}

\newglossaryentry{mucu}
{
        name=muca,
        description={barrel}
}

\newglossaryentry{ierosinājumiem}
{
        name=ierosinājums,
        description={hit}
}

\newglossaryentry{virsotnes}
{
        name=virsotne,
        description={vertex}
}
\newglossaryentry{uzticamajā reģionā}
{
        name=uzticamais reģions,
        description={fiducial region}
}

\newglossaryentry{atblāzmas novērotājs}
{
        name=atblāzmas novērotājs,
        description={tail catcher}
}


\newglossaryentry{gala stāvokļa}
{
        name=gala stāvoklis,
        description={final state}
}

\newglossaryentry{daļiņu plūsmas}
{
        name=daļiņu plūsma,
        description={particle-flow}
}

\newglossaryentry{graudainības}
{
        name=graudainība,
        description={granularity}
}

\newglossaryentry{uzticamajā}
{
        name=uzticamais,
        description={fiducial}
}


\newglossaryentry{dakstiņos}
{
        name=dakstiņš,
        description={tile}
}


\newglossaryentry{torņos}
{
        name=torņi,
        description={tower}
}


\newglossaryentry{straujumu}
{
        name=straujums,
        description={rapidity}
}

\newglossaryentry{AEF}
{
        name=AEF,
        description={HEP\\
        augstas enerģijas fizika}
}

\newglossaryentry{mīkstās}
{
        name=mīksts,
        description={soft}
}

\newglossaryentry{sakopošanas}
{
        name=sakopošana,
        description={clustering}
}

\newglossaryentry{cietā}
{
        name=ciets,
        description={hard}
}

\newglossaryentry{Augsta līmeņa trigera}
{
        name=Augsta līmeņa trigers (ALT),
        description={High Level Trigger (HLT)}
}


\newglossaryentry{uzņēmību}
{
        name=uzņēmība,
        description={acceptance}
}

\newglossaryentry{veidoliem}
{
        name=veidols,
        description={shape}
}

\newglossaryentry{sliekšņus}
{
        name=slieksnis,
        description={cut}
}

\newglossaryentry{Atlocīšana}
{
        name=atlocīšana,
        description={unfolding}
}

\newglossaryentry{traucējumu}
{
        name=traucējums,
        description={nuisance}
}

\newglossaryentry{Sagrūduma}
{
        name=sagrudūms,
        description={Pileup}
}

\newglossaryentry{profila iespējamību}
{
        name=profila iespējamība,
        description={profile likelihood}
}

\makeglossaries

\begin{document}
\maketitle
\begin{abstract}
        LHP, kas strādā ar 13~TeV masas centra enerģiju, ir virsotnes kvarku fabrika. Virsotnes kvarku ražošanas šķērsgriezums LHP ir 803~pb. Virsotnes kvarka mūža ilgums ir $3,3\times10^{-25}$~s, un tas ir tik īss, ka atšķirībā no citiem kvarkiem virsotnes kvarks sabrūk, pirms tas hadronizējas. Virsotnes kvarks sabrūk vājajā ceļā, izstarojot \PW bozonu. Ja \PW bozons sabrūk hadroniskajā veidā, stiprajā kodola mijiedarbībā tiek radītas daļiņu strūklas. Šo procesu apraksta kvantu hromodinamika, un varam modelēt virsotnes kvarka sabrukšanas procesu ar krāsu lādiņu un krāsu saitēm. Strūklas, kas radušās, sabrūkot \PW bozonam, mijiedarbojas krāsu laukā (tās ir saistītas ar krāsām). Saistība ar krāsām atstāj pamanāmus eksperimentālus nospiedumus, ko mēs varam novērot KMS detektorā, īpaši izmantojot tā trekeri, 4~T solenoīdu un kalorimetrus. KMS eksperimentā šāds pētījums tiek veikts pirmoreiz. Krāsu saistību starp strūklām, kas radušās, sabrūkot virsotnes kvarku pārim, pētām, izmantojot gala stāvokli, ko veido viens lādēts leptons, divas vieglās strūklas un divas \cPqb atzīmētās strūklas. Izmantojam vilkmes leņķi un daļiņu projicēšanu uz plaknes, ko veido divas strūklas. Tiek izmantots arī krāsu okteta \PW spēļu modelis, lai novērtētu dažādo metožu sniegumu.

\end{abstract}

\chapter*{Pateicības}
\label{chap:acknowledgements}
Pamati šim darbam tika ielikti Pikosekunžu silīcija reizinātāju-elektronikas-kristālu pētniecības (\textit{Picosecond Siliconphotomultiplier-Electronics-Crystal research}) Marī Kirī tīkla projekta ietvaros. Izsaku pateicību Etjenetei Ofrē (\textit{oriģ.} Etiennette Auffray) (CERN, Šveice) par šī projekta organizēšanu. Pateicos arī Mikelem Galinaro (\textit{oriģ.} Michele Gallinaro) (LIP, Portugāle) par ievadu darbam KMS eksperimentā.

Pateicos arī saviem komandas biedriem Marteinam Muldersam (\textit{oriģ.} Martijn Mulders) (CERN, Šveice), Pedru Silvam (\textit{oriģ.} Pedro Silva) (CERN, Šveice) un Markusam Zeidelsam (\textit{oriģ.} Markus Seidel) (CERN, Šveice) par metodoloģisko atbalstu un dalīšanos ar pieredzi.

Tāpat pateicos Rīgas Tehniskajai universitātei par nemitīgo atbalstu, kā rezultātā radās iespēja turpināt darbu KMS eksperimentā.


\tableofcontents
\printglossary[toctitle=Terminu tulkojumi, title=Latviešu-angļu terminu tulkojumi]
%% \listoffigures
%% \listoftables
\part{VISPĀRĒJS DARBA RAKSTUROJUMS}
%\chapter{Ievads}
\label{chap:introduction}
\vskip 1.5cm
\medskip
\textbf{Fizikālie pamati}
\nopagebreak\medskip

Lielais hadronu paātrinātājs (LHP) ir 27~km apkārtmēra sinhrotrons. Tas atrodas Ženēvas apkārtnē uz Francijas-Šveices robežas. LHP eksperimentālajos punktos tiek veiktas protonu-protonu (\Pp\Pp) sadursmes. Lielākā daļa šo sadursmju ir neelastīga, un ar detektoriem kā KMS tiek analizēti sadursmju gruveši. Sadursmju gruvešos ceram atrast atbildes uz daudziem fizikas jautājumiem, kā, piemēram, Higsa bozona eksistence un īpašības, Tumšās matērijas eksistence un virsotnes kvarka īpašības.

\Pp\Pp sadursmju masas centra enerģija ir 13~\TeV. Šāda enerģija ir pietiekama, lai radītu miljoniem \ttbar pāru - šī procesa šķērsgriezums pie $\sqrt{s}=13~\TeV$ ir 803~pb~\cite{Sirunyan:2018goh}. LHP ir uzskatāms par \gls{virsotnes kvarku} fabriku. Virsotnes kvarks sabrūk vājajā ceļā, izstarojot \PW bozonu. \PW bozons pieder pie krāsu singleta, tāpat arī tā sabrukuma produkti. Ja \PW bozons sabrūk par krāsainiem produktiem (kvarkiem), tad šie produkti savā starpā mijiedarbojas hromodinamiskajā laukā - tie būs saistīti ar krāsām. Darba uzdevums ir pētīt ar krāsām saistīto strūklu eksperimentālos nospiedumus.

Vieglie kvarki, kas rodas, sabrūkot \PW bozonam, hadronizējas un detektorā novērojami kā \gls{strūklas}. KMS silīcija trekeris, elektromagnētiskais un hadronu kalorimetrs ļauj izšķirt strūklu sastāvdaļas jeb hadronizācijas produktus (barionus un mezonus). Papildus, KMS 4~T supravadošā solenoīda magnētiskais lauks ļauj mērīt lādēto strūklu sastāvdaļu momentu ar augstu izšķirtspēju. Daļiņas tiek identificētas un to parametri tiek mērīti, apkopojot novērojumus dažādos apakšdektoros~\cite{Sirunyan:2017ulk}. Gadījumā, ja strūklas ir saistītas ar krāsām, to sastāvdaļām ir tieksme aizpildīt telpu starp strūklām eksperimenta inerciālajā sistēmā. Šī īpašība ir par pamatu darbā izmantotajām metodēm.

Tāpat pētām arī hipotētiska krāsu okteta \PW bozona sabrukšanu. Šajā gadījumā vieglās strūklas nav saistītas ar krāsām, un šos rezultātus varam izmantot, lai tos salīdzinātu ar ar krāsām saistīto gadījumu.

Atlasām notikumus atbilstoši $tt-\cPqb\PW(q_1q_2)\cPqb\PW(\ell\nu)$ topoloģijai - tādus, kuros ir 2 vieglās strūklas, 2 \cPqb-atzīmētās strūklas, kā arī viens lādēts leptons.

Darbā tiek izmantoti 2016. gada LHP KMS dati ar integrēto spīdumu 35,9~\fbinv. Eksperimenta novērojumi tiek salīdzināti ar Monte Karlo (MK) simulācijām. MK simulācijas ļauj novērtēt fona klātbūtni, notikumu atlases efektivitāti un mūsu gadījumā arī pārliecināties, cik precīzi ir mūsu hadronizācijas modeļi. Centrālais process tiek simulēts ar \POWHEG, bet hadronizācija ar \PYTHIA. \PYTHIA simulētā hadronizācija tiek salīdzināta ar \HERWIGpp simulēto hadronizāciju. KMS detektors tiek modelēts ar \GEANTfour. Ievērojot atšķirības starp MK simulācijām un detektora novērojumiem, MK paraugiem tiek pielietoti attiecīgi koeficienti. Tiek novērtēta arī dažādu sistemātisko nenoteiktību radītā kļūda.

Ievērojot slikto treku rekonstrukcijas efektivitāti daļiņām, kuru šķērsmoments ir mazāks par 1~\GeV, pētījumā iekļaujam tikai daļiņas, kuru šķērsmoments ir liekāks nekā 1~\GeV.
 
\medskip
\textbf{Metodes}
\nopagebreak\medskip

Izmantojam vilkmes leņķa metodi~\cite{Gallicchio:2010sw}. Atbilstoši šai metodei tiek konstruēts vilkmes vektors, zinot strūklas centru un strūklas sastāvdaļu attālumu no tā, kas svērts ar strūklas sastāvdaļu šķērsmomentu \pt. Sagaidāms, ka strūklas vilkmes vektors rādīs uz citu strūklu, kas ar šo strūklu saistīta krāsu laukā. Tātad sagaidāms, ka vilkmes leņķa sadalījumā būs novērojams paugurs ar centru 0~rad.

Pētām vilkmes leņķa sadalījumu starp ar krāsām saistītām strūklām (abas vieglās strūklas), kā arī salīdzinām rezutātus ar vilkmes leņķa sadalījumu starp ar krāsām nesaistītiem fizikāliem objektiem - \cPqb-atzīmētām strūklām, vieglo strūklu un leptonu. Interesants gadījums ir vilkmes leņķis starp strūklu un kūli.

Šķirojam gadījumus kad, \DeltaR starp strūklām ir lielāks vai mazāks par 1. Pēdējā gadījumā anti-$k_{T}$ strūklu sakopošanas algoritms inducē vilkmi no vadošās strūklas uz mīkstāko strūklu, radot būtisku ietekmi uz novērojumiem saskaņā ar vilkmes metodi.

Novērtējam vilkmes leņķa metodes jutīgumu pret dažādiem parametriem - tikai lādēto daļiņu izmantošanu (tikai lādēto daļiņu trajektorijas tiek noliektas magnētiskajā laukā), \PW bozona šķērsmomentu, strūklas sastāvdaļu skaitu, strūklas sastāvdaļu šķērsmomenta slieksni, vilkmes vektora lielumu.

Lai novērstu detektora radīto ietekmi uz novērojumiem, lietojam atlocīšanas metodi. Šī metode ļauj iegūt patiesā novērojamā lieluma sagaidāmā sadalījuma novērtējumu, taču tās trūkums ir augstā novērojamā lieluma fāžu telpas granularitāte. Novērtējam atbilstību starp atlocītajiem novērojumiem un ģenerētajiem Monte Karlo novērojumiem, kā arī dažādo sistemātisko nenoteiktību ietekmi.  

Tāpat tiek izmantota arī adaptācija metodei, kas tikusi izmantota Lielajā elektronu-pozitronu kolaiderī LEP (turpmāk to dēvēsim par ``LEP metodi''), kur strūklu sastāvdaļas tiek projicētas uz starpstrūklu plaknēm~\cite{Abbiendi:2005es}, \cite{Abdallah:2006uq}, \cite{Achard:2003pe}. Sagaidāms, ka plakne starp ar krāsām saistītām strūklām būs blīvāk aizpildīta ar strūklu sastāvdaļu projekcijām nekā plakne starp ar krāsām nesaistītām strūklām.  

Rezultāti tiek iegūti, izmantojot \CMSSW versiju \lstinline[language=sh]|CMSSW_8_0_26_patch1|, sākotnēji arī \RIVET~\cite{Buckley:2010ar}.

Visbeidzot, veicam hipotēžu pārbaudi. Šī uzdevuma ietvaros kombinējam \ttbar signālu ar krāsu okteta \PW signālu un novērtējam šīs kombinācijas atbilstību datiem.

\medskip
\textbf{Novitāte}
\nopagebreak\medskip

Vilkmes leņķa metode ir tikusi pielietota Fermilab Tevatrona \DZERO eksperimentā~\cite{Abazov:2011vh}, ATLAS eksperimentā I darba periodā~\cite{Aad:2015lxa}, kā arī ATLAS eksperimentā II darba periodā \cite{Aaboud:2018ibj}. Šo metodi CMS pirmoreiz pielietoja Zeidels, M. un citi~\cite{indico:Markus_cf}, taču šie rezultāti nekad nav tikuši publicēti. Salīdzinājumā ar ATLAS KMS detektora iespējama aptuveni divreiz labāka centrālā reģiona treku momenta izšķirtspēja, pateicoties tā 4~T solenoīdam (ATLAS aprīkots ar daudz mazāku 2~T solenoīdu ar lieliem toroīda magnētiem ārpusē~\cite{Aad:2008zzm}).

``LEP metode'' vēl nav tikusi pielietota LHP.

Jāatzīmē, ka šis darbs ir pirmais Latvijas pienesums LHP eksperimentālajai programmai. Darbam, kas aprakstīts šajā disertācijā, noritot pilnā sparā, mēs 2018. g. maijā svinīgi atzīmējām uzņemšanu par pilntiesīgiem KMS eksperimenta biedriem.

\medskip
\textbf{Aprobācija}
\nopagebreak\medskip


Šajā disertācijā ir izklāstīti KMS eksperimenta virsotnes kvarka grupas pētījuma ietvaros gūtie rezultāti. Rezultāti dažādās stadijās ir tikuši prezentēti Virsotnes modelēšanas un ģeneratoru fizikas sanāksmēs - 2016. g. 19. janvārī, 2016. g. 29. martā, 2016. g. 7. jūnijā, 2016. g. 30. augustā, 2018. g. 13. februārī un 2018. g. 17. oktobrī. Tie ir tikuši arī prezentēti CERN Zinātnes nedēļā Rīgā no 2017.g 22.-26. maijam, kā arī Augstas enerģijas fizikas Eiropas Skolā Evurā, Portugālē no 2017. g. 6. - 19. septembrim.

Šajā pētījumā iegūtie rezultāti vēl nav apstiprināti saskaņā ar KMS eksperimenta Sadarbības padomes pieņemtajiem noteikumiem par rezultātu apstiprināšanu un publicēšanu. Saskaņā ar šiem noteikumiem pētījuma rezutāti vēl nevar tikt publicēti recenzētos zinātniskos izdevumos un oficiālās konferencēs.


\part{ATSEVIŠĶO NODAĻU IZKLĀSTS}
\chapter{Fizikālie pamati}
Virsotnes kvarka pāra šķērsgriezums protonu-protonu sadursmēs ar \sqrts=13\TeV saskaņā ar mērījumiem ir 803 pb \cite{Sirunyan:2018goh}. Šķērsgriezums palielinās, pieaugot masas centra enerģijai. Šķērsgriezums kā funkcija no masas centra enerģijas ir attēlots \ref{fig:tt_curve_toplhcwg_sep18} att..

\begin{figure}[hbtp]

  \centering
  \includegraphics[width=0.6\textwidth]{fig/tt_curve_toplhcwg_sep18.pdf}
  \caption{Virsotnes pāra šķērsgriezums pie dažādas masas centra enerģijas \cite{twiki:tt_curve_toplhcwg_sep18}. Grafikā ir attēlots šķērsgriezums \Pp\Pp un \Pp\Pap sadursmēs, kā arī ir attēloti KMS un ATLAS mērījumi dažādos \ttbar pāra sabrukšanas kanālos.}
  \label{fig:tt_curve_toplhcwg_sep18}
  
\end{figure}

Izmantojot sakarību

\begin{equation}
N=\sigma\int L(t)dt
\end{equation}

pie 35,9\fbinv integrētā spīduma ir sagaidāms, ka radīsies $26,7\times10^{6}$ šādi pāri. 

LHP 2 protoni saduras ar enerģiju, kas ir pietiekami liela, lai ``saspiestu'' protonus tik cieši kopā līdz kvarki vienā protonā spēj mijiedarboties ar kvarkiem otrā protonā. Tie mijiedarbojas, apmainoties ar gluonu. Šādā apmaiņā var rasties virsotnes kvarka-antikvarka pāris. \ref{fig:top_quark_productions} att. attēloti 2 šādi iespējamie procesi. Apmainītais gluons ir tik enerģētisks, ka tas spēj sašķaidīt protonu \gls{gruvešos}. Šāda sadursme tiek saukta par neelastīgu.

\begin{figure}[h!]
  \centering
  \def\twidth{0.3}
  \subfloat[Pāra radīšana.]{%
    \includegraphics[width=\twidth\textwidth]{fig/top_quark_pair_prod_gfusion}%
    \label{fig:top_quark_production}
  }\hfil
  \subfloat[Gluonu saplūšana.]{%
    \includegraphics[width=\twidth\textwidth]{fig/top_quark_pair_prod_gluon}%
    \label{fig:top_quark_production2}
  }
  %% \subfloat[Smago kvarku saplūšana. Punktotā līnija var būt jebkurš elektriski neitrāls bozons.]{
  %%   \includegraphics[width=\twidth\textwidth]{fig/top_quark_pair_prod_qqbar}
  %%   \label{fig:top_quark_production3}
  %% }

  \caption{Virsotnes kvarka pāru radīšana \Pp\Pp sadursmē.}
  \label{fig:top_quark_productions}
\end{figure}

Virsotnes kvarks sabrūk tikai vājajā procesā (\ref{fig:quark_decay} att.). Vājajā sabrukumā tiek izstarots \PW bozons un citas \gls{smaržas} kvarks ar elektriskā lādiņa lielumu  $\frac{1}{3}e$. 

\begin{figure}[H]
  \centering
  \includegraphics[width=0.3\textwidth]{fig/fig_top_quark_decay.pdf}
  \caption{Vājais virsotnes kvarka \cPqt sabrukums. $k$ un $k'$ ir fermioni, kas rodas, sabrūkot \PW bozonam.}
  \label{fig:quark_decay}
\end{figure}

Vidējie CDF, \DZERO Tevatrona eksperimentu mērījumi \cite{Aaltonen:2015cra}, kā arī ATLAS un KMS LHP eksperimentu mērījumi \cite{twiki:tt_curve_toplhcwg_sep18} novērtē Kabibo-Kobajaši-Maskavas matricas $|V_{tb}|$ komponentes vērtību vienādu ar

\begin{equation}
  |V_{tb}|=1,009\pm0,031.
\end{equation}

Tas nozīmē, ka virsotnes kvarks jaucas ar \cPqb kvarku $(0.98)^{2}$ no visiem gadījumiem. Pārēji KKM matricas 3 kolonas un 3 rindas elementi ir ļoti nelieli \cite{Patrignani:2016xqp}:

\begin{align}
  & |V_{td}|=8,4\times10^{-3}, && |V_{ts}|=40,0\times10^{-3}.
\end{align}

Virsotnes kvarka platums saskaņā ar \DZERO kolektīva mērījumu \cite{Abazov:2010tm} ar 2,3 \fbinv integrēto spīdumu ir $\Gamma=1,99^{0.69}_{-0.55}$ GeV. Tas atbilst mūža ilgumam $\tau_{t}=3,3\times10^{-25}\text{s}$.

Šāds mūža ilgums ir mazāks nekā hadronizācijas laiks ($1/\Lambda$ $\sim$ $10^{-24}\text{s}$), kur $\Lambda^{2}$ ir apmainītā gluona $Q^{2}$ enerģija, pie kura stiprā saites koeficienta $\alpha_{s}$ vērtība kļūst vienāda ar $\sim$ 1, kas ir tuvu tā asimptotiskajai vērtībai pie \gls{norobežojuma barjeras}. Līdz ar to virsotnes kvarks sabrūk, pirms tas hadronizējas, un eksperimentētājam paveras vienreizēja iespēja īsu laika sprīdi novērot ``kailu'' kvarku.

Virsotnes kvarka mūža ilgums ir mazāks arī par virsotnes kvarku pāra spina dekorelācijas laika posmu - $M/{\Lambda^{2}}=3\times 10^{-21}\text{s}$. Tātad virsotnes kvarku pāris saglabā savus spina stāvokļus pirms tas sabrūk un nodod savus spina stāvokļus sabrukuma produktiem \cite{Cristinziani:2016vif}.

Virsotnes kvarka sabrukšanas zarojuma attiecības pēc būtības ir tādas pašas kā \PW bozona sabrukuma procesam. \PW bozons sabrūk par leptonu pāriem (\Pe\Pgne, \Pgm\Pgngm, \Pgt\Pgngt) vai kvarku pāriem \cPqu, \cPqd' un \cPqc, \cPqs' (apostrofs norāda uz to, ka smaržas simetrija netiek saglabāta precīzi). Taču kvarku pāriem var būt 3 krāsas. Līdz ar to kopējais satāvokļu skaits ir $3+2\times3=9$. \PW bozona sabrukšanas zarojuma attiecību vienkāršs novērtējums un tā eksperimenta ceļā iegūtie mērījumi ir sniegti \ref{tab:W_br} tab.

\begin{table}[h!]
  \centering
  \caption{\PW bozona sabrukšanas zarojuma attiecības.}
  \label{tab:W_br}
  \begin{tabular}{l r r}
    Zars                  & $\Gamma_{j}/\Gamma$ & $\Gamma_{j}/\Gamma$\\
                          & vienkāršojums       & novērojums \cite{Patrignani:2016xqp}\\
    \hline
    $e\nu_{e}$            & $\frac{1}{9}$       & (10,71 $\pm$ 0,16) \%\\
    $\mu\nu_{\mu}$        & $\frac{1}{9}$       & (10,63 $\pm$ 0,15) \%\\
    $\tau\nu_{\tau}$      & $\frac{1}{9}$       & (11,38 $\pm$ 0,21) \%\\
    kvarku pāris          & $\frac{2}{3}$       & (67,41 $\pm$ 0,27) \%
  \end{tabular}
\end{table}

\PW bozona hadroniskajā sabrukšanas ceļā tiek izstarotas ar krāsām saistītas strūklas (\ref{fig:ttbar_cf} att.). Kvarkiem, kas rada šīs strūklas, ir pretēji vērsti momenti masas centra inerciālajā sistēmā. Kvarkiem attālinoties vienam no otra, to kinētiskā enerģija tiek atdota krāsu laukam. Krāsu laukā esošā papildu enerģija, kas līdzvērtīga apmēram $m_{\PW}$ (80,4 \GeV), tiek izlietota jaunu daļiņu radīšanai. Vienkāršots jaunu hadronu radīšanas process ir attēlots \ref{fig:combination} att., kas balstīts uz Lundas modeli \cite{Andersson:1983ia}. Alternatīvs atainojums, kas balstīts Fainmana diagramās, ir sniegts \ref{fig:colour_field} att.

\begin{figure}[htp]
  \centering
  \includegraphics[width=0.8\textwidth]{fig/combinationlv.pdf}
  \caption{Process, kurā divi enerģētiski kvarki rada jaunus hadronus.}
  \label{fig:combination}
\end{figure}

  \begin{figure}[hbtp]
    \centering
    \includegraphics[width=1.0\textwidth]{fig/colour_field_fulllv.pdf}
    \caption{Hadronu radīšana divu kvarku krāsu laukā.}
    \label{fig:colour_field}

  \end{figure}

Hadroniskajā \PW bozona sabrukšanā rodas šādas daļiņu sugas:

  \begin{table}[h!]

    \centering
    \begin{tabular}{ l l l l }
      \textbf{daļiņa}  & \textbf{masa} [GeV]  & \textbf{mūža ilgums} [s] & \textbf{novērojamais signāls}\\
      \Pgpz              & 135,0               & $8,5\times10^{-27}$  & 2\cPgg absorbēti ECAL\\
      \Pgppm             & 139,6               & $2,6\times10^{-8}$   & \gls{trekeris}, ECAL, HCAL \gls{lietus}\\
      \PKzS              & 497,6               & $8,95\times10^{-11}$ & ECAL, HCAL lietus\\
      \PKzL              & 497,6               & $5,1\times10^{-8}$   & ECAL, HCAL lietus\\
      \PKpm              & 493,7               & $1,2\times10^{-8}$   & trekeris, ECAL, HCAL lietus\\
      \Pn                & 939,6               & $881,5$              & ECAL, HCAL lietus\\
      \Pp                & 938,3               & $\infty$             & trekeris, ECAL, HCAL lietus\\
    \end{tabular}
    \caption{Jaunās daļiņas, kas rodas krāsu laukā starp enerģētiskiem ar krāsam saistītiem kvarkiem, kuri ir izstaroti, hadroniski sabrūkot \PW bozonam.}
    \label{tab:particles}

  \end{table}

  \begin{figure}[hbtp]

    \centering
     \def\twidth{0.45}
    \includegraphics[width=\twidth\textwidth]{fig/ttbar_cf_cropped.pdf}
    \caption{Krāsu plūsmas virsotnes kvarku pāra sabrukuma procesā.}
    \label{fig:ttbar_cf}
    
  \end{figure}

  Attiecīgās rezonanses ir skaidri saskatāmas ģeneratora līmenī (\ref{fig:mass_resonances} att.). Tikai neitrālais pions sabrūk pirms to var tiešā veidā novērot detektorā.

  \begin{figure}[hbtp]
    \centering
     \def\twidth{0.45}
    \includegraphics[width=\twidth\linewidth]{fig/histos/L/gen/charge/allconst/L_JetConst_M_allconst_gen_leading_jet.png}
    \caption{\protect\ref{tab:particles}. tab. norādīto daļiņu rezonanses. \\
    \footnotesize Piezīme: Šis grafiks, kā arī vairāki turpmākie grafiki atbilst KMS pieņemtajam formātam, kā attēlot novērojamo lielumu skaitīšanas eksperimentā. Šī formāta paskaidrojumi ir sniegti \protect\ref{chap:results} nod.}
    \label{fig:mass_resonances}
  \end{figure}

\ref{fig:number} att. ir attēlots daļiņu skaita sadalījums, kas veido vadošo vieglo strūklu. \ref{fig:charged_content} att. ir attēlota attiecība starp elektriski lādēto daļiņu skaitu pret kopējo daļiņu skaitu. Vadošā vieglā strūkla ir tā strūkla, kas izstarota no \PW bozona sabrukuma procesa, kurai ir vislielākais šķērsmoments.

  \begin{figure}[hbtp]
    \centering
     \def\twidth{0.45}
    \includegraphics[width=\twidth\linewidth]{fig/histos/L/reco/charge/allconst/L_JetConst_N_allconst_reco_leading_jet.png}
    \caption{Kopējais daļiņu skaits, kas veido vadošo vieglo strūklu.}
    \label{fig:number}

\end{figure}
% \begin{linenomath}
     \begin{figure}[hbtp]
     \centering
     \def\twidth{0.45}
     \includegraphics[width=\twidth\linewidth]{fig/histos/L/reco/L_JetConst_EventChargedContentN_reco_leading_jet.png}
     \caption{Lādēto daļiņu skaita attiecība pret kopējo daļiņu skaitu, kas veido vadošo vieglo strūklu.}
  \label{fig:charged_content}
   \end{figure}
 % \end{linenomath}

Tā kā mēs pētām viegās strūklas, kas radušās \PW bozona sabrukšanas rezultātā, rodas jautājums, kādēļ strādājam ar \ttbar procesa paraugiem. \PW bozona radīšanas šķērsgriezums ir $>20\times$ lielāks nekā \ttbar šķērsgriezums. Pētījumā mums jāizmanto $\PW\rightarrow \cPq\cPq'$ notikumi, jo leptoniskajos sabrukumos nav krāsu plūsmas. $\PW\rightarrow \cPq\cPq'$ notikumus ar pietiekoši zemu \pt slieksni ir sarežģīti izmantot trigera palaišanai. Tādēļ mēs izmantojam \ttbar notikumus, kur viens no \PW bozoniem sabrūk leptoniski, kas tiek izmantots trigera palaišanai, bet otrs \PW bozons sabrūk hadroniski, kas tiek izmantots krāsu plūsmas pētīšanai.

\PW bozons pieder pie krāsu singleta:

\begin{equation}
\frac{1}{\sqrt{3}}\left(\text{R}\overline{\text{R}}+\text{G}\overline{\text{G}}+\text{B}\overline{\text{B}}\right),
\end{equation}

kur $R$, $G$ un $B$ ir trīs krāsu viļna funkcijas kvantu stāvokļi.

Objets, kas pieder pie krāsu singleta, ir bezkrāsas un tas nevar piedalīties stiprajā mijiedarbībā. Mēs pieminam šo īpašību, jo tālāk aprakstīsim krāsu okteta \PW bozonu.

Tiek pieņemts pie krāsu okteta piederošs \PW bozons. Tā krāsu viļna funkcijas var pieņemt kādu no 8 kombinācjām:

\begin{align}
\text{R}\overline{\text{G}}, &&
\text{R}\overline{\text{B}}, &&
\text{G}\overline{\text{R}}, &&
\text{G}\overline{\text{B}}, &&
\text{B}\overline{\text{R}}, &&
\text{B}\overline{\text{G}}, &&
\frac{1}{\sqrt{2}}\left(\text{R}\overline{\text{R}}-\text{G}\overline{\text{G}}\right), &&
\frac{1}{\sqrt{6}}\left(\text{R}\overline{\text{R}}+\text{G}\overline{\text{G}}-2\text{B}\overline{\text{B}}\right).
\end{align}

Vienīgā dabā novērotā daļiņa, kas pieder pie krāsu okteta ir gluons. Krāsains \PW bozons ir pilnībā hipotētiska daļiņā. Tiek pieņemts, ka krāsu okteta \PW bozona masa ir vienāda ar $m_{\PW}$. Šīs bozons sasaistītu krāsu laukā vieglos kvarkus ar hadronisko \cPqb un hadronisko \cPqt. Tikmēr vieglie kvarki tiktu atsaistīti viens no otra (\ref{fig:ttbar_cf_octet} att.).
  
  \begin{figure}[h!]
  \centering
  \includegraphics[width=0.4\textwidth]{fig/ttbar_cf_flip_cropped.pdf}
  \caption{Krāsu plūsma virsotnes kvarku pāra sabrukuma procesā, pieņemot hipotētisku krāsu okteta \PW bozonu.}
  \label{fig:ttbar_cf_octet}
\end{figure}



\chapter{Eksperimentālais aprīkojums}
\label{sec:experimental_setup}

Šis pētījums ir veikts ar iespējams vienu no vissarežģītākajiem un vislielākajiem zinātnes aprīkojumiem cilvēces vēsturē, kuru izmanto un apkalp viens no visglobālākajiem kolektīviem pētniecībā. LHP un tā eksperimenti tika iecerēti un izveidoti, lai atbildētu uz fundamentāliem jautājumiem fizikā:

\begin{itemize}
\item Pētīt elektrovājo simetrijas laušanu un meklēt Higsa bozonu. Higsa bozona pastāvēšana tika paredzēta teorētiski 1964.g.  \cite{Higgs:1964ia} \cite{Englert:1964et} un ilgu laiku tas bija iztrūkstošais akmentiņš Standarta modeļa mozaīkā. Ja to atklātu, tad tiktu apstiprināti mūsu priekštati par subatomāro pasauli. Par attiecīgo atkājumu vienlaicīgi paziņoja KMS in ATLAS 2012. gadā \cite{Chatrchyan:2012xdj} \cite{Aad:2012tfa} pēc gandrīz 50 gadus ilgiem meklējumiem.
\item Petīt Standarta modeļa fiziku ar līdz šim neredzētu precizitāti, izmantojot vismodernākos detektorus, lielu integrēto spīdumu un lielu masas centra enerģiju. Viena no visintersantākajām tēmām ir nesen atklātā virsotnes kvarka pētniecība. Dēļ savas lielās masas, tas labi saistās ar Higsa bozonu.
\item Izveidot apstākļus senatnīgās kvarku-gluonu plazmas pastāvēšana, šādi rodot atbildes uz fundamentāliem jautājumiem par mūsu Visuma evolūciju.
\item Meklēt Tumšo matēriju, eksotiskās daļiņas, supersimetriskos partnerus, \gls{papildus dimensijas}, kā arī risināt citas mīklas un hipotēzes ārpus Standara modeļa. Uz šiem jautājumiem joprojām nav rastas pārliecinošas atbildes un to dēļ ir iecerēts radīt \gls{Augsta spīduma LHP}, \gls{Nākotnes apļveida kolaideri} un citas lielizmēra eksperimentālas iekārtas. 
\end {itemize}

KMS eksperiments ir viens no LHP pamata eksperimentar. Tādēļ šajā aprakstā LHP tiks stādīts priekšā pirmais, kam sekos KMS aprīkojuma apraksts.

\section{LHP}

LHP ir divapļu supravadošs hadronu paātrinātājs un kolaiders, kas ierīkot 26,7 km garā tunelī 45-170 m zem zemes, šķērsojot Francijas un šveices robežu Ženēvas apkārtnē - sk \ref{fig:LHC_underground}. zīm. Hadronu riņķo ar nemainīgu radiusu bet mainīgu frekvenci. Tātad LHP ir sinhrotrons. LHP atkārtoti izmanto Liela elektronu-pozitronu paātrinātāja tuneli un \gls{ievades ķēdi}.

Sakotnēji LHP projekts sastapās ar sīvu konkurenci no daudz jaudīgākā Supravadošā superkolaidera ASV. K. Rubia (\textit{orig.} C. Rubbia) apgalvoja, ka 10 reizes lielāks spīdums LHP kompensētu tās mazāko enerģiju, salīdzinājumā ar SSK. Galu galā 1993. g. SSK projekts tika atcelts. Būtisku lomu spēlēja ievērojamais budžeta pieaugums. CERN padome LHP projektu apstiprināja 1994.gadā. Tas sāka iegūt datus 2008. g.

\begin{figure}[htpb]
  \centering
  \includegraphics[width=0.8\textwidth]{fig/LHC_underground.png}
  \caption{Lielais hadronu paātrinātājs, kas atrodas pazemē uz Francijas-Šveices robežas Ženēvas apkārtnē.}
  \label{fig:LHC_underground}
\end{figure}

Protoni LHP riņķo praktiski ar gaismas ātrumu. Enerģija uz protonu ir 7 TeV, $\gamma$ faktors ir 7461. Paātrināt protonu no miera stāvokļa līdz šādai enerģijai vienā paātrinātājā nav praktiski. Tādēļ, līdz šādas enerģijas sasniegšanai protoni tiek paātrināti vairākās stadijas CERN paātrinātāju kompleksā - sk \ref{fig:CERN_accelerator_complex} att.:

\begin{itemize}
\item līdz 50 MeV Linac2
\item līdz 1.4 GeV PS \gls{pastiprinātājā}
\item līdz 26 GeV Protonu sinhrotronā (PS)
\item līdz 450 GeV Superprotonu sinhrotronā (SPS)
\end{itemize}

\begin{figure}[htpb]
  \centering
  \includegraphics[width=1\textwidth]{fig/CERNacceleratorcomplex.jpg}
  \caption{CERN paātrinātāju komplekss.}
  \label{fig:CERN_accelerator_complex}
\end{figure}

Pēc tam, kad protoni ir tikuši pilnībā paātrināti, to riņķveida kustība tiek uzturēta - LHP ir \gls{uzkrāšanas aplis}. Vienā \gls{pūlī} ir sakopoti $1.15\times10^{11}$ protoni, kopā riņķo 2808 pūļi. Apgrieziena frekvence ir 11.245 kHz \cite{Bruning:2004ej}. Katra pūļa šķērsojums ilgts 25 ns. Kūļa caurulēs tiek uzturēts ārkārtīgi augsts vakuums.

LHP izmanto supravadošu magnētu sistēmas. Īpaši izceļams ir dipola magnēts, kas kūli ieloka apļveida arkā un kvadrupola magnēts, kas kūli sašaurina pirms sadursmes punktiem. Augstākas pakāpes magnēti precizē kūļa kustību un koriģē to. Magnētu sistēmas balstās uz NbTi Ruzerforda kabeli, kurš ir atdzesēts ar heliju zem 2 K - zem helija lambdas punkta \footnote{Helija lambdas punkts ir temperatūra, pie kuras normāls helijs pāriet suprašķidruma stāvoklī.}. Līdz ar to atšķirībā no citiem lieliem paātrinātājiem, kas arī izmanto NbTi, bet strāda virs helija lambda punkta (Tevatron-FNAL, HERA-DESY un RHIC-BNL), LHP dipolu magnētos tiek iegūts daudz stiprāks 8T lauks. Priekš LHP tika izstrādāts īpašs divi-vienā dipola magnēts, kurā tiek izmantos viens jūgs divu polaritāšu laukiem protonu kūļiem, kas riņķo pretējos virzienos. Lai atdzesētu magnētus, tiek izmantota vislielākā kriogenā sistēma uz Zemes \cite{MYERS:2013hra} \cite{Evans:2008zzb}.

LHP nominālā masas centra enerģija ir 14 TeV. Pirmajā datu gūšanas periodā no 2010. - 2013. g. tas strādāja ar \sqrts=7-8 TeV. Šo periodu sauc par I \gls{darba periodu}. Otrajā datu gūšanas periodā no 2015. - 2018. g., ko sauc par II darba periodu, tas strādāja ar \sqrts=13-14 TeV. Sis pētījums ir veikts a ar II darba perioda datiem.

LHP ietilpst divi liela spīduma ekpserimentālie ievietojumi - KMS un ATLAS, katram no kuriem nominālais spīdums ir virs $10{\frac{1}{pb\cdot s}}$, vienu \cPqb fizikas eksperimentu, kura nominālais spīdums ir $0.1\frac{1}{pb\cdot s}$ un vienu speciālu jonu sadursmju eksperimentu - ALICE. 

\section{KMS detektors}

KMS detektors atrodas LHP 5. punktā netālu no franču ciemata Sesī (\text{orig.} Cessy)) starp Ženēvas ezeru un Žura kalniem (\textit{orig} Jura). Tas atrodas pazemes bunkuros 100 dziļumā. 

KMS detektors ir projektēts dažādām fizikas programmām TeV skalā. Tas ir sīpola tipa detektors aptverot $4\pi$ telpiskā leņķa ap sadursmes punktu. KMS detektoru veido sekojoši slāņi sākot no kūļa ass - silīcija pikseļu un \gls{strēmeļu} trekeris, vara volframāta elektromagnētiskais kalorimetrs, misiņa un plastikāta scintilators, supravadošs magnēts, kas rada 3.8 - 4.0 T magnētisko lauku, un gāzu jonizācijas mionu spektrometrs \cite{Chatrchyan:2008aa}. KMS detektoram ir cilindra forma. Tā abos galos ir \gls{gala segumi}, bet tā centrālā daļa tiek saukta par \gls{mucu}. KMS detektora garums ir 21,6 m, diametrs 14,6 m un kopējais svars 12 500 t. KMS detektora atvērums attēlots \ref{fig:CMS_detector}. att.

\begin{figure}[hbtp]
\centering
\def\twidth{1}
\includegraphics[width=\twidth\textwidth]{fig/cms_120918_03}
\caption{KMS detektora atvērums \cite{Sakuma:2013jqa}.}
\label{fig:CMS_detector}
\end{figure}


Sākot no kūļu mijiedarbības reģiona, daļiņas vispirms nonāk trekeri, kurš no signāliem (\gls{ierosinājumiem}) jutīgajos slāņos rekonstruē lādēto daļiņu trajektorijas un sākuma punktus (\gls{virsotnes}). Trekeris atrodas magnētiskā laukā, kas noliec lādēot daļiņu trajektorijas un ļauj izmērīt to elektrisko lādiņu un momentus. Elektroni un fotoni pēc tam tiek absorbēti elektromagnētiskajā kalorimetrā (ECAL). Atiecīgie elektromagnētiskie lieti tiek detektēti kā enerģijas \gls{puduri} vienkopus esošas šūnās, no kuriem var izmērīt daļiņu enerģiju un virzienu. Lādēti un neitrāli hadroni ECAL var ierosināt hadronu lietu, kuru pēc tam pilnībā absorbē hadroniskais kalorimetrs (HCAL). Attiecīgie puduri tiek izmantotu, lai novērtētu to enerģiju un virzienu.  Mioni un pioni sķērso kalirometros ar vāju vai nekādu mijiedarbību. Kamēr neitrino izbēg nedetektēti, mioni rada ierosinājumus aiz kalorimetriem esošos papildus trekēšanas slāņos, kurus sauc par mionu detektoriem. Šis vienkāršotais apskats ir atspoguļots \ref{fig:CMSpf}. att., kurā ir attēlot KMS detektora loksne.

Ievērojami uzlabots notikumu apraksts tiek iegūts, atrodot sakarības starp novērojumiem dažādos detektora slānos (trekus un pudurus) lai identificētu katru \gls{gala stāvokļa} daļiņu, un apkopojot to mērījumus, la rekostruētu daļiņas īpašības, balstoties uz šo identifikāciju. Šī visaptverošo pieeju sauc par \gls{daļiņu plūsmas} (PF) rekonstukciju \cite{Sirunyan:2017ulk}.

\begin{figure}[h]
  \centering
  \includegraphics[width=1\textwidth]{fig/CMSpf.png}
  \caption{Daļiņu mijiedarbību skice KMS šķērsgriezumā no kūļa mijiedarbības punkta līdz mionu detektoram.}
  \label{fig:CMSpf}
\end{figure}

Smalkās \gls{graudainības} un ātrās atbildes trekeris (\cite{Karimaki:368412}, \cite{tracker_addendum}) ir svarīgs segments smalko kūļa sastāvdaļu izšķiršanā. Tas ir precīzi novietots līdz ar kūļa asi un tā kopējais garums ir 5.8 m, radiuss 2.5 m. KMS soleonīds rada homogēnu un koaksiālu 3,8 - 4,0 T magnētisko lauka visā terekera tilpumā. Pie radiusa zem 10 cm ierosinājumu biežums ir ar pakāpi $100\frac{kHz}{\text{mm}^2}$. Lai sasniegtu vēlamo izšķirtspēju 100 $\mu$m $\times$ 150 $\mu$m, tiek izmantoti pikseļu detektori. Pie lielāka radiusa samazinātā daļiņu plūsma pieļauj silīcija mikrostrēmeļu detektoru izmantošanu. To tipiskais izmēris ir 10 cm $\times$ 80 $\mu$m līdz 25 cm $\times$ 150 $\mu$m, izmēram pieaugot ar pieaugošu radiusu. Kopā ir 66 milj. pikseļu ar $\frac{1}{\text{m}^2}$ aktīvo virsmu pikseļu detektorā un 9.3 milj. strēmeļu ar 193 m${}^2$ aktīvo virsmu strēmeļu detektorā.

KMS elektromagnētiskais kalorimetrs (ECAL) ir hermētisks homogens kalorimetrs, kurš izveidots no 61 200 svina volframāta ($\text{PbWO}_{4}$) kristāliem, kuri ir uzstādīti mucā, kā arī 7 324 kristāliem gala segumos. Muca aptver pseido\gls{straujumu} intervālā $\left|\eta\right|<1.479$, gala segumi aptver pseidostraujumu intervālā $1.479<\left|\eta\right|<3.0$. Pirms gala segumiem ir uzstādīts pirmslietus detektors. Mucā tiek izmantotas avalanša fotodiodes (APDs), bet gala segumos tiek izmantotas vakuuma fototriodes (VPTs). $\text{PbWO}_{4}$ kristāliem piemīt tādas īpašības, kas tos padara piemērotus LHP elektromagnētiskajam kalorimetram. To lielais blīvums 8.28 $\frac{\text{g}}{\text{cm}^3}$, mazais radiācijas garums (0.89 cm)  un mazais Moljēra radiuss (2.2cm) ļauj radīt smalkas graudainības un kompaktu kalorimetru. Scintilācijas rimšanas laiks $\text{PbWO}_{4}$ ir tādas pašas pakāpes lielums, kā LHP pūļu šķērsojuma laiks: ampēram 80\% gaismas ir izstarota 25 ns. Muca kristālu šķērsgriezuma laukas atbilst aptuveni 0.0174 $\times$ 0.0174 $\eta - \phi$ mērvienības, kam atbilst priekšējā šķērsgriezuma laukums 22 $\times$ 22$\text{mm}^2$ un aizmugurējā šķērsgriezuma laukums 26$\times$26$\text{mm}^2$. Kristālu garums ir 230 mm, kas atbilst 25.8 $X_{0}$. Mucā ir 61 200 kristāli. Gala segumos kristālu aizmugurējā šķērsgriezuma laukums 30 $\times$ 30$\text{mm}^2$,  un priekšējā šķērsgriezuma laukums 28.62 $\times$ 28.62$\text{mm}^2$, kā arī to garums ir 220 mm, kas atbilst 24.7$X_{0}$). Bez tam uzticamajā reģionā $1.653 < \left|\eta\right| <2.6$ atrodas pirmslietus detektors, kura galvenais mērķis ir identificēt neitrālos pionus gala segumos. Pirmslietus detektors sastāv no svina izstarotāja kurā inākošie elektroni/fotoni izraisa elektromagnētiskos lietus. Aiz svina izstarotāja atrodas silīcija strēmeles, lai izmērīto atstāto enerģiju un lietus šķērsprofilus. Enerģijas izšķirtspēja mucas elektromagnētiskajā kalorimetrā ir atkarīga no ienākošās enerģijas un tā ir novērtēta no 0.94 \% ($\sigma/E$) pie 20 \GeV līdz 0.34 \% pie 250\GeV \cite{Adzic:2007mi}. 

Hadroniskais kalorimetrs \cite{HCAL_report} sastāv no mucas ($\left|\eta\right| < 1.3 $) un diviem gala seguma diskiem ($1.3 < \left|\eta\right| < 3.0 $). Hadronu kalorimetra tilpums centrālajā pseidostrauja intervālā ir ierobežota. Tādēļ tiek izmantots ārējs atblāzmas novērotājs aiz solenoīda. Solenoīds tiek izmantos kā papildus absorbējošaias materiāls atblāzmas novērotājam. Absorbetājs sastāv no 40 mm biezas priekšējās tērauda plāksnes, kurai seko astoņas 50.5 mm biezas misiņa plāksnes, sešas 56.5 mm biezas misiņa plāksnes, un 75 mm bieza tērauda aizmugurējā plāksne. Kopējais absorbētāja biezums 00$^{\circ}$ leņķī ir 5.82 mijiedarbības garumi ($\lambda_{I}$). Kā aktīvais materiāls tiek izmantots plastikāta scintilātors, kas izkārtots \gls{dakstiņos}. Gaismas izvadīšanai tiek izmantotas viļna garuma pārbīdes šķiedras.Nolasījumi hadroniskajā kalorimetrā tiek veikti individuālos \gls{torņos} ar kopējo šķērsgriezuma laukumu $\Delta\eta\times\Delta\phi=0.087\times0.087$ intervālā $\left|\eta\right|<1.6$ un $0.17\times0.17$ pie liekākiem pseidostraujumiem. Hadronisko kalorimeteru papildina šaurleņka hadronu kalorimetri $\left|\eta\right|$ intervālā līdz $\simeq5.0$, kur daļiņu plūsama un radiācijas bojājumi ir vislielākie. Hadronu šaurleņķā kalorimetrs sastāv no tērauda absorbētāja, kuru veido rievotas plāksnes. Rievās gareniski kūļa virzienam ir ievietotas radiācijas noturīgas kvarca šķiedras the beam direction un tās nolasa fotoreizinātāji. Signāli tiek sagrupēti tā, lai varētu izveidot kalorimetra torņus ar šķērsgriezumu $\Delta\eta\times\Delta\phi=0.175\times0.175$ lielākajā daļā pseidostraujuma intervāla. 

Magnēts ir novietots aiz kalorimetriem un trekera, lai nodrošinātu, ka pēc iespējas mazāks materiāla daudzums atrodas starp minētajiem apakšdetektoriem un mijiedarbības punkta. Magnēta garums ir 12.5 m un tā brīvurbuma radiuss ir 3.15 m. Tinums rada 3.8-4.0 T vienmērīgu aksiālu magnētisko lauku trekerī un kalorimetros. Magnēts strādā pie 4.45 K un tajā tiek izmantoti NbTi supravadoši tinumi. Magnētu raksturo augsta uzkrātās eneŗgijas/massas attiecība - 11.6 $\frac{kJ}{kg}$.

Mionu kanāls ir ļoti efektīvs instruments interesējošu \gls{AEF} procsu pētīšanai un tas ir bijis ļoti nozīmīgs jau pirmajās KMS eksperimenta iecerēs. Tas izkaidrojams ar to, ka mionus ir ļoti vieglu novērot un tiem raksturīgi nelieli radiācijas zudumi trekera materiālā. Četras mionu detektora plāksnes atrodas ārpus solenoīda tinuma, kuras caurvij trīs tērauda jūga slāņi \cite{muon_tech_rep}. Mucā- $\left|\eta\right|<1.2$ kur mionu biežums ir zems un 4 T magnētiskais lauks ir vienmērīgs un pārsvarā ierobežots tēradau jūgā, tiek izmantoti dreifa kameras. Gala segumos $0.9<\left|\eta\right|<2.4$ kur mionu biežums un trokšņu līmenis ir augsts, kā arī magnētiskais lauks ir liels un nevienmērīgs mionu sistēma izmanto katoda strēmeļu kameras (CSC). Dēļ trokšņu līmeņā nenoteiktības un mionu sistēmas nenoteiktības tās spējā nomērīt pareizo kūļa šķērsošanas laiku LHP sasniedzot nominālo spīdumu, papildus gan mucā, gan gala segumos tiek izmantota speciāla trigueru sistēma, kas sastāv no pretestības plākšņu kamerām (RPC). RPC nodrošina ātru, neatkarīgu un ļoti segmentētu trigeri ar asu \pt slieksni lielā mionu sistēmas pseidostraujuma intervālā ($\left|\eta\right|<1.6$) of the muon system. Daļiņu plūsmas rekonstrukcija izmanto globālu trajektorijas saderināšanu mionu detektorā un iekšējā trekerī. 

Strūklas tiek rekonstruētas, izamntojot anti-$k_{t}$ algoritmu \cite{Cacciari:2008gp} ar radiusa parametru $R=0.4$ un FastJet pakotnes izpildījumu \cite{Cacciari:2011ma}. Attālums starp strūklām $d_{ij}$ tiek noteikts, izmantojot $p=-1$, atbilstoši šādai formulai:

\begin{equation}
d_{ij}=\text{min}(k_{ti}^{2p}, k_{tj}^{2p})\frac{\Delta_{ij}^{2}}{R^{2}},
\end{equation}

kur $\Delta_{ij}^{2} = (y_{i}-y_{j})^{2} + (\phi_{i} - \phi_{j})^{2}$ un $k_{ti}$, $y_{i}$, $\phi_{i}$ ir attiecīgi daļiņas $i$ šķērsmoments, straujums, un azimuts. 

Svarīģa šī algoritma priekšrocība ir tāda, ka \gls{mīkstās} daļiņās neizmaina strūklas formu. Ja attālums starp strūklām $\Delta_{ij}\leq2R$, tad tās ieņem konisku formu. 


\chapter{Metodoloģija}
\label{chap:methodology}
\section{Vilkmes leņķis}

Mēs izmantojam metodoloģiju, ko piedāvā \cite{Gallicchio:2010sw}, un kas balstās uz vilkmes leņķi krāsu saistības noteikšanai starp divām kvarku strūklām. Vilkmes leņķis $\theta_{p}$ starp vilkmes vektoru $\vec{v}_{p}$ un starpību starp divām strūklām $\vec{J}_{2}-\vec{J}_{1}$ ir attēlots \ref{fig:pull_angle} att. Tiek izmantota $\phi$ - $y$ koordinātu sistēma. 

\begin{figure}[hbtp]
  \centering
  \includegraphics[width=1.0\textwidth]{fig/pull_anglelv.pdf}
  \caption{Vilkmes leņķis $\theta_{p}$, vilkmes vektors $\vec{v}_{p}$ attēlots $y$ - $\phi$ plaknē.}
  \label{fig:pull_angle}
\end{figure}

Vilkmes vektoru aprēķina saskaņā ar formulu

\begin{equation}
  \vec{v}_{p}=\sum_{i\in J}\frac{p^{T}_{i}|\vec{r}_{i}|}{p^{T}_{J}}\vec{r}_{i},
  \label{Eq:pull_angle}
\end{equation}

kur $i$ ir strūklas $J$ sastāvdaļas indekss, $p^{T}_{i}$ ir strūklas sastāvdaļas šķērsmoments, $\vec{r}_{i}$ ir vektoriālā starpība starp strūklas sastāvdaļu un strūklu, $p^{T}_{J}$ - strūklas šķērsmoments.

Sagaidāms, ka gadījumā, ja strūklas ir saistītas ar krāsām, to sastāvdaļas būs izkliedētas reģionā starp strūklām. Līdz ar to $J_{1}$ vilkmes vektors rādīs uz $J_{2}$ un vilkmes leņķis būs šaurs. Gadījumā, ja strūklas nav saistītas ar krāsām, to sastāvdaļas tiks izkliedētas izotropiski.

Vilkmes leņķa metodoloģija ir tikusi pielietota Tevatrona \DZERO eksperimentā\cite{Abazov:2011vh} un LHP ATLAS eksperimentā I datu gūšanas periodā \cite{Aad:2015lxa} un II datu gūšanas periodā \cite{ATLAS:2017iaz}. Mēs ceram, ka mums izdosies iegūt vēl labākus rezultātus saskaņā ar vilkmes leņķa metodoloģiju, jo mūsu rīcībā ir 4 T KMS solenoīds.

Anti-$k_{T}$ \gls{sakopošanas} algoritms nodrošina, ka strūklas ieņems konisku formu, ja attālums starp strūklām \DeltaR ir divreiz lielāks nekā parametrs $R$, kurš KMS ir noteikts kā 0,4. Šīs gadījums ir attēlots \ref{fig:anti_kt_a} att. Gadījumā, ja starpība starp strūklām \DeltaR ir mazāka nekā divkāršs parametrs $R$, \gls{cietā} strūkla pievienos sev mīkstās strūklas sastāvdaļas. Šis gadījums ir attēlots \ref{fig:anti_kt_b} att. Pēdējais gadījums atstās iespaidu uz krāsu plūsmas analīzi ar vilkmes leņķa metodi, jo šādi tiks ierosināta vienas strūklas vilkme uz otru strūklu. Tādēļ ir pamatota gadījumu $\DeltaR\leq2R$, $\DeltaR>2R$ nošķiršana. 

\begin{figure}[hbtp]
  \def\twidth{0.5}
  \subfloat[$\Delta_{ij}=3,15$.]{
    \includegraphics[width=\twidth\textwidth]{fig/dR-3p150-pt2-075.pdf}
    \label{fig:anti_kt_a}
  }%
 \subfloat[$\Delta_{ij}=1,95$.]{
    \includegraphics[width=\twidth\textwidth]{fig/dR-1p950-pt2-075.pdf}
    \label{fig:anti_kt_b}
  }
   \caption{Strūklu formas, kas iegūtas ar anti-$k_{t}$ sakopošanas algoritmu. Šajā piemērā iek izmantots $R=1,5$. Tiek attēloti divi gadījumi - $\Delta_{ij}=3,15$ un  $\Delta_{ij}=1,95$. Cietās strūklas \pt ir 100 GeV, mīkstās strūklas \pt ir 75 GeV. Par iespēju izmantot attēlus pateicos Cacciari, Salam and Soyez.}
  \label{fig:anti_kt}
\end{figure}

Detektora trekēšanas efektivitāte nav ideāla. Tā ir atkarīga no treku meklēšanas algoritma un detektora īpašībām, kā, piemēram, ģeometrisko \gls{uzņēmību} un materiālu saturu. \ref{fig:2011_trackPerformance_MC_SingleParticles_pi_efficiencyVsPt} att. redzama pionu trekēšanas efektivitāte. Pions lielā daudzumā rodas kvarku hadronizācijas procesā. Trekēšanas efektivitāte ir definēta kā to simulēto lādēto daļiņu skaita, kuras saistītas ar rekonstruētiem trekiem, īpatsvars. Trekēšanās efektivitāte pasliktinās pie zema daļiņas \pt. Mūsu analīzē esam izvēlējušies 1 GeV kā \pt slieksni, zem kura daļiņas tiek izslēgtas no analīzes.

\begin{figure}[hbtp]
    \includegraphics[width=0.6\textwidth]{fig/figs_2011_trackPerformance_MC_SingleParticles_pi_efficiencyVsPt.png}
    \caption{Pionu treku rekonstruēšanas efektivitāte, kuri izturējuši augstas tīrības prasības. Rezultāti ir atainoti kā funkcija no \pt mucas, pārejas un gala segumu reģionos, kas attiecīgi atbilst $\left|\eta\right|$ intervāliem 0 - 0,9, 0,9 - 1,4 un 1,4 - 2,5 \cite{Chatrchyan:2014fea}.}
    \label{fig:2011_trackPerformance_MC_SingleParticles_pi_efficiencyVsPt}
\end{figure}

\section{LEP metode}

LEP dažādos eksperimentos tika izmantota arī cita metode ar krāsām saistītu strūklu pētīšanai $e^{+}e^{-}\rightarrow q\overline{q}q\overline{q}$ procesā ar \sqrts=189-207 \GeV (\cite{Abdallah:2006uq}, \cite{Abbiendi:2005es}, \cite{Achard:2003pe}). Tiek apskatītas divas starp-\PW plaknes, kuras veido ar krāsām saistīti kvarki un divas ārpus-\PW plaknes, kuras veido ar krāsām nesaistīti kvarki kā attēlots \ref{fig:LEP_method} att. Daļiņas tiek projicētas uz šīm plaknēm un tiek noteikts leņķis $\chi_{1}$ ar kvarku, kas atrodas kreisajā pusē. Ja šis leņķis ir mazāks nekā leņķis $\chi_{0}$ starp kvarkiem, kas veido plakni (tas nozīmē, ka daļiņā tiek projicēta starp attiecīgajiem kvarkiem), tad normalizētais leņķis $\chi_{R}=\frac{\chi_{1}}{\chi_{0}}$ tiek iekļauts grafika reģionā, kas atbilst šai plaknei, pēc normalizētā leņķa lineāras transformācijas

\begin{equation}
  \chi=\chi_{R} + n_{\text{plane}} - 1.
\end{equation}

\begin{figure}[hbtp]
  \centering
  \includegraphics[width=0.6\textwidth]{fig/L3methodlv.pdf}
  \caption{Starp-\PW un ārpus-\PW plaknes $e^{+}e{-}\rightarrow q\overline{q}q\overline{q}$ procesā un relatīvais leņķis $\chi_{R}=\frac{\chi_{1}}{\chi_{0}}$.}
  \label{fig:LEP_method}
\end{figure}

 \ttbar semileptoniskajā sabrukumā tāds grupējums, kā attēlots \ref{fig:LEP_method} att., nav iespējams. Tādēļ tiek izmantota pielāgošana, kas piedāvāta \ref{fig:LEP_method_adaptation} att. Tiek izmantota viena plakne, kuru veido ar krāsām saistītas strūklas - vadošā vieglā strūkla \leadingjet un otra vadošā vieglā strūkla \scndleadingjet, kas radušās hadroniskajā \PW bozona sabrukuma procesā. Bez tam tiek izmantoti 3 no krāsām brīvi reģioni, kurus veidi 1) tālākā vieglā strūkla $j^{W}_{f}$ un hadroniskā \cPqb strūkla \hadronicb 2) hadroniskā \cPqb strūkla \hadronicb un tuvākā vieglā strūkla $j^{W}_{c}$, 3) vadošā \cPqb strūkla \leadingb un otra vadošā \cPqb strūkla \scndleadingb. Attālums starp strūklām tiek noteikts, vadoties pēc leņķā starp tām Eiklīda telpā. Reģionos, kas atainoti \ref{fig:LEP_method_adaptation_qfhb} att. un \ref{fig:LEP_method_adaptation_hbqc}  att., varam cerēt, ka spēsim novērot krāsu atkārtota savienojuma efektus.

\begin{figure}[hbtp]
  \centering
  \def\twidth{0.24}
  \subfloat[Ar krāsām saistītais reģions \leadingjet - \scndleadingjet.]{%
    \includegraphics[width=\twidth\textwidth]{fig/LEP_adaptation/qlq2llv.pdf}%
    \label{fig:LEP_method_adaptation_qlq2l}
  }\hfil
 \subfloat[No krāsam brīvais reģions $j^{W}_{f}$ - \hadronicb.]{%
    \includegraphics[width=\twidth\textwidth]{fig/LEP_adaptation/qfhblv.pdf}%
    \label{fig:LEP_method_adaptation_qfhb}
 }\hfil
  \subfloat[No krāsām brīvais reģions \hadronicb - $j^{W}_{c}$.]{%
    \includegraphics[width=\twidth\textwidth]{fig/LEP_adaptation/hbqclv.pdf}%
    \label{fig:LEP_method_adaptation_hbqc}
  }\hfil
  \subfloat[No krāsam brīvais reģions \leadingb - \scndleadingb.]{%
    \includegraphics[width=\twidth\textwidth]{fig/LEP_adaptation/blb2llv.pdf}%
    \label{fig:LEP_method_adaptation_blb2l}
  }
  \caption{LEP metodes adaptācija \ttbar semileptoniskajam sabrukumam ar ar krāsām saistītu reģionu un 3 no krāsām brīviem reģioniem.}
  \label{fig:LEP_method_adaptation}
\end{figure}

Lai metodi varētu lietot, nepieciešams nošķirt hadronisko un leptonisko \cPqb kvarku. Katrs no \cPqb kvarkiem tiek sapārots ar \PW bozonu un kopējā invariantā masa tiek salīdzināta ar \cPqt kvarka masu - 173,34 \GeV. \cPqb kvarks tiek piesaistīts tam zaram, kur masu starpība ir vismazākā. 



\chapter{Datu un MK paraugi}

The data analysed for present study consists of the 2016{B-H} data taking periods for a total certified luminosity of 35.9\fbinv for all the channels analysed. The luminosity has been computed with the \textsc{brilcalc} tool \cite{site:brilcalc} using the following command:

\begin{lstlisting}[language=sh, breaklines=true]
brilcalc lumi  -b "STABLE BEAMS" --normtag /afs/cern.ch/user/l/lumipro/public/Normtags/normtag_DATACERT.json -i lumiSummary.json
\end{lstlisting}

All data used for this study are listed in Table \ref{tab:datasets}.  

\begin{table}[htb]
\begin{center}
\caption{Primary datasets used in this analysis. PD is an abbreviation for SingleMuon or SingleElectron.}
\label{tab:datasets}
\begin{tabular}{ lc }
\hline
Primary dataset                    & Integrated luminosity\\
\hline
/PD/Run2016B-23Sep2016-v3/MINIAOD  & \multirow{8}{*}{35.9 \fbinv}\\
/PD/Run2016C-23Sep2016-v1/MINIAOD  & \\
/PD/Run2016D-23Sep2016-v1/MINIAOD  & \\
/PD/Run2016E-23Sep2016-v1/MINIAOD  & \\
/PD/Run2016F-23Sep2016-v1/MINIAOD  & \\
/PD/Run2016G-23Sep2016-v1/MINIAOD  & \\
/PD/Run2016H-PromptReco-v2/MINIAOD & \\
/PD/Run2016H-PromptReco-v3/MINIAOD & \\\cline{1-2}
\hline
\end{tabular}
\end{center}
\end{table}

The list of simulated samples can be found in Table \ref{tab:mcdatasets}. The cross sections are theoretical predictions. Practically, they are obtained from from~\cite{twiki:SingleTopRefXsec} and ~\cite{twiki:SM13} except for \ttbar for which the generator cross section is quoted according to~\cite{site:MCM}. At NNLO the expected \ttbar cross section is $832^{ +20}_{-29}~(\text{scale})~\pm 35~(\text{PDF}+\alpha_S)$~\cite{twiki:TTbarNLO}. We use the NNLO reference to normalise all \ttbar samples.

\begin{table}
\caption{Simulation samples are from RunIISummer16MiniAODv2-PUMoriond17\_80X\_mcRun2\_asymptotic\_2016\_TrancheIV\_v6 production. We quote the cross section used to normalise the sample in the analysis.}
\label{tab:mcdatasets}
\begin{longtable}{ p{0.16\textwidth}ll }
\hline
Process                      & Dataset                                                                     & $\sigma[pb]$\\
\hline
\multicolumn{3}{l}{\bf Signal} \\
\hline
\ttbar                       & \small  TT\_TuneCUETP8M2T4\_13TeV-powheg-pythia8                            & 832\\
\hline
\multicolumn{3}{l}{\bf Background} \\
\hline
\multirow{2}{*}{\ttbar+\PW}  & \small TTWJetsToLNu\_TuneCUETP8M1\_13TeV-amcatnloFXFX-madspin-pythia8       & 0.20 \\
                             & \small TTWJetsToQQ\_TuneCUETP8M1\_13TeV-amcatnloFXFX-madspin-pythia8        & 0.41 \\\hline
\multirow{2}{*}{\ttbar+\cPZ} & \small TTZToQQ\_TuneCUETP8M1\_13TeV-amcatnlo-pythia                        & 0.53 \\
                             & \small TTZToLLNuNu\_M-10\_TuneCUETP8M1\_13TeV-amcatnlo-pythia8              & 0.25 \\\hline
\PW\cPZ                      & \small WZTo3LNu\_TuneCUETP8M1\_13TeV-amcatnloFXFX-pythia8                   & 5.26 \\\hline
\multirow{2}{*}{\PW\PW}      & \small WWToLNuQQ\_13TeV-powheg                                              & 50.0 \\
                             & \small WWTo2L2Nu\_13TeV-powheg                                              & 12.2 \\\hline
\multirow{2}{*}{\cPZ\cPZ}    & \small ZZTo2L2Nu\_13TeV\_powheg\_pythia8                                    & 0.564 \\
                             & \small ZZTo2L2Q\_13TeV\_amcatnloFXFX\_madspin\_pythia8                      & 3.22 \\\hline
\multirow{3}{*}{\PW+jets}    & \small WToLNu\_0J\_13TeV-amcatnloFXFX-pythia8                               & 49540 \\
                             & \small WToLNu\_1J\_13TeV-amcatnloFXFX-pythia8                               & 8041 \\
                             & \small WToLNu\_2J\_13TeV-amcatnloFXFX-pythia8                               & 3052 \\\hline
\multirow{2}{*}{Drell-Yan}   & \small DYJetsToLL\_M-10to50\_TuneCUETP8M1\_13TeV-madgraphMLM-pythia8        & 18610 \\
                             & \small DYJetsToLL\_M-50\_TuneCUETP8M1\_13TeV-madgraphMLM-pythia8            & 6025 \\\hline
\multirow{10}{=}{QCD $\mu$ enriched}
                             & \small QCD\_Pt-30to50\_MuEnrichedPt5\_TuneCUETP8M1\_13TeV\_pythia8          & 1652471.46\\ 
                             & \small QCD\_Pt-50to80\_MuEnrichedPt5\_TuneCUETP8M1\_13TeV\_pythia8          & 437504.1\\
                             & \small QCD\_Pt-80to120\_MuEnrichedPt5\_TuneCUETP8M1\_13TeV\_pythia8         & 106033.66\\
                             & \small QCD\_Pt-120to170\_MuEnrichedPt5\_TuneCUETP8M1\_13TeV\_pythia8        & 25190.52\\
                             & \small QCD\_Pt-170to300\_MuEnrichedPt5\_TuneCUETP8M1\_13TeV\_pythia8        & 8654.49\\
                             & \small QCD\_Pt-300to470\_MuEnrichedPt5\_TuneCUETP8M1\_13TeV\_pythia8        & 797.35\\
                             & \small QCD\_Pt-470to600\_MuEnrichedPt5\_TuneCUETP8M1\_13TeV\_pythia8        & 45.83\\
                             & \small QCD\_Pt-600to800\_MuEnrichedPt5\_TuneCUETP8M1\_13TeV\_pythia8        & 25.1\\
                             & \small QCD\_Pt-800to1000\_MuEnrichedPt5\_TuneCUETP8M1\_13TeV\_pythia8       & 4.71\\
                             & \small QCD\_Pt-1000toInf\_MuEnrichedPt5\_TuneCUETP8M1\_13TeV\_pythia8       & 1.62\\\hline
\multirow{6}{=}{QCD $e$ enriched}
                             & \small QCD\_Pt-30to50\_EMEnriched\_TuneCUETP8M1\_13TeV\_pythia8             & 6493800.0\\
                             & \small QCD\_Pt-50to80\_EMEnriched\_TuneCUETP8M1\_13TeV\_pythia8             & 2025400.0\\
                             & \small QCD\_Pt-80to120\_EMEnriched\_TuneCUETP8M1\_13TeV\_pythia8            & 478520.0\\
                             & \small QCD\_Pt-120to170\_EMEnriched\_TuneCUETP8M1\_13TeV\_pythia8           & 68592.0\\
                             & \small QCD\_Pt-170to300\_EMEnriched\_TuneCUETP8M1\_13TeV\_pythia8           & 18810.0\\
                             & \small QCD\_Pt-300toInf\_EMEnriched\_TuneCUETP8M1\_13TeV\_pythia8           & 1350.0\\

\hline
\end{longtable}
\end{table}

The colour octet sample is listed in \ref{tab:mcdatasets_flip}. 

\begin{table}[htb]
\begin{center}
\caption{Simulation samples for the colour octet \PW boson. We quote the cross section used to normalise the sample in the analysis.}
\label{tab:mcdatasets_flip}
\hspace*{-0.5cm}
\begin{tabular}{ llc }
\hline
Process & Dataset & $\sigma[pb]$\\
\multicolumn{3}{l}{\bf Background} \\
\hline
Colour octet $W$ boson &  {\small TT\_TuneCUETP8M2T4\_13TeV-powheg-colourFlip-pythia8} & 832 \\
\hline
\end{tabular}
\end{center}
\end{table}

\section{Corrections applied to the simulation}
\label{sec:mccorrections}

Based on differences between data and simulated events different sets of corrections are applied to the latter.

\begin{description}

\item[Pileup re-weighting]

\item[Lepton identification and isolation efficiency]
\item[Trigger efficiency]

\item[Generator level weights]

\item[Jet energy scale and resolution]
\item[\cPqb tagging efficiency]


\end{description}



\chapter{Notikumu atlase}
\label{chap:event_selection}
Notikumu atlases mērķis ir nodalīt signālu no fona. Notikumu atlase ir atšķirīga detektora līmeņa MK notikumiem un ģeneratora līmeņa MK notikumiem. Dati tiek atlasīti atbilstoši detektora līmeņa notikumu atlasei.

Šīs nodaļas izklāsts ir pielāgots pēc~\cite{CMS-AN-2017-159} parauga, jo šajā pētījumā tiek izmantota līdzīga notikumu atlase.

\section{Detektora līmenis}
\label{sec:detector_level}

Notikumu atlase ir balstīta \ttbar$\to$ leptons + strūklas topoloģijā, kur viens no \PW bozoniem sabrūk par lādētu leptonu ($\ell=e, \mu$) un atbilstošo neitrino, bet otrs \PW bozons sabrūk par kvarkiem, kas rada strūklas.

Lai rekonstruētu gala stadijas objektus, tiek izmantots daļiņu plūsmas algoritms~\cite{Sirunyan:2017ulk}. Lai celtu rekonstrukcijas kvalitāti, šis algoritms kombinē signālus no visiem apakšdetektoriem, un ar tā palīdzību var identificēt mionus, elektronus, fotonus, lādētos hadronus un neitrālos hadronus pēc \Pp\Pp sadursmes.

Datu paraugi tiek ievākti, izmantojot \gls{Augsta līmeņa trigera} viena leptona trigera ceļus, kas apkopoti \ref{tab:triggers}~tab.

\begin{table}[htp]
\centering
\caption{Analīzē izmantotās tiešsaistes atlases trigera ceļi.}
\label{tab:triggers}
\begin{tabularx}{\linewidth}{lllXX}\hline
Gala stadija                & Ceļš                                       & Darbības intervāls & Funkcija & L1 sākums\\\hline
$e$ + strūklas                      & \small HLT\_Ele32\_eta2p1\_WPTight\_Gsf\_v & visi       & \small Atlasīt $e$ ar $\left|\eta\right|<2,1$ un $\pt>32~\GeV$ ciešajā darbības punktā, izmantojot GSF, lai rekonstruētu trekus
                                                                                         & \small L1\_SingleEG40\newline VAI\newline L1\_SingleIsoEG22er\newline VAI\newline L1\_SingleIsoEG24er\newline VAI\newline L1\_SingleIsoEG24\newline VAI\newline L1\_SingleIsoEG26\\\hline
\multirow[t]{2}{*}{$\mu$ + strūklas}
                            & \small HLT\_IsoMu24\_v                     & visi       & \small Atlasīt izolētu $\mu$ ar $\pt>20$~\GeV, izmantojot L3 trekera algoritmu
                                                                                         & \multirow[t]{2}{*}{\small L1\_SingleMu18}\\
                            & \small HLT\_IsoTkMu24\_v                   & visi       & \small Atlasīt izolētu $\mu$ ar $\pt>20$~\GeV, izmantojot HLT trekera mionu algoritmu
                            & \\\hline
\end{tabularx}
\end{table}

Bezsaistē tiek prasīts viens ciešs elektrons/mions ar $\pt>34/26~\GeV$ un $|\eta|<2,1/2,4$. Notikumam, kurā ir otrs vaļīgs leptons ar $\pt>15~\GeV$ un $|\eta|<2,4$, tiek uzlikts veto.

Notikumā jābūt četrām strūklām, kas sakopotas ar anti-$k_{\text{T}}$ algoritmu, strūklu atdalījumu ar $R=0,4$ un lādēto hadronu atņemšanu (AK4PFchs) un kurām $\pt>30\GeV$ un $|\eta|<2,4$. 

Vismaz divām strūklām jābūt \cPqb atzīmētām ar {\it Combined Secondary Vertex} algoritma (CSVv2) vidējo darbības punktu. 

Notikumā jābūt vismas divām neatzīmētām (vieglajām) strūklām, kurām jārada \PW bozona kandidāts, kura invariantā masa $\left|m_{jj}-80,4\right|<15~\GeV$.

Notikumu raža dažādās atlases stadijās ir redzama \ref{fig:_reco_selection}~att. un sniegta \ref{tab:yields}~tab. \ref{tab:yields_cflip}~tab. attēlota notikumu raža krāsu okteta \PW paraugam. Signāla īpatsvars palielinās no 0,1~\% sākotnējā stadijā līdz 94,2~\% gala stadijā \textendash\ tas ir šīs atlases efektivitātes mērs.

\figureEML{/reco/}{_reco_selection}{Notikumu raža dažādās notikumu atlases stadijās: $1 \ell$, $1 \ell + \geq 4 j$, $1 \ell + \geq 4 j (2 b)$, $1 \ell + \geq 4 j (2 b, 2 lj)$.}

\input{tables/event_yields_tableslv/event_yields_table.txt}

\input{tables/event_yields_tableslv/event_yields_table_cflip.txt}

\section{Ģeneratora līmenis}
\label{sec:generator_level}

Simulācijā bezsaistes atlase daļiņu līmenī tiek imitēta, izmantojot \PSEUDOTOPPRODUCER rīku~\cite{code:pseudotop}, izmantojot kopīgu leptonu atlasi gan elektroniem, gan mioniem ar $\pt>26\GeV$ un $|\eta|<2,4$, kā arī tādām pašām prasībām strūklām $\pt/\eta$ ($\pt>30~\GeV$, $|\eta|<2,4$) un \PW masai ($\left|m_{jj}-80,4\right|<15~\GeV$) kā bezsaistes atlasei detektora līmenī.

Lādētie leptoni, kas izdalās cietajā procesā, tiek apzaroti ar leptoniem $R=0,1$ konusā. Strūklas tiek sakopotas ar anti-$k_\text{T}$ algoritmu ar $R=0,4$ konusu pēc apzaroto leptonu kā arī visu neitrino noņemšanas. Lai noteiktu strūklas smaržu daļiņu līmenī, strūklas sakopojumā tiek iekļauti ``spoku'' $B$ hadroni, kad to moments ir reizināts ar $10^{-20}$ tā, lai tie būtiski nemainītu strūklas enerģijas mērogu daļiņu līmenī.


\chapter{Sistemātiskās nenoteiktības}
\label{chap:systematic_uncertainties}
Nenoteiktības tiek iedalītas eksperimentālajās un teorētiskajās nenoteiktībās. Iekļaujot kādu nenoteiktību no pirmās grupas, tiek variēts kāds parametrs notikumu atlasē, piemēram, datu/MK koeficients. Teorētiskās nenoteiktības atspoguļo mūsu zināšanu trūkumu par reālo pasauli, piemēram, hadronizācijas procesa norises niansēm. 

Šīs nodaļas izklāsts ir pielāgots pēc ~\cite{CMS-AN-2017-175} un \cite{CMS-AN-2017-159} parauga, jo šajos pētījumos tiek izmantoti līdzīgi nenoteiktību kopumi.

\section{Eksperimentālās nenoteiktības}
\begin{description}
\item[Sagrūdums] Kaut gan sagrūdums ir iekļauts simulācijā, pastāv nenoteiktība tā modelēšanā. Lai novērtētu modelēšanas kļūdu, tiek mainīts minimālās noslieces šķērsgriezuma parametrs par 5~\% attiecībā pret tā sākotnējo novērtējumu. 

\item[Trigera un atlases efektivitāte] Trigera efektivitātes nenoteiktība un leptona identifikācijas un izolācijas koeficienta nenoteiktība tiek iestrādāta, mainot nominālos parametrus augšup un lejup. Mionu trekera efektivitātes nenoteiktība arī ir iekļauta šajā kategorijā un ir pievienota otrajā pakāpē, kaut gan tās sagaidāmā ietekme ir nenozīmīga. 

\item[Strūklu enerģijas izšķirtspēja] Izmantojam ieteiktos strūklas enerģijas mērījumus~\cite{twiki:JER}. Katra strūkla tiek tālāk izsmērēta augšup un lejup atkarībā no tās \pt un $\eta$ attiecībā pret centrālo vērtību, kas izmērīta datos. Galvenā šīs nenoteiktības ietekme ir notikumu izslēgšana/iekļaušana, ja to strūklu parametri ir tuvu atlases slieksnim. 

\item[Strūklu enerģijas mērogs] Lai kalibrētu strūklas simulācijā, izmantojam strūklas enerģijas mēroga parametrizāciju, kas atkarīga no \pt-$\eta$. Šī parametrizācija balstās uz Spring16 V3 korekcijām, kuras sizstrādājusi JetMET Fizikas objektu grupa~\cite{twiki:JES}. Variējot strūklu enerģijas skalu, tiek pārrēķināts arī iztrūkstošās šķērsenerģijas $E^{\text{miss}}_{T}$ novērtējums. Galvenā šīs nenoteiktības ietekme ir notikumu izslēgšana/iekļaušana, ja to strūklu parametri ir tuvu bezsaistes atlases slieksnim. 

\item[\cPqb-atzīmēšana] \cPqb-atzīmēšanas nominālā efektivitāte tiek koriģēta ar no \pt-atkarīgiem koeficientiem, kurus piedāvā BTV Fizikas objektu grupa~\cite{twiki:BTV}. Atkarībā no strūklas garšas, \cPqb-atzīmēšanas lēmums tiek atjaunināts saskaņā ar koeficientu. Arī koeficients tiek variēts saskaņā ar tā nenoteiktību. Galvenā šīs nenoteiktības ietekme ir uz lēmumiem par strūklu \cPqb-atzīmēšanas apstiprināšanu/noraidīšanu.
  
\item[Trekēšanas efektivitāte] TRK un MUO Fizikas objektu grupas ir izstrādājušas trekēšanas efektivitātes koeficientus kā funkciju no treka $\eta$ va rekonstruēto virsotņu daudzuma. Visi šie koeficienti ir atkarīgi no datu gūšanas periodiem.
\end{description}

\section{Teorētiskās nenoteiktības}
\begin{description}
\item[KHD koeficientu izvēle] Apskatām antikorelētas faktorizācijas un renormalizācijas koeficientu variācijas ar kārtu 0,5 un 2 ($\mu_R/\mu_F$) \ttbar paraugos. Šīs variācijas tiek noglabātas simulētajos notikumos kā alternatīva svaru kopa, kas tiek izmantota šīs nenoteiktības novērtēšanā. Tiek izmantota 7 variāciju kopums (izņemot pretēju variāciju $\mu_R/\mu_F$).

\item[\EVTGEN] Sistemātiskā nenoteiktība rodas, ja smago garšu daļiņu, pamatā $B$ un $D$ mezonu sabrukums tiek simulēts ar Monte Karlo notikumu ģeneratoru \EVTGEN.

\item[Hadronizatora izvēle] Sistemātiskā nenoteiktība rodas, ja hadronizācijas procesu modelē ar \HERWIGpp~\cite{Bahr:2008pv}. 

\item[Virsotnes kvarka masa] Visprecīzākie virsotnes kvarka masas mērījumi norāda uz nenoteiktību $\pm0,49~\text{GeV}$~\cite{Khachatryan:2015hba}. Tomēr mēs piesardzīgi pieņemam nenoteiktību vienādu ar $\pm1~\text{GeV}$. 

\item[\PYTHIA uzskaņojumi] Tiek izmantoti šādi \PYTHIA uzskaņojumi:
  \begin{enumerate}
  \item Matricas elementu + partonu lietus savietošanas shēma,
  \item Partonu lietus mērogs,
  \item Krāsu otrreizējās savienošanas modelis,
  \item Pamata notikuma variācijas.
  \end{enumerate}
\end{description}

\ref{tab:mcsystdatasets}~tab. ir apkopots teorētisko nenoteiktību simulāciju paraugu un to šķērsgriezumu uzskaitījums. Šie simulāciju paraugi ir iegūti no

RunIISummer16MiniAODv2-PUMoriond17\_80X\_mcRun2\_asymptotic\_2016\_TrancheIV\_v6

izstrādes.

\begin{table}[!htp]
\begin{center}
\caption{Teorētisko nenoteiktību simulāciju paraugi.}
\label{tab:mcsystdatasets}
\begin{tabular}{llr}
\hline
Signāla variācija & Datu kopa & $\sigma[pb]$\\
\hline
\multirow{4}{*}{Partonu lietus mērogs}
& {\small TT\_TuneCUETP8M2T4\_13TeV-powheg-isrup-pythia8}     & 832\\
& {\small TT\_TuneCUETP8M2T4\_13TeV-powheg-isrdown-pythia8}   & 832\\
& {\small TT\_TuneCUETP8M2T4\_13TeV-powheg-fsrup-pythia8}     & 832\\
& {\small TT\_TuneCUETP8M2T4\_13TeV-powheg-fsrup-pythia8}     & 832\\\hline
\multirow{2}{*}{Pamata notikums}
& {\small TT\_TuneCUETP8M2T4up\_13TeV-powheg-pythia8 }        & 832\\
& {\small TT\_TuneCUETP8M2T4down\_13TeV-powheg-pythia8}       & 832\\\hline
\multirow{2}{*}{ME-PS savietošanas skala (hdamp)}
& {\small TT\_hdampUP\_TuneCUETP8M2T4\_13TeV-powheg-pythia8}  & 832\\
& {\small TT\_hdampDOWN\_TuneCUETP8M2T4\_13TeV-powheg-pythia8}& 832 \\\hline
\multirow{3}{*}{Krāsu otrreizējā savienošanās}
& {\small TT\_TuneCUETP8M2T4\_erdON\_13TeV-powheg-pythia8 }   & 832\\
& {\small TT\_TuneCUETP8M2T4\_QCDbasedCRTune\_erdON\_13TeV-powheg-pythia8} & 832\\
& {\small TT\_TuneCUETP8M2T4\_GluonMoveCRTune\_13TeV-powheg-pythia8} & 832\\\hline
\multirow{2}{*}{Virsotnes masa}
& {\small TT\_TuneCUETP8M2T4\_mtop1715\_13TeV-powheg-pythia8 }& 832\\
& {\small TT\_TuneCUETP8M2T4\_mtop1735\_13TeV-powheg-pythia8} & 832\\\hline
\HERWIGpp & {\small TT\_TuneEE5C\_13TeV-powheg-herwigpp}      & 832\\
\hline
\end{tabular}
\end{center}
\end{table}




\chapter{Rezultāti}
\label{chap:results}
\label{chap:results}
\section{Notikumu attēls}
\ref{fig:event_display}. att. redzams notikums, kuru veido vieglās strūklas, \cPqb kvarku strūklas un leptons $\eta-\phi$ plaknē. Attēlots arī vilkmes vektors. Attēls veidots līdzīgi kā \ref{fig:pull_angle}. att.

\begin{figure}[hbtp]
  \centering
  \includegraphics[width=1.0\textwidth]{fig/individual_plots/reco_allconst_total_1111_DeltaR_2p846131_pull_angle_1p964620.png}
  \caption{Vadošās strūklas vilmes vektors (pārtraukta līnija ar punktiem), kas veido 1.96 rad vilkmes leņķi ar vektoriālo starpību (pārtraukta līnija) starp otro vadošo vieglo strūklu un vadošo vieglo strūklu. Vadošās strūklas sastāvdaļas ir attēlotas ziā, krāsā, bet otrās vadošās strūklas sastāvdaļas ir attēlotas sarkanā krāsā. Vadošā strūkla ir attēlota ar nepārtrauktau līniju, bet otrā vadošā strūkla ir attēlota ar pārtrauktu līniju. Hadroniskā \cPqb kvarka strūkla un tās sastāvdaļas attēlotas zaļā krāsā, bet leptoniskā \cPqb kvarka strūkla un tās sastāvdaļas attēlotas violetā krāsā. Vilkmes vektors ir palielināts 200 reizes, bet apļiem, kas attēlo strūklas, radiuss ir vienās ar $\frac{p_{T}}{75.0}$, savukārta apļiem, kas attēlo strūklu sastāvdaļas, radiuss ir vienās ar $\frac{p^{\text{constituent}}_{T}}{p^{\text{jet}}_{T}}$.}
  \label{fig:event_display}
\end{figure}

\section{Vilkmes vektors}

Pētījuma veikšanai tika izstrādāts rīku kopums \textsc{CFAT} \cite{url:cfat}. Galvenā pētījumu daļā tika veikta, izmantojot \CMSSW versiju \lstinline[language=sh]|CMSSW_8_0_26_patch1|. Grafiki ir veidoti, izmantojot \ROOT \cite{Brun}. Vilkmes vektori tiek noteikti visāmm novērojamajām strūklā - vadošajai vieglajai strūklai \leadingjet (ar vislielāko \pt), otrajai vadošajai vieglajai strūklai \scndleadingjet, vadošajai hadroniskā \cPqb strūklai \leadingb un otrajai vadošajai hadroniskā \cPqb strūklai \scndleadingb. Visos gadījumos, tiek izdalīti apakšgadījumus, iekļaujot vilkmes vektora aprēķinā visas strūklaas sastāvdaļās vai tikai elektriski lādētās sastāvdaļas. Rezultāti ir izdalīti pa $e$ + strūklas0, $\mu$ + strūklas un kopējo leptons + strūklas kanāliem.

$\eta$ dimensija vilkmes vektoram ar visām strūklas sastāvdaļām ir attēlots \ref{fig:_eta_PV_allconst_reco_leading_jet}. - \ref{fig:_eta_PV_allconst_reco_leading_jet}. att.

Vietā ir paskaidrojums par KMS grafiku formātu. Augšējais grafiks \ref{fig:_eta_PV_allconst_reco_leading_jet}. att. attēlo datus un Monte Karlo simulācijas. Ja nav norādīts citādi, Monte Karlo simulācijas ir attēlotas rekonstrukcijas līmenī. Zilā josla attēlo sistemātiskās nenoteiktības. Ja dota sistemātskā nenoteiktība ar indeksu $k$ mēs to identificējam kā pozitīvu sistemātiku $U^{k}_{i}$, ja vērtības intervālā $i$ sistemātika $S^{k}_i$ pārsniedz nominālo vērtību $N_{i}$. Pretējā gadījuma sistemātiskā nenoteiktība tiek noteikta kā negatīva sistemātikā nenoteiktība $D^{k}_{i}$. Kopējā pozitīvā un negatīvā sistemātiskā nenoteiktība tiek noteikta kā kvadrātu summa:

\begin{align}
U_{i}=\sqrt{\sum_{k}\left(U^{k}_{i}-N_{i}\right)^{2}} && D_{i}=\sqrt{\left(\sum_{k}D^{k}_{i}-N_{i}\right)^{2}}.
\end{align}

Zilās joslas platums atbilst sistemātiskajai kļūdai, kas noteikta kā $\frac{U_{i}+D_{i}}{2}$. Joslas centrs atbilst $N_{i} + \frac{U_{i}-D_{i}}{2}$. Teiktais attiecas arī uz rozā joslu ar atšķirību, ka sistemātikas ir tikušas normalizētas ar signāla integrāli (šādas normalizētas histogramas tiek sauktas par \gls{veidoliem}). Apakšejā ielikumā attēlota datu attiecība pret Monte Karlo, kā arī sistemātikas, kas normalizētas pret Monte Karlo.

\figureEML{/reco/PV/charge/allconst/}
          {_eta_PV_allconst_reco_leading_jet}
          {\leadingjet vilkmes vektora $\eta$ dimensija, iekļaujot visas strūklas sastāvdaļas.}

\leadingjet vilkmes vektora  $\phi$ dimensija, ieļaujot visas strūklas sastāvdaļas ir attēlota \ref{fig:_phi_PV_allconst_reco_leading_jet}. - \ref{fig:_phi_PV_allconst_reco_leading_jet}. att. 

\figureEML{/reco/PV/charge/allconst/}
          {_phi_PV_allconst_reco_leading_jet}
          {\leadingjet vilkmes vektora $\phi$ dimensija, iekļaujot visas strūklas sastāvdaļas.}

Vilkmes vektora lielums ar visām strūklu komponentēm ir attēlots \ref{fig:_mag_PV_allconst_reco_leading_jet} - \ref{fig:_mag_PV_allconst_reco_leading_jet}. att. Vilkmes vektora lielums parasti nepārsniedz 0.02 [bez mērvienības].

\figureEML{/reco/PV/charge/allconst/}
          {_mag_PV_allconst_reco_leading_jet}
          {\leadingjet vilkmes vektora lielums, iekļaujot visas strūklas sastāvdaļas.}

\section{Vilkmes leņķis}

Vilkmes leņķa distribūcija starp ar krāsām saistītām strūklām - no \leadingjet uz \scndleadingjet ar visām strūklu komponentēm un ar jebkuru \DeltaR ir attēlota \ref{fig:_pull_angle_allconst_reco_leading_jet_scnd_leading_jet_DeltaRTotal}. att.

\figureEML{/reco/pull_angle/DeltaRTotal/charge/allconst/}
          {_pull_angle_allconst_reco_leading_jet_scnd_leading_jet_DeltaRTotal}
          {Vilkmes leņķā distribūčija no \leadingjet uz \scndleadingjet ar jebkuru \DeltaR un iekļaujot visas strūklu komponentes.}


\ref{fig:_pull_angle_allconst_reco_leading_b_scnd_leading_b_DeltaRTotal}. att. attēlotas vilkmes leņķa distribūcijas, gadījumos, kas nav sagaidāma krāsu saistība starp strūklām - no \leadingb uz \scndleadingb un no \scndleadingb uz \leadingb iekļaujot visas strūklu sastāvdaļās un pie visām \DeltaR vērtībām.

\figureEML{/reco/pull_angle/DeltaRTotal/charge/allconst/}
          {_pull_angle_allconst_reco_leading_b_scnd_leading_b_DeltaRTotal}
          {Vilkmes leņķa no \leadingb uz \scndleadingb distribūcija pie visām \DeltaR vērtībām un iekļaujot visas strūklu sastāvdaļas.}

Papildu iespēja novērto vilkmes leņķa distribūciju starp objektiem, starp kuriem nav krāsu saistība, ir izvēlēties strūklu un leptonu. \ref{fig:_pull_angle_allconst_reco_leading_jet_lepton_DeltaRTotal}. att. attēlota vilkmes leņķā distribūcija starp \leadingjet un lādēto leptonu. 

\figureEML{/reco/pull_angle/DeltaRTotal/charge/allconst/}
          {_pull_angle_allconst_reco_leading_jet_lepton_DeltaRTotal}
          {Vilkmes leņķa distribūcija no \leadingjet uz lādēto leptonu pie visām \DeltaR vērtībām un iekļaujot visas strūklu sastāvdaļas.}

Attēlos redzams, ka centrālais paugurs vilkmes leņķa distribūcijā ir acīmredzams, ja iesaistītas ar krāsām saistības strūklas, kā arī tas izlīdzinās, ja ir iesaistīti ar krāsām nesaistīti objekti. 

Centrālais paugurs var būt redzams gadījumos, ja starp fizikālo objektu vekotriem pastāv kolinearitāte, kaut arī paši fizikas objekti nav saistīti ar krāsām. Šāds gadījums redzams, apskatot vilkmes leņķa distribūciju no \leadingjet uz hadronisko \PW - \ref{fig:_pull_angle_allconst_reco_leading_jet_had_w_DeltaRTotal}. att. 

\figureEML{/reco/pull_angle/DeltaRTotal/charge/allconst/}
          {_pull_angle_allconst_reco_leading_jet_had_w_DeltaRTotal}
          {Vilkmes leņķa distribūcija no \leadingjet uz hadronisko \PW pie visām \DeltaR vērtībām un iekļaujot visas strūklu sastāvdaļas.}

Interesants gadījums ir kūlis. \ref{fig:_pull_angle_allconst_reco_leading_jet_beam_DeltaRTotal}. att. attēlota vilkmes leņķa distribūcija no \leadingjet uz kūļa pozitīvo virzienu. Redzam pauguru taisnā leņķī.

\figureEML{/reco/pull_angle/DeltaRTotal/charge/allconst/}
          {_pull_angle_allconst_reco_leading_jet_beam_DeltaRTotal}
          {Vilkmes leņķa distribūcija no \leadingjet uz kūļa pozitīvo virzienu, iekļaujot visas strūklas sastāvdaļas.}

\section{\DeltaR ietekme}

Gadījumos, kad divas strūklas atrodas cieši viena pie otras $\eta-\phi$ telpā, strūklu sakopošanas algoritms mēdz pievienot vienas strūklas (ar mazāko \pt) otrai strūklai (ar lielāko \pt). Tas atstāj iespaidu uz analīzi ar vilkmes leņķi, jo vilkmes vektoram būs nosliece rādīt uz strūklu, no kuras tika atņemtas sastāvdaļas. \ref{fig:_pull_angle_allconst_reco_leading_jet_scnd_leading_jet_DeltaRle1p0}. att. un \ref{fig:_pull_angle_chconst_reco_leading_jet_scnd_leading_jet_DeltaRgt1p0}. att. attēloti divi gadījumi - cieši klātesošas strūklas ar $\DeltaR\leq1.0$ un attālas strūklas ar $\DeltaR>1.0$.

\figureEML{/reco/pull_angle/DeltaRle1p0/charge/allconst/}
          {_pull_angle_allconst_reco_leading_jet_scnd_leading_jet_DeltaRle1p0}
          {Vilkmes leņķa distribūcija pie \DeltaR$\leq1.0$ un iekļaujot visas strūklu sastāvdaļās no \leadingjet uz \scndleadingjet.}

\figureEML{/reco/pull_angle/DeltaRgt1p0/charge/allconst/}
          {_pull_angle_allconst_reco_leading_jet_scnd_leading_jet_DeltaRgt1p0}
          {Vilkmes leņķa distribūcija pie \DeltaR$>1.0$ un iekļaujot visas strūklu sastāvdaļās no \leadingjet uz \scndleadingjet.}


\section{Jutīguma analīze}

Vilkmes leņķa metodoloģijas jutīguma tika pētīta, pielietojot sekojošus \gls{sliekšņus}:

1. Hadroniskā \PW bozona \pt. Tika izvēlēts 50 \GeV slieksnies un tika iegūtas vilkmes leņķas distribūcijas hadroniskā \PW bozona \pt esot zemākai vai pārsniedzot šo slieksni. Rezultāti ir attainoti \ref{fig:_pull_angle_hadWPtgt50p0GeV_reco_leading_jet_scnd_leading_jet_DeltaRTotal}. att. - \ref{fig:_pull_angle_hadWPtle50p0GeV_reco_leading_jet_scnd_leading_jet_DeltaRTotal}. att.

2. Strūklas sastāvdaļu skaits. Tika izvēlēts strūklas sastāvdaļu skaita $N$ slieksnies vienāds ar 20 un tika iegūtas vilkmes leņķa distribūcijas $N$ pārsniedzot vai esot zemākai par šo slieksni. Rezultāti ir atainoti \ref{fig:_pull_angle_PFNgt20_reco_leading_jet_scnd_leading_jet_DeltaRTotal}. att. - \ref{fig:_pull_angle_PFNle20_reco_leading_jet_scnd_leading_jet_DeltaRTotal}. att.
                                        
3. Strūklas sastāvdaļu \pt. Tika izvēlēts strūklas sastāvdaļu \pt slieksnis vienāds ar 0.5\GeV un tika iegūtas vilkmes leņķā distribūcijas strūklu sastāvdaļu \pt esot lielākam vai mazākam par šo slieksni. Rezultāti ir attēloti \ref{fig:_pull_angle_PFPtgt0p5GeV_reco_leading_jet_scnd_leading_jet_DeltaRTotal}. att. - \ref{fig:_pull_angle_PFPtle0p5GeV_reco_leading_jet_scnd_leading_jet_DeltaRTotal}. att.

4. Vilkmes vektora lielums.  Tika izvēlēts vilkmes vektora lieluma slieksnis vienāds ar 0.005 [bez mērvienībām] un tika iegūtas vilkmes leņķa distribūcijas vilmes vektora lieluma pārsniedzot vai esot mazākam par šo slieksni. Rezultāti ir attēloti \ref{fig:_pull_angle_PVMaggt0p005_reco_leading_jet_scnd_leading_jet_DeltaRTotal}. att. - \ref{fig:_pull_angle_PVMagle0p005_reco_leading_jet_scnd_leading_jet_DeltaRTotal}. att.


\figureEML{/reco/pull_angle/DeltaRTotal/hadronic_W_Pt/hadWPtgt50p0GeV/}
          {_pull_angle_hadWPtgt50p0GeV_reco_leading_jet_scnd_leading_jet_DeltaRTotal}
          {Vilkmes leņķa distribūcija pie visiem \DeltaR un iekļaujot visas strūklu sastāvdaļas no \leadingjet uz \scndleadingjet ar \PW bozona \pt $>$ 50 \GeV.}

\figureEML{/reco/pull_angle/DeltaRTotal/hadronic_W_Pt/hadWPtle50p0GeV/}
          {_pull_angle_hadWPtle50p0GeV_reco_leading_jet_scnd_leading_jet_DeltaRTotal}
          {Vilkmes leņķa distribūcija pie visiem \DeltaR un iekļaujot visas strūklu sastāvdaļas no \leadingjet uz \scndleadingjet ar \PW bozona \pt $\leq$ 50 \GeV.}
          

\figureEML{/reco/pull_angle/DeltaRTotal/PF_number/PFNgt20/}
          {_pull_angle_PFNgt20_reco_leading_jet_scnd_leading_jet_DeltaRTotal}
          {Vilkmes leņķa distribūcija pie visiem \DeltaR un iekļaujot visas strūklu sastāvdaļas no \leadingjet uz \scndleadingjet ar strūklas sastāvdaļu skaitu $N>20$.}

\figureEML{/reco/pull_angle/DeltaRTotal/PF_number/PFNle20/}
          {_pull_angle_PFNle20_reco_leading_jet_scnd_leading_jet_DeltaRTotal}
          {Vilkmes leņķa distribūcija pie visiem \DeltaR un iekļaujot visas strūklu sastāvdaļas no \leadingjet uz \scndleadingjet ar strūklas sastāvdaļu skaitu $N\leq20$.}

\figureEML{/reco/pull_angle/DeltaRTotal/PF_Pt/PFPtgt0p5GeV/}
          {_pull_angle_PFPtgt0p5GeV_reco_leading_jet_scnd_leading_jet_DeltaRTotal}
          {Vilkmes leņķa distribūcija pie visiem \DeltaR un iekļaujot visas strūklu sastāvdaļas no \leadingjet uz \scndleadingjet, ja strūklas sastāvdaļu \pt$>$0.5 \GeV.}

\figureEML{/reco/pull_angle/DeltaRTotal/PF_Pt/PFPtle0p5GeV/}
          {_pull_angle_PFPtle0p5GeV_reco_leading_jet_scnd_leading_jet_DeltaRTotal}
          {Vilkmes leņķa distribūcija pie visiem \DeltaR un iekļaujot visas strūklu sastāvdaļas no \leadingjet uz \scndleadingjet, ja strūklas sastāvdaļu \pt $\leq$ 0.5 \GeV.}

\figureEML{/reco/pull_angle/DeltaRTotal/PV_magnitude/PVMaggt0p005}
          {_pull_angle_PVMaggt0p005_reco_leading_jet_scnd_leading_jet_DeltaRTotal}
          {Vilkmes leņķa distribūcija pie visiem \DeltaR un iekļaujot visas strūklu sastāvdaļas no \leadingjet uz \scndleadingjet ja vilkmes vektora lielums $> $0.005 [bez mērvienībām].}

\figureEML{/reco/pull_angle/DeltaRTotal/PV_magnitude/PVMagle0p005}
          {_pull_angle_PVMagle0p005_reco_leading_jet_scnd_leading_jet_DeltaRTotal}
          {Vilkmes leņķa distribūcija pie visiem \DeltaR un iekļaujot visas strūklu sastāvdaļas no \leadingjet uz \scndleadingjet, ja vilkmes vektora lielums $\leq$0.005 [bez mērvienībām].}

Vilkmes leņķa metodoloģija ir jutīga pret hadroniskā \PW bozona \pt, strūklas sastāvdaļu skaitu, strūklas sastāvdaļu \pt, bet ne īpaši jutīga pret vilkmes vektora lieluma. 



%\clearpage
\section{Atlocīšana}
Kad eksperimentētājs veic novērojumu ar detektoru, jāsamierinās ar paša detektora ietekmi uz rezultātu \textendash\ kropļojumiem. \gls{Atlocīšana} ir metode, ar kuras palīdzību detektora novērojami tiek koriģēti, ievērojot detektora radīto ietekmi. Tādējādi mēs iegūstam patieso novērojamā lieluma sadalījumu. Tomēr jārēķinās, ka atlocītā novērojamā lieluma fāzes telpa kļūst ievērojami graudaināka. 

Detektora ietekmi mēs varam novērtēt tādēļ, ka, strādājot ar Montekarlo paraugiem, katrs ģenerētais notikums tiek rekonstruēts, izmantojot detektora simulāciju. Rekonstrukcijas procesā novērojamais lielums vērtību intervālā $i$ ģenerācijas līmenī migrē uz vertību intervālu $k$ rekonstrukcijas līmenī. Apskatot lielu skaitu notikumu, iegūstam migrācijas statistiku. Atlocīšanas ceļā veicam šai migrācijai pretēju procesu - ja dots novērojamais lielums vērtību intervālā $k$, mēs piešķiram varbūtības dažādām novērojamā lieluma patiesajām vērtībām.

Tās \pullangle vērtības ģenerēšanas līmenī, kurām nav atbilstošas vērtības rekonstrukcijas līmenī, tiek ievietotas \gls{pirmsintervālā} rekonstrukcijas līmenī. Tās \pullangle vērtības rekonstrukcijas līmenī, kurām nav atbilstošas vērtības ģenerācijas līmenī, tiek ievietotas pirmsintervālā ģenerācijas līmenī. Pirmsintervāls ģenerācijas līmenī tiek uzskatīts par fonu, un tas tiek iztukšots. Sadalījumi, kas netiek pildīti ģenerācijas līmenī \textendash\ dati un MK fons \testendash\ tiek samazināti ar atbilstošu proporcijas koeficientu. Pirmsintervāls rekonstrukcijas līmenī tiek izmantots, lai ierobežotu atlocītā rezultāta pirmsintervālu. 

Atlocīšana tiek veikta ar datiem, no kuriem ir atņemts MK fons. Atlocīšana tika veikta arī prerēji, iegūstot atpakaļatlocīto rezultātu. 

Lai samazinātu atlocītā rezultāta satistisko nenoteiktību, esam ieinteresēti, lai migrācijas matrica būtu pēc iespējas diagonāla. Lai raksturotu notikumu skaitu, kas sakopoti uz migrācijas matricas diagonāles, tiek lietoti divi lielumi: stabilitāte un tīrība. Stabilitāte ir attiecība starp diagonāles saturu un kopējo notikumu skaitu rekonstrukcijas līmenī vērtību intervālā:

\begin{equation}
  \text{stabilitāte}\equiv\frac{\theta^{\text{diag}}_{\text{ievade}}}{\Sigma_{x=1}^{x=N_{x}}\theta^{x}_{\text{ievade}}},
\end{equation}

\noindent kur $x$ ir vērtību intervāla indekss rekonstrukcijas līmenī, numerāciju sākot no 1, un $N_{x}$ ir vērtību intervālu skaits rekonstrukcijas līmenī. Tīrība ir attiecība starp diagonāles saturu un kopējo notikumu skaitu ģenerācijas līmenī:

\begin{equation}
  \text{tīrība}\equiv\frac{\theta^{\text{diag}}_{\text{ievade}}}{\Sigma_{y=1}^{y=N_{y}}\theta^{y}_{\text{ievade}}},
\end{equation}

\noindent kur $y$ ir vērtību intervāla indekss ģenerācijas līmenī. Ir ieteicams, lai tīrības un stabilitātes vertības pārsniegtu 50~\% katrā vērtību intervālā.

Apskatām arī interesantu rādītāju, kas raksturo, par cik atlocītais rezultāts ir atšķirīgs no ģenerētā MK rezultāta (ideāls rādītajs būtu 0), normalizētu pret atlocīta rezultāta statistisko nenoteiktību. Šo rādītāju sauc par vilkmi:

\begin{equation}
  \text{vilkme}\equiv\frac{\theta^{\text{ģen}}_{\text{atl}}-\theta^{\text{ģen}}_{\text{ievade}}}{\sigma^{\text{ģen}}_{\text{atl}}}.
\end{equation}

Lai iegūtu vilkmes sadalījumu, ģenerējam novērojamā lieluma gadījuma spēļu sadalījumus ģenerācijas līmenī.

Vērtību intervālu skaits ģenerāciju līmenī tiek samazināts divas reizes, salīdzinot ar intervālu skaitu rekonstrukcijas līmenī, lai atlocīšana būtu skaitliski iespējama.

Atlocīšanas procesa īstenošanai tiek izmantota \ROOT klase \lstinline[language=sh]|TUnfoldDensity|~\cite{Schmitt:2012kp}. Intervālu shēma tiek pārvaldīta ar \lstinline[language=sh]|TUnfoldBinning| klasi. Netiek lietota regularizācija. \pullangle no \leadingjet uz \scndleadingjet atlocīšanas rezultāti, iekļaujot visas strūklas sastāvdaļas, ir atainoti \ref{fig:unfolding_nominal_leading_jet_allconst_pull_angle_OPT_MC13TeV_TTJets_ATLAS3}~att. Lai radītu šos attēlus, tika izveidota jauna klase \lstinline[language=sh]|CompoundHistoUnfolding|~\cite{url:compoundhistounfolding}, kas tika pievienota \ROOT, t.sk. ar ievades un izvades straumētājiem.

Atlocīšanas rezultātiem ir uzklāti sadalījumi, kas atbilst atlocīšas rezultātiem, kas iegūti ar migrācijas matricām, kas savukārt iegūtas no $\ttbar\ Herwig++$ un $\ttbar\ cflip$ paraugiem, kā arī sistemātikas $\ttbar\ fsr\ dn$ un $\ttbar\ fsr\ up$ (skat. \ref{chap:systematic_uncertainties}~nod.).

Statistisko \gls{traucējumu} nozīmība katrā vērtību intervālā kopējā atlocītā rezultāta sistemātiskajā kļūdā  (\%) ir norādīta \ref{tab:unc_table_fullpull_angle_OPT_allconst_gen_out_MC13TeV_TTJets_nominal_ATLAS3}~tab. Traucējumi, kas tieši ietekmē hadronizāciju - $\ttbar\ Herwig++$, $\ttbar\ QCDbased$ un $\ttbar\ ERDon$ ir visnozīmīgākie.

Tiek novērtēta arī sakritība starp atlocīto rezultātu un MK paredzējumu ģeneratora līmenī. Ja dots atlocīts novērojamais lielums $D$, normalizētais MK paredzējums $M$, pilnā eksperimentālo nenoteiktību kovariances matrica $\Sigma$, $\chi^{2}$ tiek aprēķināts šādi:

\begin{equation}
  \chi^{2}=(D^{T}-M^{T})\cdot\Sigma^{-1}\cdot(D-M).
  \label{eq:chi2}
\end{equation}

No $\chi^{2}$ vērtības iespējams aprēķināt $p$-vērtību, zinot, ka brīvību skaits ir vienāds ar atlocītā sadalījuma vertības intervālu skaitu, no kura atņemts 1, ņemot vēro brīvības zaudējumu, veicot sadalījuma normalizāciju. No kovariances matricas tiek atmesta viena rinda un viena kolonna $\Sigma$. Atmetamo elementu izvēle neietekmē $\chi^{2}$ vērtību.

\ref{tab:chi_table_pull_angle_OPT_allconst_nominal_ATLAS3}~tab. sniegtas \pullangle $\chi^{2}$ vērtības un $p$-vērtības, ja ietvertas visas strūklas sastāvdaļas. No rezultātiem redzams, ka MK ģeneratori diezgan neprecīzi modelē vilkmes leņķa sadalījumu. Simulācijas, vispārīgi skatot, paredz stāvāku vilkmes leņķa sadalījumu, t. i. izteiktāku krāsu plūsmas efektu. \HERWIGpp vilkmes leņķa sadalījumu modelē labāk nekā \PYTHIA 8.2. \PYTHIA 8.2 precizitāte ir jo īpaši vāja, paredzot vilkmes leņķa sadalījumu no \scndleadingjet uz \leadingjet.

$\chi^{2}$ vērtības un $p$-vērtības krāsu okteta \PW bozona modelim ir sniegtas \ref{tab:chi_table_pull_angle_OPT_allconst_cflip_ATLAS3}~tab. \POWHEG+\PYTHIA 8 * ailē $\ttbar\ cflip$ paragus ir ticis pievienots kā \ttbar sistemātiskā nenoteiktīb. Krāsu apmaiņas modelī \pullangle no \leadingjet uz \scndleadingjet sadalījums ir modelēts neprecīzāk nekā SM paredzējums. 
  
%% \ref{tab:chi_table_pull_angle_OPT_allconst_MC13TeV_TTJets_nominal_ATLAS3_full} tab. sniegtas $\chi^{2}$ vērtības, ja signāls $M$ \ref{eq:chi2} izteiksmē tiek aizvietots ar attiecīgo sistemātiku, bet saglabājot kovariances matricu $\Sigma$ nemainīgu. Sakritība ir labāka nekā \ttbar, ja krāsu plūsma tiek modelēta ar \ttbar ERDOn, \ttbar Herwig++ vai \ttbar QCD based.
\figunfoldinglv{nominal}{leading_jet}{allconst}{pull_angle}{ATLAS3}{MC13TeV_TTJets}{Vilkmes leņķa no \leadingjet uz \scndleadingjet atlocīšanas rezultāti \ttbar metodei, iekļaujot visas strūklas sastāvdaļas un izmantojot trīs vienāda izmēra vērtību intervālus}

%\Needspace{6\baselineskip}
%\newpage
\input{tables/unc_nominal_fulllv/pull_angle/ATLAS3/unc_table_full_leading_jet_allconst_pull_angle_OPT_gen_out_MC13TeV_TTJets.txt}
\input{tables/unc_cflip_fulllv/pull_angle/ATLAS3/unc_table_full_leading_jet_allconst_pull_angle_OPT_gen_out_MC13TeV_TTJets.txt}
\input{tables/chi_nominallv/pull_angle/ATLAS3/chi_table_pull_angle_OPT_allconst.txt}
\input{tables/chi_cfliplv/pull_angle/ATLAS3/chi_table_pull_angle_OPT_allconst.txt}

%\input{tables/chi_nominallv/pvmag/ATLAS3/chi_table_pvmag_OPT_allconst.txt}


\clearpage
\section{LEP metode}
\label{sec:LEP_methodology}

Tiek analizētas trīs veidu plūsmas:
\begin{itemize}
\item daļiņu plūsmā visiem notikumiem tiek piešķirsts svars, vienāds ar 1,0,
\item enerģijas plūsmā daļiņām tiek piešķirts svars, kas vienāds ar to enerģijas attiecību pret virsotnes kvarku enerģijas summu,
\item \pt plūsmā daļiņām tiek piešķirts svars, kas vienāds ar to šķērsmomentu attiecību pret attiecīgās strūklas šķērsmomentu.
\end{itemize}

Rezultāti, izmantojot LEP metodoloģiju, iekļaujot visas strūklas sastāvdaļās, ir attēloti \ref{fig:chi_allconst_N}~att. Tiek attēlotas plūsmas starp vadošo \cPqb strūklu \leadingb un otru vadošo \cPqb strūklu \scndleadingb, hadronisko \cPqb strūklu $j_{h}^{\cPqb}$ un tālāko vieglo strūklu $j_{f}^{\PW}$ (attālums starp strūklām ir mērīts kā leņķis starp strūklu 4-vektoru telpiskajām komponentēm), tuvāko vieglo strūklu $j_{c}^{\PW}$ un hadroniskā \cPqb strūklu $j_{h}^{\cPqb}$, un vadošo vieglo strūklu \leadingjet un otro vadošo vieglo strūklu \scndleadingjet.

Visos gadījumos blīvums samazinās centrālajā reģionā starp strūklām. Blīvums centrālajā regionā atkarīgs no tā, vai strūklas ir vai nav saistītas ar krāsām.

Plūsmas attiecība no krāsām brīvajos reģionos (\leadingb, \scndleadingb), ($j_{h}^{\cPqb}$, $j_{f}^{\PW}$), ($j_{c}^{\PW}$, $j_{h}^{\cPqb}$) pret plūsmu ar krāsam saistītajā reģionā (\leadingjet, \scndleadingjet), iekļaujot visas strūklu sastāvdaļas, ir attēlota \ref{fig:chirg_allconst_N}~att. Krāsu okteta \PW modeļa gadījumā novērojama atkārtota krāsus saistība reģionā ($j_{c}^{\PW}$, $j_{h}^{\cPqb}$).

\ref{fig:ratio_hbqc}~att. attēlota daļiņu plūsmas attiecība (\leadingjet, \scndleadingjet) reģionā atbilstoši krāsu okteta \PW modelim pret daļiņu plūsmu (\leadingjet, \scndleadingjet) reģionā atbilstoši Standarta modelim. Redzams, ka \PW okteta modeļa gadījumā novērojama krāsu saistības izzušana.

Kā kvantitatīvu LEP metodoloģijas rezultātu var izmantot parametru $R$, kas tiek definēts kā attiecība starp integrāli no 0,2 līdz 0,8 ar krāsām saistītajā reģionā pret integrāli no 0,2 līdz 0,8 ar krāsām nesaistītā regionā:

\begin{equation}
R=\frac{\int_{0.2}^{0.8}f^{\text{inter \PW reģions}}d\chi}{\int_{0.2}^{0.8}f^{\text{intra \PW reģions}}d\chi},
\end{equation}

\noindent kur $f(\chi)$ ir plūsmas sadalījuma blīvums.

Šis parametrs LEP tika izmantots, lai kvantificētu krāsu saistību. Tā vērtības, kas iegūtas dažādos eksperimentos, izmantojot 625~\pbinv integrēto spīdumu intervālā \sqrts=189-209~\GeV, ir sniegtas \ref{tab:LEP_R}~tab. Vērojama dažādos eksperimentos novēroto $R$ vērtību nesakritība. Turklāt, balstoties uz teorētiskiem apsvērumiem, $R$ būtu jāpārsniedz 1.

\begin{table}
\centering
\caption{LEP novērotās $R$ vērtības.}
\label{tab:LEP_R}
\begin{tabular}{lll}
LEP eksperiments & $R$ vērtība                                             & Atsauce\\
\hline
    OPAL         & 1,243                                                   & \cite{Abbiendi:2005es}\\
    Delphi       & 0,889 ($\sqrt{s}=183$~\GeV)-1,039 ($\sqrt{s}=207$~\GeV) & \cite{Abdallah:2006uq}\\
    L3           & 0,911                                                   & \cite{Achard:2003pe}\\
  \end{tabular}
\end{table} 

Mūsu gadījumā lietojam 3 $R$ vērtības, kas atbilst 3 ar krāsām nesaistītajiem reģioniem.

Integrālis no 0,2 līdz 0,8 dažādos reģionos un inversas $R$ vērtības Standarta modelim ir sniegtas \ref{tab:R_L_reco_N_MC_SM}~tab., datiem \ref{tab:R_L_reco_N_data_SM}~tab. un krāsu okteta \PW modelim \ref{tab:R_L_reco_N_MC_cflip}~tab.

\figureChilv{allconst}{N}{Daļiņu plūsmas histogramas, iekļaujot visas strūklu daļiņas.}

\figureratiographslv{allconst}{N}{Daļiņu plūsma, iekļaujot visas strūklu daļiņas, attiecībā pret daļiņu plūsmu $\leadingjet, \scndleadingjet$ apgabalā.}

\begin{figure}[htpb]
\def\twidth{0.6}
\centering
\includegraphics[width=\twidth\textwidth]{fig/ratiographs_merged_SM/L_qlq2l_N_allconst_reco.png}
\caption{Attiecība starp daļiņu plūsmu (\leadingjet, \scndleadingjet) reģionā krāsu okteta \PW modelim pret daļiņu plūsmu (\leadingjet, \scndleadingjet) reģionā atbilstoši Standarta modelim.}
\label{fig:ratio_hbqc}
\end{figure}

\input{tables/Rvalues_SMlv/R_L_reco_MC_N_SM.txt}

\input{tables/Rvalues_SMlv/R_L_reco_data_N_SM.txt}

\input{tables/Rvalues_cfliplv/R_L_reco_MC_N_cflip.txt}




\clearpage
\section{Hipotēžu pārbaude}
\label{subsec:hypo_testing}

Mūsu rīcībā esošie krāsu maiņas MK paraugi sniedz iespēju gūt priekšstatu par to, vai krāsu okteta \PW signāls ir redzams datos. Šādi rezultāti jāapskata ar piesardzību, jo sakritība starp datiem un MK paraugiem nav īpaši laba. Izmantosim metodi, kuru daļiņu fiziķi lieto atklājumu veikšanai: tikai fona hipotēzes pārbaude pret signāla + fona hipotēzi ar zīmīgumu $Z$ vismaz 5~\cite{Cowan:2010js}. Pirmā hipotēze tiek dēvēta par nulles hipotēzi \Hnull, bet otrā hipotēze tiek dēvēta par alternatīvo hipotēzi \Halt.

Mēs konstruējam divu hipotēžu modeli, lai kombinētu fona, \ttbar un krāsu maiņas \ttbar signālus:

\begin{equation}
  n=\mu\left(\left(1-x\right)f_{t\overline{t}} + xf_{t\overline{t}_{\text{cflip}}}\right) + b,
  \label{eq:two_hypo_model}
\end{equation}

\noindent kur $n$ ir sagaidāmais notikumu skaits, $\mu$ - signāla stiprums, $x$  - parametrs, kas tiek lietots, lai piešķirtu svaru \ttbar un krāsu maiņas \ttbar signālam tā, lai to kopējais svars būtu vienāds ar 1, $b$ - MK fons. Turpmākajā datoranalīzē $\mu$ ir iestatīts kā 1, un $x$ iestatīts kā interesējošais parametrs.

Kā testa statistisko parametru izmantojoam Tevatrona testa statistisko parametru. Tas pazīstams arī kā Neimana-Pīrsona (oriģ. \textit{Neyman-Pearson}) testa statistiskais parametrs. Tevatrona testa statiskais parametrs tiek definēts kā:

\begin{equation}
  q^{\text{TEV}}=-2\ln{\frac{L(\Hnull)}{L(\Halt)}}=-2\ln{\frac{L\left(\text{data}|p=0,\hat{\theta}_{0}\right)}{L\left(\text{data}|p=P,\hat{\theta}_{P}\right)}},
\end{equation}

\noindent kur $p$ ir interesējošais parametrs, $\theta$ ir traucējumu faktors un $\hat{\theta}$ ir traucējumu faktors, kas maksimizē profila iespējamību. Profila iespējamība $L$ tiek definēta kā hipotēzes varbūtība pie dotajiem datiem. Pieņemot hipotēzi ar signāla stiprumu $\mu$, iespējamība tiek aprēķināta kā:

\begin{equation}
  L\left(\mu, \theta_{s}, \theta_{b}\right) = \prod_{i=1}^{N}\frac{\left(\mu s_{i}\left(\theta_{s}\right) + b_{i}\left(\theta_{b}\right)\right)^{n_{i}}}{n_{i}!}e^{-\left(\mu s_{i}\left(\theta_{s}\right) + b_{i}\left(\theta_{b}\right)\right)},
\end{equation}

\noindent kur $i$ ir fāžu telpas parametrs (vērtību intervāla indekss), $n_{i}$ ir novērojumi (dati) attiecīgajā fāzē (vērtību intervālā).

Tevatrona testa statistiskais parametrs mūs interesē tādēļ, ka $x$ ir iestatīts kā interesējošais parametrs divu hipotēžu modelī \ref{eq:two_hypo_model}~vien. un $P$ ir iestatīts kā 1. Līdz ar to, izmantojot $q^{TEV}$ statistisko parametru, \Hnull (ar $x=0$) tiek definēta kā $t\overline{t}+b$ sadalījums, kamēr \Halt ir definēta kā $t\overline{t}_{\text{cflip}}+b$ sadalījums.

Lai testētu \Hnull un \Halt hipotēzes, nepieciešams aprēķināt to p-vērtības. Labās puses p-vērtība tiek aprēķināta kā

\begin{equation}
p\equiv\int_{q_{obs}}^{\infty}f(q)dq,
\end{equation}
    
\noindent kur $q_{obs}$ ir testa statistiskā parametra vērtība, kas novērota datos, un $f$ varbūtības sadalījuma funkcija atbilstoši pieņemtajai hipotēzei. Zema p-vērtība liecina pret pieņemto hipotēzi. Nozīmīgums $Z=5$ atbilst p-vērtībai $2,87\times10^{-7}$. Neimana-Pīrsona testa statistiskais parametrs atbilstoši \Hnull ir rēķināms no labās puses, bet \Halt ir rēķināms no kreisās puses. Šī sakarība attēlota \ref{fig:npstatistic}~att.

\begin{figure}
  \centering
  \includegraphics[width = 0.6\textwidth]{fig/npstatistic.pdf}
  \caption{Hipotēžu pārbaude atbilsoši Neimana-Pīrsona testa statistikai.}
  \label{fig:npstatistic}
\end{figure}

Hipotēžu pārbaudei un visu sagatavošanas darbu veikšanai mēs lietojam CMS \lstinline[language=sh]|combine| rīku~\cite{url:combine}.
%Datu karte, kas izmantota, lai radītu RooFit \cite{url:roofit} darba telpu ir sniegta \ref{a:datacard} pielikumā.
Testa statistiskā parametra sadalījuma ģenerēšanai izmantojam  \lstinline[language=sh]|combine| \lstinline[language=sh]|HybridNew| metodi. Lai aprēķinu teorētiskos testa statistiskā parametra sadalījumus, dati tiek novērtēti no MK paraugiem atbilstoši \textit{frequentist} pieejai. \lstinline[language=sh]|HybridNew| metodes pielietojums sniegts sekojošā komandu sarakstā:

\begin{lstlisting}[language=sh, breaklines=true]
  combine -M HybridNew -T 500 -i 2 --fork 6 --clsAcc 0 --fullBToys -m 125.7 TwoHypo.root --seed 8192 --testStat=TEV  --saveHybridResult --singlePoint 1
\end{lstlisting}

\noindent kur  \lstinline[language=sh]|TwoHypo.root| ir \ROOT datne, kas satur darba telpu, \lstinline[language=sh]|--singlePoint 1| nozīmē, ka pieprasām, lai $x$ - interesējošais parametrs \ref{eq:two_hypo_model}~vien. būtu vienāds ar 1 \Halt gadījumā. Šajā stadijā izmantojam tikai 500 izmēģinājumus. $\frac{q}{2}$ sadalījums, kur $q$ ir testa statistiskais parametrs, pieņemot \Hnull un \Halt, kā arī $\frac{q_{\text{obs}}}{2}$ vērtība ir sniegta \ref{fig:hypo1p0}~att.

\begin{figure}
  \centering
  \includegraphics[width = 0.6\textwidth]{fig/hypo1p0.png}
  \caption{$\frac{q}{2}$ sadalījums, pieņemot \ttbar hipotēzi (sarkanā), krāsām mainītu \ttbar hipotēzi (zilā) un $\frac{q_{\text{obs}}}{2}$.}
  \label{fig:hypo1p0}
\end{figure}

\Halt un \Hnull p-vērtības ir tuvas nullei. Līdz ar to nav iespējams izdarīt secinājumus - nevar noraidīt \Hnull par labu \Halt un noraidīt \Halt par labu \Hnull.

\lstinline[language=sh]|combine| rīks ietver \lstinline[language=sh]|MultiDimFit| metodi, ar kuru var iegūt profila iespējamības attiecības (PLR) līkni. Profila iespējamības attiecība tiek aprēķināta kā

\begin{equation}
  \text{PLR}(x, \theta)=-2\ln\frac{L(x=0, \theta)}{\hat{x}, \hat{\theta}}.
\end{equation}

Pie $\hat{x}$ un $\hat{\theta}$ PLR sasniedz minimumu. Šajā punktā MK vislabāk saskan ar datiem. PLR līkni var iegūt ar sekojošām komandām

\begin{lstlisting}[language=sh, breaklines=true]
combine -M MultiDimFit --algo grid --points 50 TwoHypo.root
\end{lstlisting}

PLR līkne ir attēlota \ref{fig:likelihood}~att. un tās minimums ir pie $x=0,335$.

\begin{figure}
  \centering
  \includegraphics[width = 0.6\textwidth]{fig/likelihood}
  \caption{PLR līkne kā funkcija no $x$.}
  \label{fig:likelihood}
\end{figure}

Aprēķinot iespējamību, \lstinline[language=sh]|combine| rīks kombinē nominālo signālu ar traucējumiem un meklē kombināciju, kas maksimizē profila iespējamību. Dažādiem traucējumiem ir dažāda ietekme. Traucējuma parametra $\theta$ ietekme tiek definēta kā interesējošā parametra nobīde $\Delta x$, iekļaujot traucējumu pie tā $\pm\sigma$ vērtībām:

\begin{equation}
  \Delta x = x\bigg\rvert_{\theta \text{at} \pm\sigma}-x_{0}.
\end{equation} 

Lai iegūtu maksimālo profila iespējamību, dažādi traucējumi jāpiemēro dažādos apjomos. Traucējuma parametra $\theta$ vilkme, kas kvantificē šo apjomu, tiek definēta kā:

\begin{equation}
  P = \frac{\hat{\theta}-\theta_{0}}{\delta\theta},
\end{equation} 

\noindent kur $\hat{\theta}$ ir $\theta$, kas maksimizē profila iespējamību, $\theta_{0}$ ir pirmspiemērošanas vērtība, $\delta\theta$ - pirmspiemērošanas nenoteiktība.

Lai noteiktu traucējumu parametru ietekmi un vilkmi, izmantojam \lstinline[language=sh]|combine| rīka \lstinline[language=sh]|Impact| metodi. Lietotās komandas ir sekojošas:

\begin{lstlisting}[language=sh, breaklines=true]
  combineTool.py -M Impacts -d TwoHypo.root -m 125.7 --doInitialFit --robustFit 1
  combineTool.py -M Impacts -d TwoHypo.root -m 125.7 --robustFit 1 --doFits
  combineTool.py -M Impacts -d TwoHypo.root -m 125.7 -o impacts.json
  plotImpacts.py -i impacts.json -o impacts
\end{lstlisting}

Dažādo traucējuma parametru ietekme un vilkme ir attēlota \ref{fig:impacts}~att.

\begin{figure}
  \centering
  \includegraphics[width = 1\textwidth]{fig/impacts.pdf}
  \caption{Dažādu traucējumu parametru ietekme un vilkme.}
  \label{fig:impacts}
\end{figure}

Pēc tam, kad esam ieguvuši, ka $\hat{x}=0,335$ (\ref{fig:likelihood}~att.), varam atgriezties pie hipotēžu pārbaudes, šoreiz nosakot, ka $x=\hat{x}$. Šajā gadījumā mēs pārbaudīsim hipotēzi, saskaņā ar kuru signāls sastāv tikai no \ttbar (\Hnull) pret hipotēzi, saskaņā ar kuru signāls ir kombinēts no 66,5~\% \ttbar procesa un 33,5~\% krāsu apmainītā \ttbar procesa (\Halt). Testa statistiskā parametra sadalījums pie $x=\hat{x}$ ir attēlots \ref{fig:hypo0p335}~att.

\begin{figure}
  \centering
  \includegraphics[width = 0.6\textwidth]{fig/hypo0p335.png}
  \caption{$\frac{q}{2}$ sadalījums, pieņemot vienīgi \ttbar hipotēzi (sarkanā), hipotēzi saskaņā ar kuru signāls sastāv no 66,5~\% \ttbar un 33,5~\% krāsu apmainītā \ttbar procesa (zilā) un $\frac{q_{\text{obs}}}{2}$.}
  \label{fig:hypo0p335}
\end{figure}

Pie $x=\hat{x}$ \Hnull p-vērtība ir vienāda ar 0, bet \Halt p-vērtība ir 0,25. Līdz ar to varam noraidīt \Hnull par labu \Halt.


\chapter{Noslēgums}
Esam guvuši pārliecību, ka vilkmes leņķa metode, kas balstīta labā treku rekonstrukcijā, ir jutīga ar krāsām saistītu strūklu identificēšanā. Vilkmes leņķa sadalījumā skaidri izšķirams paugurs, kas centrēts 0~rad ar krāsām saistītu strūklu gadījumā. Vilkmes leņķa sadalījums ir monotons gadījumā, ja strūklas nav saistītas ar krāsām.

Pārliecinosi rezultāti ir gūti arī, pielietojot ``LEP metodi''. Daļiņu blīvums ir lielāks starp strūklām, kas saistītas ar krāsām, nekā bezkrāsu reģionos. 

Rezultāti tika salīdzināti ar krāsu okteta \PW paraugiem, kuros krāsu saistība starp hadroniskajiem sabrukuma produktiem tika noņemta. Šādā gadījumā, pielietojot vilkmes leņķa metodi un ``LEP metodi'', vieglās strūklas atbilst ar krāsām nesaistītām strūklām. 

Veicām vilkmes leņķa atlocīšanas procedūru, lai iegūtu tā patieso distribūciju pirms dektorā veiktās rekonstrukcijas. Būtiskas izmaiņas secinājumos atlocīšanas procedūra neieviesa.

Pamanāms, ka \POWHEG + \PYTHIA MK simulācijas pārspīlē krāsu efektus salīdzinājumā ar detektora reālās pasaules novērojumiem. Šie pārspīlējumi redzami kā izteiktāks paugurs vilkmes leņķa sadalījumā MK simulācijā. \HERWIGpp un atsevišķi \PYTHIA uzskaņojumi krāsu saistību hadronizācijā modelē precīzāk.

Kopumā, sakritība starp datiem un MK nav īpaši laba. $\sim\frac{2}{3}$ \ttbar kombinācija ar $\sim\frac{1}{3}$ $\ttbar\ cflip$ labāk atbilst detektora novērojumiem. Šādi rezultāti tika iegūti, veicot hipotēžu pārbaudi.


\clearpage
\printbibliography[heading=bibintoc]

\end{document}






