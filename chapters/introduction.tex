We look for experimental signatures of colour connection between hadron jets resulting from the decay of a top quark pair. The top quark pair is produced in \Pp\Pp collisions at a centre of momentum energy $\sqrt{s}=13\TeV$. Observations are conducted at the CMS experiment of the CERN LHC. Particular focus is on the light jets resulting from the decay of the \PW boson. They are colour connected and experimentally we could infer about it indirectly. We also study the decay of a hypothetical colour octet \PW boson. In this case the light jets are no longer colour connected and we can use these results to compare the colour connected case. 

We use a method where the pull angle \cite{Gallicchio:2010sw} is observed. This method has been applied at the \DZERO experiment of the Fermilab Tevatron \cite{Abazov:2011vh}, at Run 1 in ATLAS \cite{Aad:2015lxa} and also at Run II in ATLAS \cite{Aaboud:2018ibj}. This method was first applied at CMS by Seidel, M. et al \cite{indico:Markus_cf} but the results have never been published. In comparison to the ATLAS detector, the CMS detector is better equipped to identify jet constituens because of the big and strong 4T field from the large solenoid of CMS. Compared to ATLAS It has a better momentum resolution for tracks in the central region by roughly a factor 2 (ATLAS has a much smaller 2T solenoid + big toroid magnets on the outside \cite{Aad:2008zzm}).

Also used is an adaptation of a methodology used at LEP (hereinafter referred to as the ``LEP methodology'') wherein jet constituents are projected onto inter-jet planes \cite{Abbiendi:2005es}, \cite{Abdallah:2006uq}, \cite{Achard:2003pe}. This method has never been applied at the LHC.

This thesis shows results from a research activity undertaken by the Top Quark group of the CMS experiment. The results at various stages have been presented in the Top Modelling and Generator physics meetings - on 19 January 2016, 29 March 2016, 7 June 2016, 30 August 2016, 13 February 2018 and 17 October 2018.

The results shown in this thesis to this date have not followed the approval procedure of the CMS experiment \cite{twiki:PhysicsApprovals}. Therefore they cannot be regarded as a CMS public result and plots are marked as private work. The CMS approval is envisioned as a subsequent step in this analysis.

When work referenced in this thesis was in full progress we in May, 2018 celebrated the adhesion of Rīgas Tehniskā universitāte to a full membership of the CMS experiment. This work is the first contribution of Latvia to the experimental programme of the CERN LHC.
