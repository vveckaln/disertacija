The LHC operating at 13~TeV centre of mass energy is a factory of top quarks. The cross section of the production of the top quark pair at the LHC is 803~pb. The lifetime of the top quark is $3.3\times10^{-25}$~s and it is so short that unlike other quarks the top quark decays before it hadronises. The top quark decays weakly emitting a \PW boson. In the case of the hadronic decay of the \PW boson, jets of particles are created through the strong nuclear interaction. This process is described by quantum chromodynamics and allows us to model the top quark decay process in terms of colour charge and colour strings. The jets from the hadronic decay of the \PW boson are interacting in the colour field (they are colour-connected). The colour connection leaves distinct experimental signatures that we are able to resolve in the CMS detector, particularly relying on its tracker, 4~T solenoid and calorimeters. Such a study is conducted for the first time at the CMS experiment. The colour connection between jets from the decay of top quark pairs is studied using final states with one lepton, two light jets and two \cPqb-tagged jets. Pull angles and projections of particle directions onto a plane formed by two jets are used. A colour octet \PW toy-model is used to benchmark the performance of the methods.
