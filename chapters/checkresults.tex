%% \section{Event display}
%% A selected event observed in the CMS detector is displayed in Fig. \ref{fig:event_display}, showing the light jets, the \cPqb-tagged jets, the charged lepton and the pull vector in the $\eta - \phi$ plane in a manner analogous to Fig. \ref{fig:pull_angle}.

%% \begin{figure}[hbtp]
%%   \centering
%%   \includegraphics[width=1.0\textwidth]{fig/individual_plots/reco_allconst_total_1111_DeltaR_2p846131_pull_angle_1p964620.png}
%%   \caption{An event observed with the CMS detector. Pull vector (dash-dotted) of the leading light jet forming a pull angle of 1.96 rad with the difference between the second leading light jet and the leading light jet (dashed). Constituents of the leading light jet are marked in blue while the constituents of the second leading light jet ar marked in red. The leading light jet is marked with a solid line while the second leading light jet is marked with a dotted line. The pull vector is enhanced 200 times, while the radius of the circles representing jets is equal to $\frac{p_{T}}{75.0}$ and the radius of the circles representing constituents is equal to $\frac{p^{\text{constituent}}_{T}}{p^{\text{jet}}_{T}}$. The hadronic \cPqb jet and its constituents are marked in green, while the leptonic \cPqb jet and its constituents are marked in magenta. Also shown with X is the charged lepton.}
%%   \label{fig:event_display}
%% \end{figure}
%% \clearpage
%% \section{Pull vector}

%% A set of base tools \textsc{CFAT} \cite{url:cfat} was developed having in mind that the analysis can be implemented both in \RIVET and \CMSSW. Initial tests were done with \RIVET because before the colour octet \PW samples were developed this procedure provided the only means to generate colour-flipped events. Results with \RIVET are shown in Fig. \ref{fig:resultsRivet}. Fig. \ref{fig:pull_angle_allconst_Rivet_leading_jet_2nd_leading_jet_DeltaRTotal_4j2t} shows the distribution of \pullangle between \leadingjet and \scndleadingjet. The central peak which is the experimental signature of colour connect jets is present in the SM results but disappears in the \PW colour octet results. On the other hand the distribution of \pullangle suffers no alterations between \leadingjet and lepton as shown in Fig. \ref{fig:pull_angle_allconst_Rivet_leading_jet_lepton_DeltaRTotal_4j2t}.

\begin{figure}[htp]
\centering
  \def\twidth{0.4}
  \centering
  \subfloat[]{
%    \includegraphics[width=\twidth\textwidth]{fig/pull_angle_allconst_Rivet_leading_jet_2nd_leading_jet_DeltaRTotal_4j2t.png}
    \includegraphics[width=\twidth\textwidth]{example-image-a}
    \label{fig:pull_angle_allconst_Rivet_leading_jet_2nd_leading_jet_DeltaRTotal_4j2t}
  }%
  \subfloat[]{
    \includegraphics[width=\twidth\textwidth]{fig/pull_angle_allconst_Rivet_leading_jet_lepton_DeltaRTotal_4j2t}
    \label{fig:pull_angle_allconst_Rivet_leading_jet_lepton_DeltaRTotal_4j2t}
  }
\caption{}
\label{fig:resultsRivet}
\end{figure}

%% A more comprehensive analysis with data and simulated events at generator and reconstruction level was implemented in \CMSSW version \lstinline[language=sh]|CMSSW_8_0_26_patch1|. The plots are rendered with \ROOT \cite{Brun}. The pull vectors were obtained for all observable jets - the leading light jet \leadingjet (highest \pt), the second leading light jet \scndleadingjet, the leading hadronic $b$ jet \leadingb and the second leading hadronic $b$ jet \scndleadingb. In each case it was diffentiated whether all jet particles or only charged ones should be included in determining the pull jet. The results are separated into $e$ + jets, $\mu$ + jets and combined lepton + jets channels.

%% The $\eta$ dimension of the pull vector with all jet components is given in Fig. \ref{fig:_eta_PV_allconst_reco_leading_jet} - \ref{fig:_eta_PV_allconst_reco_leading_jet}.

%% An explanation of how CMS plots are represented is in order. The top plot in Fig. \ref{fig:_eta_PV_allconst_reco_leading_jet} shows data and Monte Carlo simulations. Unless otherwise specified the Monte Carlo is in reconstruction level. The blue band shows systematics. Given a systematic with index $k$ we identify it as an upside systematic $U^{k}_{i}$ if in bin $i$ the systematic $S^{k}_i$ exceeds the nominal value $N_{i}$. In the opposide case we classify the systematic as a downside systematic $D^{k}_{i}$. The total upside and downside systematic is given as a sum of squares:

%% \begin{align}
%% U_{i}=\sqrt{\sum_{k}\left(U^{k}_{i}-N_{i}\right)^{2}} && D_{i}=\sqrt{\sum_{k}\left(D^{k}_{i}-N_{i}\right)^{2}}.
%% \end{align}

%% The width of the blue band corresponds to the systematical error calculated as $\frac{U_{i}+D_{i}}{2}$. It is centred on $N_{i} + \frac{U_{i}-D_{i}}{2}$. The same applies to the pink band except that the systematics are normalised to the integral of the signal (such normalised histograms are referred to as shapes). The bottom inset shows the ratio of data to Monte Carlo, as well as systematics and systematics fom shapes normalised to Monte Carlo.

%% \figureEML{/reco/PV/charge/allconst/}
%%           {_eta_PV_allconst_reco_leading_jet}
%%           {$\eta$ dimension of the pull vector of \leadingjet with all jet components.}
%% \figureEML{/reco/PV/charge/allconst/}
%%           {_eta_PV_allconst_reco_scnd_leading_jet}
%%           {$\eta$ dimension of the pull vector of \scndleadingjet with all jet components.}
%% \figureEML{/reco/PV/charge/allconst/}
%%           {_eta_PV_allconst_reco_leading_b}
%%           {$\eta$ dimension of the pull vector of \leadingb with all jet components.}
%% \figureEML{/reco/PV/charge/allconst/}
%%           {_eta_PV_allconst_reco_scnd_leading_b}
%%           {$\eta$ dimension of the pull vector of \scndleadingb with all jet components.}

%% The $\phi$ dimension of the pull vector with all jet components is given in Fig. \ref{fig:_phi_PV_allconst_reco_leading_jet} - \ref{fig:_phi_PV_allconst_reco_leading_jet}. 

%% \figureEML{/reco/PV/charge/allconst/}
%%           {_phi_PV_allconst_reco_leading_jet}
%%           {$\phi$ dimension of the pull vector of \leadingjet with all jet components.}
%% \figureEML{/reco/PV/charge/allconst/}
%%           {_phi_PV_allconst_reco_scnd_leading_jet}
%%           {$\phi$ dimension of the pull vector of \scndleadingjet with all jet components.}
%% \figureEML{/reco/PV/charge/allconst/}
%%           {_phi_PV_allconst_reco_leading_b}
%%           {$\phi$ dimension of the pull vector of \leadingb with all jet components.}
%% \figureEML{/reco/PV/charge/allconst/}
%%           {_phi_PV_allconst_reco_scnd_leading_b}
%%           {$\phi$ dimension of the pull vector of \scndleadingb with all jet components.}

%% The magnitude of the pull vector with all jet components is given in Fig. \ref{fig:_mag_PV_allconst_reco_leading_jet} - \ref{fig:_mag_PV_allconst_reco_leading_jet}. The magnitude of the pull vector is usually contained below 0.02 [a.u.].

%% \figureEML{/reco/PV/charge/allconst/}
%%           {_mag_PV_allconst_reco_leading_jet}
%%           {The magnitude of the pull vector of \leadingjet with all jet components.}
%% \figureEML{/reco/PV/charge/allconst/}
%%           {_mag_PV_allconst_reco_scnd_leading_jet}
%%           {The magnitude of the pull vector of \scndleadingjet with all jet components.}
%% \figureEML{/reco/PV/charge/allconst/}
%%           {_mag_PV_allconst_reco_leading_b}
%%           {The magnitude the pull vector of \leadingb with all jet components.}
%% \figureEML{/reco/PV/charge/allconst/}
%%           {_mag_PV_allconst_reco_scnd_leading_b}
%%           {The magnitude of the pull vector of \scndleadingb with all jet components.}

%% \section{Pull angle}

%% The plots of the pull angle between colour connected jets - \leadingjet to \scndleadingjet and \scndleadingjet to \leadingjet with all jet constituents and including all values of $\Delta R$ are shown in Fig. \ref{fig:_pull_angle_allconst_reco_leading_jet_scnd_leading_jet_DeltaRTotal} and Fig. \ref{fig:_pull_angle_allconst_reco_scnd_leading_jet_leading_jet_DeltaRTotal}.

%% \figureEML{/reco/pull_angle/DeltaRTotal/charge/allconst/}
%%           {_pull_angle_allconst_reco_leading_jet_scnd_leading_jet_DeltaRTotal}
%%           {Distribution of the pull angle from \leadingjet to \scndleadingjet for all \DeltaR and including all particles.}

%% \figureEML{/reco/pull_angle/DeltaRTotal/charge/allconst/}
%%           {_pull_angle_allconst_reco_scnd_leading_jet_leading_jet_DeltaRTotal}
%%           {Distribution of the pull angle from \scndleadingjet to \leadingjet for all \DeltaR and including all particles.}

%% Additionally, the plots of the pull angle between jets where we expect no colour connection - \leadingb to \scndleadingb and \scndleadingb to \leadingb with all jet constituents and including all values of $\DeltaR $ are shown in Fig. \ref{fig:_pull_angle_allconst_reco_leading_b_scnd_leading_b_DeltaRTotal} and Fig. \ref{fig:_pull_angle_allconst_reco_scnd_leading_b_leading_b_DeltaRTotal}.

%% \figureEML{/reco/pull_angle/DeltaRTotal/charge/allconst/}
%%           {_pull_angle_allconst_reco_leading_b_scnd_leading_b_DeltaRTotal}
%%           {Distribution of the pull angle from \leadingb to \scndleadingb for all \DeltaR and including all particles.}

%% \figureEML{/reco/pull_angle/DeltaRTotal/charge/allconst/}
%%           {_pull_angle_allconst_reco_scnd_leading_b_leading_b_DeltaRTotal}
%%           {Distribution of the pull angle from \scndleadingb to \leadingb for all \DeltaR and including all particles.}


%% Another chance to look at the distribution of a pull angle between objects that are not colour connected is to choose a jet and a lepton. Fig. \ref{fig:_pull_angle_allconst_reco_leading_jet_lepton_DeltaRTotal} shows the distribution between \leadingjet and the charged lepton. 

%% \figureEML{/reco/pull_angle/DeltaRTotal/charge/allconst/}
%%           {_pull_angle_allconst_reco_leading_jet_lepton_DeltaRTotal}
%%           {Distribution of the pull angle from \leadingjet to the charged lepton for all \DeltaR and including all particles.}

%% As can be readily observed, the central peak in the distribution of the pull angle is prominent in case of colour connected jets and flattens out in the case of objets that are not colour connected.

%% The central peak can reappear in the case of collinearities of the vectors of physics objects even though they are not colour connected. Such a case is seen in the distribution of the pull angle from \leadingjet to hadronic $W$ - Fig. \ref{fig:_pull_angle_allconst_reco_leading_jet_had_w_DeltaRTotal}
%% . 

%% \figureEML{/reco/pull_angle/DeltaRTotal/charge/allconst/}
%%           {_pull_angle_allconst_reco_leading_jet_had_w_DeltaRTotal}
%%           {Distribution of the pull angle from \leadingjet to the hadronic $W$ for all \DeltaR and including all particles.}

%% Another interesting case is choosing the beam. In Fig. \ref{fig:_pull_angle_allconst_reco_leading_jet_beam_DeltaRTotal} we show the distribution of \pullangle from \leadingjet to the positive direction of the beam. We see a peak at a right angle.

%% \figureEML{/reco/pull_angle/DeltaRTotal/charge/allconst/}
%%           {_pull_angle_allconst_reco_leading_jet_beam_DeltaRTotal}
%%           {Distribution of the pull angle from \leadingjet to the positive direction of the beam including all particles.}

%% The QCD samples contribute peaks to the plots because only a few QCD events pass the selection criteria, but they are assigned a large weight. Each event gets effectively assigned a weight

%% \begin{equation}
%% w=\mathcal{L}\cdot\sigma\frac{1}{N_{gen}}.
%% \end{equation}

%% The cross section $\sigma$ for QCD events is very large but the number of generated MC events $N_{gen}$ is very low. Therefore a few QCD events represent an entire distribution.

%% \section{\DeltaR bias}

%% When two jets are close to each other in $\eta-\phi$ space, the jet clustering algorithm is inclined to associate particles of one jet (lowest \pt jet) to another (highest \pt jet). This effect creates a bias in the pull angle analysis as the pull vector is more likely to point to the jet from which the particles were weaned. Figs. \ref{fig:_pull_angle_allconst_reco_leading_jet_scnd_leading_jet_DeltaRle1p0} - \ref{fig:_pull_angle_chconst_reco_leading_jet_scnd_leading_jet_DeltaRgt1p0} illustrates the distribution of pull angle for two cases - closely spaced jets with $\DeltaR\leq1.0$ and well separated jets with $\DeltaR>1.0$.

%% \figureEML{/reco/pull_angle/DeltaRle1p0/charge/allconst/}
%%           {_pull_angle_allconst_reco_leading_jet_scnd_leading_jet_DeltaRle1p0}
%%           {Distribution of the pull angle with \DeltaR$\leq1.0$ and including all jet constituents from \leadingjet to \scndleadingjet.}

%% \figureEML{/reco/pull_angle/DeltaRgt1p0/charge/allconst/}
%%           {_pull_angle_allconst_reco_leading_jet_scnd_leading_jet_DeltaRgt1p0}
%%           {Distribution of the pull angle with \DeltaR$>1.0$ and including all jet constituents from \leadingjet to \scndleadingjet.}

%% \figureEML{/reco/pull_angle/DeltaRle1p0/charge/chconst/}
%%           {_pull_angle_chconst_reco_leading_jet_scnd_leading_jet_DeltaRle1p0}
%%           {Distribution of the pull angle with \DeltaR$\leq1.0$ and only charged jet constituents from \leadingjet to \scndleadingjet.}

%% \figureEML{/reco/pull_angle/DeltaRgt1p0/charge/chconst/}
%%           {_pull_angle_chconst_reco_leading_jet_scnd_leading_jet_DeltaRgt1p0}
%%           {Distribution of the pull angle with \DeltaR$>1.0$ and only charged jet constituents from \leadingjet to \scndleadingjet.}


%% \section{Sensitivity analysis}

%% Sensitivity of the pull angle methodology was studied by applying cuts to the following parameters:

%% 1. The transverse momentum \pt of the hadronic \PW boson. A cut was chosen at 50\GeV and the distribution of the pull angle was obtained at a \pt of the hadronic \PW\ boson greater than and less than or equal to this value. This cut is near the median in the lower half of the distribution of the \pt of the hadronic \PW boson - see Fig. \ref{fig:L_jet_pt_reco_had_w}. The results are shown in Fig. \ref{fig:_pull_angle_hadWPtgt50p0GeV_reco_leading_jet_scnd_leading_jet_DeltaRTotal} - \ref{fig:_pull_angle_hadWPtle50p0GeV_reco_leading_jet_scnd_leading_jet_DeltaRTotal}.

%% 2. Number of jet constituents. A cut was chosen at the number of jet constitutents $N$ being 20 and the distribution of the pull angle was obtained at $N$ greater than and less than or equal to this value. The distribution of the number of jet constituens is given in Fig. \ref{fig:number}. The results are shown in Fig. \ref{fig:_pull_angle_PFNgt20_reco_leading_jet_scnd_leading_jet_DeltaRTotal} - \ref{fig:_pull_angle_PFNle20_reco_leading_jet_scnd_leading_jet_DeltaRTotal}.
                                        
%% 3. The transverse momentum \pt of jet constituents. A cut was chosen at \pt of the jet constituents being 0.5\GeV and the distribution of the pull angle was at obtained at \pt of the jet constituents being greater than and less than or equal to this value. The distribution of the transverse momentum of particles constituting the leading light jet and the leading \cPqb jet are shown in Fig. \ref{fig:L_JetConstPT_allconst_reco}. The results are shown in Fig. \ref{fig:_pull_angle_PFPtgt0p5GeV_reco_leading_jet_scnd_leading_jet_DeltaRTotal} - \ref{fig:_pull_angle_PFPtle0p5GeV_reco_leading_jet_scnd_leading_jet_DeltaRTotal}.

%% 4. Magnitude of the pull vector.  A cut was chosen at magnitude of the pull vector being 0.005[a.u.] and the distribution of the pull angle was obtained at the magnitude the pull vector being greater than and less than or equal tothis value. The distribution of the magnitute of the pull vector is shown in Figs. \ref{fig:_mag_PV_allconst_reco_leading_jet} - \ref{fig:_mag_PV_allconst_reco_leading_jet}. The results are shown in Fig. \ref{fig:_pull_angle_PVMaggt0p005_reco_leading_jet_scnd_leading_jet_DeltaRTotal} - \ref{fig:_pull_angle_PVMagle0p005_reco_leading_jet_scnd_leading_jet_DeltaRTotal}.

%% \begin{figure}
%% \centering
%% \includegraphics[width = 0.45\textwidth]{fig/histos/L/reco/L_jet_pt_reco_had_w.png}
%% \caption{Distribution of the transverse momentum \pt of the hadronic \PW boson.}
%% \label{fig:L_jet_pt_reco_had_w}
%% \end{figure}

%% \begin{figure}[hbtp]
%%     \centering
%%         \def\twidth{0.45}
%%         \subfloat[Leading light jet.]{%
%%     \includegraphics[width=\twidth\textwidth]{fig/histos/L/reco/charge/allconst/L_JetConst_Pt_allconst_reco_leading_jet.png}%
%% }
%%         \subfloat[leading \cPqb jet.]{%
%%     \includegraphics[width=\twidth\textwidth]{fig/histos/L/reco/charge/allconst/L_JetConst_Pt_allconst_reco_leading_b.png}%
%% }
%% \caption{The transverse momentum \pt of particles constituting the leading light jet, and the leading \cPqb jet. In both cases all jet constituents are included.}
%% \label{fig:L_JetConstPT_allconst_reco}

%% \end{figure}

%% \figureEML{/reco/pull_angle/DeltaRTotal/hadronic_W_Pt/hadWPtgt50p0GeV/}
%%      {_pull_angle_hadWPtgt50p0GeV_reco_leading_jet_scnd_leading_jet_DeltaRTotal}
%%      {Distribution of the pull angle for all \DeltaR and all particles from \leadingjet to \scndleadingjet with \pt of \PW\ $>$ 50\GeV.}

%% \figureEML{/reco/pull_angle/DeltaRTotal/hadronic_W_Pt/hadWPtle50p0GeV/}
%%      {_pull_angle_hadWPtle50p0GeV_reco_leading_jet_scnd_leading_jet_DeltaRTotal}
%%      {Distribution of the pull angle for all \DeltaR and all particles from \leadingjet to \scndleadingjet with \pt of \PW $\leq$ 50\GeV.}

%% \figureEML{/reco/pull_angle/DeltaRTotal/PF_number/PFNgt20/}
%%      {_pull_angle_PFNgt20_reco_leading_jet_scnd_leading_jet_DeltaRTotal}
%%      {Distribution of the pull angle for all \DeltaR and all particles from \leadingjet to \scndleadingjet with the number of jet constituents $N>20$.}

%% \figureEML{/reco/pull_angle/DeltaRTotal/PF_number/PFNle20/}
%%      {_pull_angle_PFNle20_reco_leading_jet_scnd_leading_jet_DeltaRTotal}
%%      {Distribution of the pull angle for all \DeltaR and all particles from \leadingjet to \scndleadingjet with the number of jet constituents $N\leq20$.}

%% \figureEML{/reco/pull_angle/DeltaRTotal/PF_Pt/PFPtgt0p5GeV/}
%%      {_pull_angle_PFPtgt0p5GeV_reco_leading_jet_scnd_leading_jet_DeltaRTotal}
%%      {Distribution of the pull angle for all \DeltaR and all particles from \leadingjet to \scndleadingjet with the \pt of jet constituents $>$ 0.5\GeV.}

%% \figureEML{/reco/pull_angle/DeltaRTotal/PF_Pt/PFPtle0p5GeV/}
%%      {_pull_angle_PFPtle0p5GeV_reco_leading_jet_scnd_leading_jet_DeltaRTotal}
%%      {Distribution of the pull angle for all \DeltaR and all particles from \leadingjet to \scndleadingjet with the \pt of jet constituents $\leq$ 0.5\GeV.}

%% \figureEML{/reco/pull_angle/DeltaRTotal/PV_magnitude/PVMaggt0p005}
%%      {_pull_angle_PVMaggt0p005_reco_leading_jet_scnd_leading_jet_DeltaRTotal}
%%      {Distribution of the pull angle for all \DeltaR and all particles from \leadingjet to \scndleadingjet with the magnitude of the pull vector $> $0.005[a.u.].}

%% \figureEML{/reco/pull_angle/DeltaRTotal/PV_magnitude/PVMagle0p005}
%%     {_pull_angle_PVMagle0p005_reco_leading_jet_scnd_leading_jet_DeltaRTotal}
%%     {Distribution of the pull angle for all \DeltaR and all particles from \leadingjet to \scndleadingjet with the magnitude of the pull vector $\leq $0.005[a.u.].}

%% From a simple qualitative observation we conclude that the pull angle methodology is sensitive to \pt of the hadronic \PW boson, number of jet constituents, \pt of jet constituents but not particularly sensitive to the magnitude of the pull vector.



%\subsection{Unfolding}
%\label{sec:unfolding}
%When a detector makes an observation the end results suffers from the inefficiencies of the detector. Unfolding is a method where the observation made at the detector is corrected for detector effects. Hence we can obtain an estimate of the true distribution of the observable. However, it comes at a cost of a signficant loss of granularity of the phase-space of the observable.

We infer about the detector effects because in Monte Carlo samples each generated event is reconstructed. Therefore an observable in bin $i$ at generation level migrates to bin $k$ at reconstruction level. By accumulating a large number of events we obtain statistics of migration. In unfolding we revert the migration - given an observable at bin $k$ we assign probabilities to the true values of the observable.

Values of \pullangle at generation level that do not have a corresponding value at reconstruction level are put in the underflow bin at reconstruction level. Values of \pullangle at reconstruction level that do not have a corresponding value at generation level are put in the underflow bin at generation level. The underflow bins at generation level are treated as background and are removed. Distributions that are not filled at generation level - data and MC backgrounds are reduced by a corresponding scale factor. The underflow bin at reconstruction level is used to constrain the underflow bin for the unfolded result.

Unfolding is performed on data from which the MC backgrounds have been subtracted. We also performed the unfolding procedure in reverse obtaining the folded back output.

We are interested to have the migration matrix as diagonal as possible to reduce statistical uncertainties on the unfolding result. Two measures are used to characterise the share of statistics on the diagonal of the migration matrix - stability and purity. Stability is the ratio of the contents of the diagonal element to the total number of events at reconstruction level in the bin:

\begin{equation}
  \text{stability}\equiv\frac{\theta^{\text{diag}}_{\text{input}}}{\Sigma_{x=1}^{x=N_{x}}\theta^{x}_{\text{input}}},
\end{equation}

\noindent where $x$ is the bin index at reconstruction level, starting the numbering from 1 and $N_{x}$ is the number of bins at reconstruction level. Purity is the ratio of the contents of the diagonal element to the total number of events at generation level in the bin:

\begin{equation}
  \text{purity}\equiv\frac{\theta^{\text{diag}}_{\text{input}}}{\Sigma_{y=1}^{y=N_{y}}\theta^{y}_{\text{input}}},
\end{equation}

\noindent where we have used $y$ as the bin index at generation level. The values of purity and stability are recommended to exceed 50~\% at each bin.

An interesting measure is the amount by which the unfolded result is different from the generated result at MC (an ideal result would be 0), normalised to statistical uncertainty of the unfolded result. This measure is called the pull. A mathematical expression for the pull is

\begin{equation}
  \text{pull}\equiv\frac{\theta^{\text{gen}}_{\text{unf}}-\theta^{\text{gen}}_{\text{in}}}{\sigma^{\text{gen}}_{\text{unf}}}.
\end{equation}

We generate random toy distributions of the observable at generation level, thus obtaining a distribution of the pull.

The number of bins at generation level is reduced by a factor of 2 with regard to the number of bins at reconstruction level in order that unfolding be computationally feasible.

The class \lstinline[language=sh]|TUnfoldDensity|~\cite{Schmitt:2012kp} of \ROOT is used to do the unfolding procedure. The binning scheme is managed with class \lstinline[language=sh]|TUnfoldBinning|. No regularisation is applied. The unfolding results of \pullangle from \leadingjet to \scndleadingjet including all jet constituents are shown in Fig.~\ref{fig:unfolding_nominal_leading_jet_allconst_pull_angle_ORIG_MC13TeV_TTJets_ORIG}. In order to create the plots shown herein a new class \lstinline[language=sh]|CompoundHistoUnfolding| \cite{url:compoundhistounfolding} was developed which was added to \ROOT complete with input and output streamers.

The unfolding results are shown in Fig.~\ref{fig:unfolding_nominal_leading_jet_allconst_pull_angle_ORIG_MC13TeV_TTJets_ORIG}. Distributions corresponding to unfolding results with migration matrices from $\ttbar\ Herwig++$ and $\ttbar\ cflip$ as well as systematics $\ttbar\ fsr\ dn$ and $\ttbar\ fsr\ up$ (see Chap. \ref{chap:systematic_uncertainties}) are laid over the unfolding plots. In the unfolded distibution there are very large uncertainties and poor purity and stability in most of the bins. In order to mitigate these effects a bin optimisation algorithm was tried. The algorithm proceeds as follows:

\begin{itemize}
\item In each bin of the reconstructed observable, the particle-level distribution is fitted with a gaussian distribution.
\item Starting from the low edge of the reconstructed distribution, a bin $i$ is searched that fulfils $\mu_{i}-f\sigma_{i}/2 > 0$, $\mu_{i}$ being the mean and $\sigma_{i}$ the standard distribution from the fit at generation level in each bin $i$ at reconstruction level. The factor $f$ is chosen so that $f\sigma\sim\frac{\theta_{p,\max}-\theta_{p, \min}}{3}$, in order to obtain 4-3 optimised bins. In practice, this factor has to be chosen very small - 0.15 for pull angle and 0.3 for the magnitude of the pull vector, indicating that $\sigma$ is large compared to the range of the phase space of the pull angle.
\item The new optimised bin is then defined from 0 to $\text{mean}+f\text{sigma}_{i}/2$. 
\item The algorithm is iterated until the edge of the histogram is reached.
\item The obtained binning is used to present the result on particle (i.e. generation) level. To obtain the final migration matrix used in the unfolding each bin at particle level is split by two to obtain a suitable reconstruction level binning.
\end{itemize}

This algorithm is depicted in Fig.~\ref{fig:gaussiancurves}. The parameter $b$ is given by $\frac{\sigma_{n}}{\sigma_{n} + \sigma_{n+1}}$.

\begin{figure}
  \centering
  \includegraphics[width = 0.8\textwidth]{fig/gaussiancurves}
  \caption{Method of optimising the binning scheme for unfolding.}
  \label{fig:gaussiancurves}
\end{figure}

The unfolded result with the optimised binning is shown in Fig.~\ref{fig:unfolding_nominal_leading_jet_allconst_pull_angle_OPT_MC13TeV_TTJets_SIGMA_0p15}. The purity and stability in the central bin is still poor. Therefore a scheme using 3 regular bins as in the ATLAS analysis~\cite{ATLAS:2017iaz} was tried.

The results with the regular binning scheme are shown in Fig.~\ref{fig:unfolding_nominal_leading_jet_allconst_pull_angle_OPT_MC13TeV_TTJets_ATLAS3}. The stability and purity levels with this binning scheme reach acceptable levels at each bin and it was adopted for further analysis.

The unfolding results using the migration matrix from the sample $\ttbar\ cflip$ are shown in Fig.~\ref{fig:unfolding_cflip_leading_jet_allconst_pull_angle_OPT_MC13TeV_TTJets_cflip_ATLAS3}. The \ttbar cflip is included as a systematic for \ttbar.

The unfolding results of the \pullangle from \leadingb to \scndleadingb with all jet constituents are shown in Fig.~\ref{fig:unfolding_nominal_leading_b_allconst_pull_angle_OPT_MC13TeV_TTJets_ATLAS3}.

As an additional observable the magnitude of the pull vector \pvmag was unfolded. Fig.~\ref{fig:unfolding_nominal_leading_jet_allconst_pvmag_OPT_MC13TeV_TTJets_ATLAS3} shows the unfolding results of \pvmag from \leadingjet to \scndleadingjet including all jet constituents.

The bin-per-bin significance (\%) of nuisances in the total systematical error in the unfolded result are given in Table~\ref{tab:unc_table_fullpull_angle_OPT_allconst_gen_out_MC13TeV_TTJets_nominal_ATLAS3}. Nuisances that directly affect the hadronisation $\ttbar\ Herwig++$, $\ttbar\ QCDbased$ and $\ttbar\ ERDon$ are the most significant.

In addition to the \POWHEG+\PYTHIA 8 sample, we also investingate a \POWHEG+\PYTHIA 8 * sample in which $\ttbar\ cflip$ has been added as a systematic to \ttbar. Table~\ref{tab:unc_table_fullpull_angle_OPT_allconst_gen_out_MC13TeV_TTJets_cflip_ATLAS3} shows the aditional bin-per-bin $\ttbar cflip$ uncertainty for the \POWHEG+\PYTHIA 8 * sample.

The agreement between the unfolded result and MC prediction at generation level is quantified using a goodness-of-fit method. Given the normalised unfolded detector observation $D$, the normalised MC prediction $M$, the full covariance matrix $\Sigma$ of normalised experimental uncertainties, the $\chi^{2}$ is calculated as follows:

\begin{equation}
  \chi^{2}=(D^{T}-M^{T})\cdot\Sigma^{-1}\cdot(D-M).
  \label{eq:chi2}
\end{equation}

From the $\chi^{2}$ value the p-value can be computed using the number of degrees of freedom equal to the number of bins in the unfolded distribution subtracted by 1 to account for a loss of freedom when normalising the distributions. One row and one column is discarded from the covariance matrix $\Sigma$. $\chi^{2}$ value does not depend on the choice of the discarded elements.

Table~\ref{tab:chi_table_pull_angle_OPT_allconst_nominal_ATLAS3} shows the $\chi^{2}$ values and p-values for \pullangle using all jet constituents. The results show that the pull angle distribution is poorly modelled by the MC generators. In general, the simulation predicts a more sloped distribution, i.e. a stronger colour flow effect. \HERWIGpp models better the pull angle distribution than \PYTHIA 8.2. Accuracy of \PYTHIA 8.2 is particularly poor when predicting the distribution of \pullangle from \scndleadingjet to \leadingjet.

The $\chi^{2}$ values and p-values for the \PW colour octet model are given in Table~\ref{tab:chi_table_pull_angle_OPT_allconst_cflip_ATLAS3}. In the colour flip model the distribution of \pullangle from \leadingjet to \scndleadingjet is modelled less acurately than the SM prediction.
  
Table~\ref{tab:chi_table_pull_angle_OPT_allconst_MC13TeV_TTJets_nominal_ATLAS3_full} shows the values of $\chi^{2}$ and if signal $M$ in Eq.~\ref{eq:chi2} is replaced by the respective systematic, but leaving the covariance matrix $\Sigma$ unchanged. The agreement is better than \ttbar when the colour flow is modelled with the $\ttbar\ ERDOn$, $\ttbar\ Herwig++$ and $\ttbar\ QCDbased$ setup.

\figunfolding{nominal}{leading_jet}{allconst}{pull_angle}{ORIG}{MC13TeV_TTJets}

\figunfolding{nominal}{leading_jet}{allconst}{pull_angle}{SIGMA_0p15}{MC13TeV_TTJets}

\figunfolding{nominal}{leading_jet}{allconst}{pull_angle}{ATLAS3}{MC13TeV_TTJets}

\figunfolding{cflip}{leading_jet}{allconst}{pull_angle}{ATLAS3}{MC13TeV_TTJets_cflip}

\figunfolding{nominal}{leading_b}{allconst}{pull_angle}{ATLAS3}{MC13TeV_TTJets}

\figunfolding{nominal}{leading_jet}{allconst}{pvmag}{ATLAS3}{MC13TeV_TTJets}

\input{tables/unc_nominal_full/pull_angle/ATLAS3/unc_table_full_leading_jet_allconst_pull_angle_OPT_gen_out_MC13TeV_TTJets.txt}

\input{tables/unc_cflip_full/pull_angle/ATLAS3/unc_table_full_leading_jet_allconst_pull_angle_OPT_gen_out_MC13TeV_TTJets.txt}

\input{tables/chi_nominal/pull_angle/ATLAS3/chi_table_pull_angle_OPT_allconst.txt}

\input{tables/chi_cflip/pull_angle/ATLAS3/chi_table_pull_angle_OPT_allconst.txt}

\input{tables/chi_nominal/pvmag/ATLAS3/chi_table_pvmag_OPT_allconst.txt}

\input{tables/chi_cflip/pvmag/ATLAS3/chi_table_pvmag_OPT_allconst.txt}

\input{tables/chi_full_nominal/pull_angle/ATLAS3/chi_table_pull_angle_OPT_allconst_MC13TeV_TTJets_full.txt}


%% \subsection{Colour octet \PW\ boson}

%% Fig. \ref{fig:_pull_angle_allconst_reco_leading_jet_scnd_leading_jet_DeltaRTotal} - \ref{fig:_pull_angle_chconst_reco_leading_jet_lepton_DeltaRTotal} shows the results of the colour octet \PW\ boson (in blue) compared to the standard model (in red). The pull angle distribution between the light jets has become flat in the colour octet \PW\ boson model.

%% \figureEML[_merged]{/reco/pull_angle/DeltaRTotal/charge/allconst/}
%%           {_pull_angle_allconst_reco_leading_jet_scnd_leading_jet_DeltaRTotal}
%%           {Pull angle distribution the \ttbar sample for all \DeltaR and all particles between the leading jet and the second leading jet at reconstruction level. SM - red, colour octet \PW\ model - blue.}

%% \figureEML[_merged]{/reco/pull_angle/DeltaRTotal/charge/allconst/}
%%           {_pull_angle_allconst_reco_leading_jet_lepton_DeltaRTotal}
%%           {Pull angle distribution for the \ttbar sample for all \DeltaR and all particles between the leading jet and electron at reconstruction level. SM - red, colour octet \PW\ model - blue.}

%% \figureEML[_merged]{/reco/pull_angle/DeltaRTotal/charge/chconst/}
%%           {_pull_angle_chconst_reco_leading_jet_scnd_leading_jet_DeltaRTotal}
%%           {Pull angle distribution for the \ttbar sample for all \DeltaR and all particles between the leading jet and the second leading jet at reconstruction level. SM - red, colour octet \PW\ model - blue.}

%% \figureEML[_merged]{/reco/pull_angle/DeltaRTotal/charge/chconst/}
%%           {_pull_angle_chconst_reco_leading_jet_lepton_DeltaRTotal}
%%           {Pull angle distribution for the \ttbar sample for all \DeltaR and all particles between the leading jet and the lepton at reconstruction level. SM - red, colour octet \PW\ model - blue.}

%\subsection{Results of the LEP methodology}
%\label{sec:LEP_methodology}
In order to correctly apply the LEP methodology one needs to separate the \cPqb\ quarks on the hadronic and the leptonic branch. The methodology to achieve this goal was described in \ref{chap:methodology}. As a test of the validity of the methodology, one can use the invariant mass of the \cPqt quark, formed by the sum of the \cPqb quark and the \PW boson. Fig. \ref{fig:L_jet_mass_reco} shows the resonance of the \cPqt quark. Tables \ref{tab:mass_L_reco_MC} - \ref{tab:mass_L_reco_data} provide measurements of the masses of the \cPqt quark and the \PW boson on the hadronic and leptonic branches using a polynomial fit.

  \begin{figure}[hbtp]
    \def\twidth{0.5}
    \subfloat[Observed mass of \PW\ on the hadronic branch.]{
    \includegraphics[width=\twidth\textwidth]{fig/histos/L/reco/L_jet_mass_reco_had_w.png}
    \label{fig:L_jet_mass_reco_had_w}
}
  \subfloat[Observed mass of \cPqt\ on the hadronic branch.]{
    \includegraphics[width=\twidth\textwidth]{fig/histos/L/reco/L_jet_mass_reco_had_t.png}
    \label{fig:L_jet_mass_reco_had_t}
}

 \subfloat[Observed mass of \PW\ on the leptonic branch.]{
    \includegraphics[width=\twidth\textwidth]{fig/histos/L/reco/L_jet_mass_reco_lept_w.png}
    \label{fig:L_jet_mass_reco_lept_w}
  }
 \subfloat[Observed mass of \cPqt\ on the leptonic branch.]{
    \includegraphics[width=\twidth\textwidth]{fig/histos/L/reco/L_jet_mass_reco_lept_t.png}
    \label{fig:L_jet_mass_reco_lept_t}
}
 \caption{Observed masses of objects.}
  \label{fig:L_jet_mass_reco}
\end{figure}


\input{chapters/results/masses_tables/mass_L_reco_MC.txt}

\input{chapters/results/masses_tables/mass_L_reco_data.txt}

\input{chapters/results/masses_tables/mass_L_reco_MC_flip.txt}


Three types of flows are analysed:
\begin{itemize}
\item in particle flow all particles are assigned a weight equal to 1.0.
\item in energy flow particles are assigned a weight proportional to their energy normalised to the sum of the energy of the top quarks.
\item in \pt flow particles are assigned a weight proportional to their transverse momentum normalised to the transverse momentum of the respective jet.
\end{itemize}

The results of the LEP methodology using particle flow are shown in Fig. \ref{fig:chi_allconst_N} with all jet constituents and in Fig. \ref{fig:chi_chconst_N} including only charged jet constituents. The flow is plotted between the leading \cPqb jet \leadingb and the 2nd leading b jet \scndleadingb, the hadronic \cPqb jet $j_{h}^{\cPqb}$ and the furthest light quark $j_{f}^{\PW}$ (jet distance is measured with the angle between the spatial components of the 4-vectors of the jets), the closest light quark $j_{c}^{\PW}$ and the hadronic \cPqb jet $j_{h}^{\cPqb}$, and the leading light jet \leadingjet and the second leading light jet \scndleadingjet.

The results using energy flow are shown in Fig. \ref{fig:chi_allconst_E} with all jet constituents and in Fig. \ref{fig:chi_chconst_E} including only charged jet constituents.

The results o using \pt flow are shown in Fig. \ref{fig:chi_allconst_Pt} with all jet constituents and in Fig. \ref{fig:chi_chconst_Pt} including only charged jet constituents.

In all cases the density drops in the middle area between jets compared to the jet centre with the central density varying between colour connect jets and jets not connected in colour.

The bin-per-bin ratios of the flow in colour-free regions (\leadingb, \scndleadingb), ($j_{h}^{\cPqb}$, $j_{f}^{\PW}$), ($j_{c}^{\PW}$, $j_{h}^{\cPqb}$) to the flow in the colour-connected region (\leadingjet, \scndleadingjet) are given in Fig. \ref{fig:chirg_allconst_N} including all jet constituents and Fig. \ref{fig:chirg_chconst_N} including only charged jet constituents. Significant colour reconnection is noticeable in the region ($j_{c}^{\PW}$, $j_{h}^{\cPqb}$) assuming the colour octet \PW model.

Fig. \ref{fig:ratio_qlq2l} shows the bin-per-bin ratio of the particle flow in the region (\leadingjet, \scndleadingjet) in the SM model to the particle flow in the region (\leadingjet, \scndleadingjet) in the colour octet \PW model. A loss of colour connection in this region in the colour octet \PW model is evident.

As a quantitative result from the LEP methodology one can use the parameter $R$ which is defined as the ratio between the integral from 0.2 to 0.8 in the colour connected region to the integral from 0.2 to 0.8 in the region not connected in colour:

\begin{equation}
R=\frac{\int_{0.2}^{0.8}f^{\text{inter \PW region}}d\chi}{\int_{0.2}^{0.8}f^{\text{intra \PW region}}d\chi},
\end{equation}

where $f(\chi)$ is the density of the flow distribution.

This parameter was used at LEP to quantify colour connection effects and their values from different experiment corresponding to 625\pbinv of data in the range \sqrts=189-209\GeV are given in Table \ref{tab:LEP_R}. We note inconsistency in the $R$ values reported by different experiments. Furthermore,  $R$ should exceed 1 on theoretical basis. The range 0.2 - 0.8 is identified as sensitive to colour-connection effects. 
\begin{table}
\centering
\begin{tabular}{lll}
LEP experiment & $R$ value - data                                        & reference\\
\hline
    OPAL       & 1.243                                                   & \cite{Abbiendi:2005es}\\
    Delphi     & 0.889 ($\sqrt{s}=183$ GeV) - 1.039 ($\sqrt{s}=207$ GeV) & \cite{Abdallah:2006uq}\\
    L3         & 0.911                                                   & \cite{Achard:2003pe}\\
  \end{tabular}
\caption{R values observed at LEP}
\label{tab:LEP_R}
\end{table} 

In our case we use 3 $R$ values for any of the regions not connected in colour with normalisation to the colour connected region (\leadingjet, \scndleadingjet).

The integral of particle flow from 0.2 to 0.8 in different regions and the inverse of $R$ values for the SM model is given Table \ref{tab:R_L_reco_N_MC_SM}, for data in Table \ref{tab:R_L_reco_N_data_SM} and for the \PW colour octet model in Table \ref{tab:R_L_reco_N_MC_cflip}.

The integral of energy flow from 0.2 to 0.8 in different regions and the inverse of $R$ values for the SM model is given Table \ref{tab:R_L_reco_E_MC_SM}, for data in Table \ref{tab:R_L_reco_E_data_SM} and for the \PW colour octet model in Table \ref{tab:R_L_reco_E_MC_cflip}.

The integral of \pt flow from 0.2 to 0.8 in different regions and the inverse of $R$ values for the SM model is given Table \ref{tab:R_L_reco_Pt_MC_SM}, for data in Table \ref{tab:R_L_reco_Pt_data_SM} and for the \PW colour octet model in Table \ref{tab:R_L_reco_Pt_MC_cflip}.


\figureChi{allconst}{N}

\figureChi{chconst}{N}

\figureChi{allconst}{E}

\figureChi{chconst}{E}

\figureChi{allconst}{Pt}

\figureChi{chconst}{Pt}

\figureratiographs{allconst}{N}

\figureratiographs{chconst}{N}

\begin{figure}[htpb]
\def\twidth{0.45}
\centering
\subfloat[Result using all jet constituents.]{
\includegraphics[width=\twidth\textwidth]{fig/ratiographs_merged_SM/L_qlq2l_N_allconst_reco.png}
\label{fig:L_qlq2l_N_allconst_reco}
}%
\subfloat[Result using only charged jet constituents.]{
\includegraphics[width=\twidth\textwidth]{fig/ratiographs_merged_SM/L_qlq2l_N_chconst_reco.png}
\label{fig:L_qlq2l_N_chconst_reco}
}
\caption{Bin-per-bin ratio of particle flow in region (\leadingjet, \scndleadingjet) in the SM model to particle flow in region (\leadingjet, \scndleadingjet) in the \PW colour octet model.}
\label{fig:ratio_qlq2l}
\end{figure}

\clearpage

\input{chapters/results/Rvalues_SM/R_L_reco_MC_N_SM.txt}

\input{chapters/results/Rvalues_SM/R_L_reco_data_N_SM.txt}

\input{chapters/results/Rvalues_cflip/R_L_reco_MC_N_cflip.txt}

\input{chapters/results/Rvalues_SM/R_L_reco_MC_E_SM.txt}

\input{chapters/results/Rvalues_SM/R_L_reco_data_E_SM.txt}

\input{chapters/results/Rvalues_cflip/R_L_reco_MC_E_cflip.txt}

\input{chapters/results/Rvalues_SM/R_L_reco_MC_Pt_SM.txt}

\input{chapters/results/Rvalues_SM/R_L_reco_data_Pt_SM.txt}

\input{chapters/results/Rvalues_cflip/R_L_reco_MC_Pt_cflip.txt}


