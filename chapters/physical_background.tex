The top quark is a third generation quark with charge $\frac{2}{3}e$, where $e$ is the elementary electrical charge - the magnitude of charge carried by the electron. Its place in the Standard Model is shown in Fig.~\ref{fig:top_quarkSM}. The Standard Model is the most widely accepted model to describe subatomic physics with experimental verification to spectacular agreement.

  \begin{figure}[hbtp]

    \centering
    \includegraphics[width=0.8\textwidth]{fig/tquarkSM/tquarkSM.pdf}
    \caption{The top quark in the Standard Model.}
    \label{fig:top_quarkSM}
    
  \end{figure}

The existence of the top quark and also its counterpart - the bottom quark was predicted by Kobayashi and Maskawa in 1973 to explain the CP-violations in the decay of the kaon~\cite{Kobayashi:1973fv}. The discovery of the top quark was announced in 1995 by two experiments at the Fermilab Tevatron - CDF~\cite{Abe:1995hr} and \DZERO~\cite{D0:1995jca} based on observations at \sqrts=1.8~\TeV and integrated luminosity $\sim$50-67~\fbinv.

A worldwide combination of integrated luminosity from ATLAS, CMS, CDF and \DZERO gives the measurement of the top quark mass of 173.34~$\pm$~(stat)~0.27~$\pm$~0.71~(syst)~\GeV. The top quark is the heaviest of all known particles. It is heavier than the Higgs boson (126~\GeV) and much heavier than the bottom quark whose mass is estimated around $4.2$~\GeV~\cite{Hoang:1999ye}.

The cross section of top quark pair production in proton-proton collisions at \sqrts=13~\TeV is measured to be 803~pb~\cite{Sirunyan:2018goh}. The cross section increases once the centre of mass energy is increased as illustrated in Fig.~\ref{fig:tt_curve_toplhcwg_sep18}. At lower energies \Pp\Pap colliders are better than \Pp\Pp colliders at producing the top quark pair.

The top quark is assumed to have occured naturally as a constituent of the Quark-Gluon Plasma in the first picosecond after the Big Bang~\cite{Husdal:2016haj}. The stars are too cold to produce the top quark. The top quark would be produced in an environment where $k_{B}T>m_{t}$. Thus it is likely that the colliders on Earth are the only places in the Universe where the top quark is synthetised.

\begin{figure}[hbtp]
  \centering
  \includegraphics[width=0.6\textwidth]{fig/tt_curve_toplhcwg_sep18.pdf}
  \caption{Inclusive cross section of the top pair at different centre of mass energies~\cite{twiki:tt_curve_toplhcwg_sep18}. The result shows the cross section at \Pp\Pp and \Pp\Pap collisions. The results from CMS and ATLAS are given for different channels of the decay of the \ttbar pair.}
  \label{fig:tt_curve_toplhcwg_sep18}
\end{figure}

Using the relationship

\begin{equation}
N=\sigma\int L(t)dt
\end{equation}

\noindent where $N$ is the number of \ttbar pairs, $\sigma$ - the \ttbar cross section, $L$ - the instantaneous luminosity, at 35.9~\fbinv integrated luminosity $26.7\times10^{6}$ of such pairs are expected to be created. 

In the LHC 2 protons collide with an energy large enough to ``squeeze`` the protons so closely together that the quarks in one proton are able to interact with the quarks in the other proton. They interact by exchanging a gluon. By such an exchange the top quark-antiquark pair can be created. Fig.~\ref{fig:top_quark_productions} illustrates 2 such scenarios. The gluon exchanged is so energetic as to smash the proton into debris. Such a collision is called inelastic.

\begin{figure}[h!]
  \centering
  \def\twidth{0.45}
  \subfloat[Pair creation.]{%
    \includegraphics[width=\twidth\textwidth]{fig/top_quark_pair_prod_gfusion}%
    \label{fig:top_quark_production}
  }\hfil
  \subfloat[Gluon fusion.]{%
    \includegraphics[width=\twidth\textwidth]{fig/top_quark_pair_prod_gluon}%
    \label{fig:top_quark_production2}
  }
  %% \subfloat[Heavy quark fusion. The dotted line can be any uncharged boson.]{
  %%   \includegraphics[width=\twidth\textwidth]{fig/top_quark_pair_prod_qqbar}
  %%   \label{fig:top_quark_production3}
  %% }
  \caption{Top quark pair production in a \Pp\Pp collision.}
  \label{fig:top_quark_productions}
\end{figure}

The top quark decays exclusively in the weak decay process (Fig.~\ref{fig:quark_decay}). In the weak decay \PW boson and a quark of different flavour and magnitude of electrical charge $\frac{1}{3}e$ is emitted. 

\begin{figure}[H]
  \centering
  \includegraphics[width=0.3\textwidth]{fig/fig_top_quark_decay.pdf}
  \caption{Weak decay of the top quark \cPqt. $p'$ is a quark of different flavour and $k$ and $k'$ are fermions resulting from the decay of the \PW boson.}
  \label{fig:quark_decay}
\end{figure}

The average of measurements by CDF, \DZERO experiments of Tevatron~\cite{Aaltonen:2015cra}, and ATLAS and CMS experiments of LHC~\cite{twiki:tt_curve_toplhcwg_sep18} yield the result of the $|V_{tb}|$ term of the Cabbibo-Kobayashi-Maskawa matrix

\begin{equation}
  |V_{tb}|=1.009\pm0.031.
\end{equation}

This implies the top quark mixes with (decays by emitting) the \cPqb quark in at least $(0.98)^{2}$ of the cases. The other elements of the CKM matrix are very small~\cite{Patrignani:2016xqp}:

\begin{align}
  & |V_{td}|=8.4\times10^{-3}, && |V_{ts}|=40.0\times10^{-3}.
\end{align}

The width of the top quark as measured by the \DZERO collaboration~\cite{Abazov:2010tm} with 2.3~\fbinv of integrated luminosity is $\Gamma=1.99^{+0.69}_{-0.55}$~\GeV. This translates into a lifetime of $\tau_{t}=3.3\times10^{-25}~\text{s}$.

This lifetime is smaller than the hadronisation timescale ($1/\Lambda\sim10^{-24}~\text{s}$), where $\Lambda^{2}$ is the value of $Q^{2}$ of the exhanged gluon at which the strong coupling constant $\alpha_{s}$ becomes $\sim$1, close to its asymptotic value at the confinement barrier. Thus the top quark decays before it hadronises and the experimentalist has a unique opportunity to observe a ``bare'' quark for a very short time.

The lifetime of the top quark is also smaller than the spin decorrelation of the top quark pair $M/{\Lambda^{2}}=3\times10^{-21}~\text{s}$. This means that the top quark pair maintain their spin states before they decay and transfer the spin states to their decay products ~\cite{Cristinziani:2016vif}.

The branching ratios of the decay of the top quark are essentially those of the decay of the \PW boson. The \PW boson can decay to any of the pair of leptons (\Pe\Pgne, \Pgm\Pgngm, \Pgt\Pgngt) or the pairs of \cPqu, \cPqd' and \cPqc, \cPqs' quarks (where the apostrophe means flavour symmetry is not exactly conserved). However, the quark pairs can have 3 colours. Thus the total number of states is $3+2\times3=9$. A simple estimate and experimentally observed branching ratios from the decay of the \PW boson are given in Table~\ref{tab:W_br}.

\begin{table}[h!]
  \centering
  \caption{Branching ratios from the decay of the \PW boson.}
  \label{tab:W_br}
  \begin{tabular}{l r r}
    Mode                  & $\Gamma_{j}/\Gamma$ & $\Gamma_{j}/\Gamma$\\
                          & simplified          & observed \cite{Patrignani:2016xqp}\\
    \hline
    $e\nu_{e}$            & $\frac{1}{9}$       & (10.71 $\pm$ 0.16) \%\\
    $\mu\nu_{\mu}$        & $\frac{1}{9}$       & (10.63 $\pm$ 0.15) \%\\
    $\tau\nu_{\tau}$      & $\frac{1}{9}$       & (11.38 $\pm$ 0.21) \%\\
    pair of quarks        & $\frac{2}{3}$       & (67.41 $\pm$ 0.27) \%
  \end{tabular}
\end{table}


Colour connected jets are emitted in the hadronic decay of the \PW boson (Fig.~\ref{fig:ttbar_cf}). The quarks originating these jets have opposite momenta in their COM frame. As the quarks try to separate, their kinetic energy is transfered to the colour field. The extra energy in the colour field equal to about $m_{\PW}$ (80.4~\GeV) is expended to create new particles. A simplified portrayal of the birth of new hadrons is given in Fig.~\ref{fig:combination}, which is based on the Lund model~\cite{Andersson:1983ia}. An alternative portrayal based on Feynman diagrams is given in Fig.~\ref{fig:colour_field}.

\begin{figure}[htp]
\centering
\includegraphics[width=0.8\textwidth]{fig/combination.pdf}
\caption{Creation of new hadrons by two energetic quarks.}
\label{fig:combination}
\end{figure}

The following species of particles are created in the case of a hadronic decay of the \PW boson:

\begin{table}[h!]
\caption{New particles created in the colour field between energetic colour connected quarks originating from a hadronic decay of the \PW boson.}
\label{tab:particles}
\centering
\begin{tabular}{ l l l l }
\textbf{Particle}  & \textbf{Mass}~[GeV]  & \textbf{Lifetime}~[s] & \textbf{Observable signal}\\
\Pgpz              & 135.0               & $8.5\times10^{-27}$  & 2 \cPgg absorbed at ECAL\\
\Pgppm             & 139.6               & $2.6\times10^{-8}$   & tracker, ECAL, HCAL showers\\
\PKzS              & 497.6               & $8.95\times10^{-11}$ & ECAL, HCAL showers\\
\PKzL              & 497.6               & $5.1\times10^{-8}$   & ECAL, HCAL showers\\
\PKpm              & 493.7               & $1.2\times10^{-8}$   & tracker, ECAL, HCAL showers\\
\Pn                & 939.6               & $881.5$              & ECAL, HCAL showers\\
\Pp                & 938.3               & $\infty$             & tracker, ECAL, HCAL showers\\
\end{tabular}

\end{table}

\begin{figure}[hbtp]

\centering
\includegraphics[width=0.4\textwidth]{fig/ttbar_cf_cropped.pdf}
\caption{Colour flow in the decay of a top quark pair.}
\label{fig:ttbar_cf}

\end{figure}

\begin{figure}[hbtp]
\centering
\includegraphics[width=1.0\textwidth]{fig/colour_field_full.pdf}
\caption{Creation of hadrons in the colour field of two quarks.}
\label{fig:colour_field}
\end{figure}

The respective resonances are clearly discernible at the generation level (Fig.~\ref{fig:mass_resonances}). Only the neutral pion decays before being directly observed in the detector.

\begin{figure}[htbp]
\centering
\def\twidth{0.45}
\subfloat[Leading light jet.]{%
\includegraphics[width=\twidth\textwidth]{fig/histos/L/gen/charge/allconst/L_JetConst_M_allconst_gen_leading_jet.png}%
}
\subfloat[Leading \cPqb jet.]{%
\includegraphics[width=\twidth\textwidth]{fig/histos/L/gen/charge/allconst/L_JetConst_M_allconst_gen_leading_b.png}%
}
\caption{Resonances corresponding to particles listed in Table~\protect\ref{tab:particles} constituting the leading light jet and the leading \cPqb jet. In both cases all jet constituents are included.}
\label{fig:mass_resonances}
\end{figure}

The distribution of the number of particles that constitute the leading light jet and the leading \cPqb jet is shown in Fig.~\ref{fig:number}, the distribution of the ratio of the number of electrically charged particles to the total number of particles is given in Fig.~\ref{fig:charged_contentN}, and the distribution of the ratio of the energy of electrically charged particles to the total energy of particles is given in Fig.~\ref{fig:charged_contentE}. Leading light jet is the jet from the decay of the \PW boson that has the highest transverse momentum \pt while the leading \cPqb jet is either of the \cPqb jets that has the highest transverse momentum.

\begin{figure}[hbtp]
    \centering
        \def\twidth{0.45}
        \subfloat[Leading light jet.]{%
    \includegraphics[width=\twidth\textwidth]{fig/histos/L/reco/charge/allconst/L_JetConst_N_allconst_reco_leading_jet.png}%
}\hfil
        \subfloat[Second leading light jet.]{%
    \includegraphics[width=\twidth\textwidth]{fig/histos/L/reco/charge/allconst/L_JetConst_N_allconst_reco_scnd_leading_jet.png}%
}\\
        \subfloat[Leading \cPqb jet.]{%
    \includegraphics[width=\twidth\textwidth]{fig/histos/L/reco/charge/allconst/L_JetConst_N_allconst_reco_leading_b.png}%
}\hfil
        \subfloat[Second leading \cPqb jet.]{%
    \includegraphics[width=\twidth\textwidth]{fig/histos/L/reco/charge/allconst/L_JetConst_N_allconst_reco_scnd_leading_b.png}%
}
\caption{Total number of particles constituting the leading light jet, the second leading light jet, the leading \cPqb jet and the second leading \cPqb jet. In both cases all jet constituents are included.}
\label{fig:number}

\end{figure}

% \begin{linenomath}
\begin{figure}[hbtp]
\centering
\def\twidth{0.45}
\subfloat[Leading light jet.]{%
\includegraphics[width=0.4\linewidth]{fig/histos/L/reco/L_JetConst_EventChargedContentN_reco_leading_jet.png}%
}\hfil
\subfloat[Leading \cPqb jet.]{%
\includegraphics[width=0.4\linewidth]{fig/histos/L/reco/L_JetConst_EventChargedContentN_reco_leading_b.png}%
}
\caption{Ratio of the number of charged particles to the total number of particles constituting the leading light jet and the leading \cPqb jet.}
\label{fig:charged_contentN}
\end{figure}

\begin{figure}[hbtp]
\centering
\def\twidth{0.45}
\subfloat[Leading light jet.]{%
\includegraphics[width=\twidth\linewidth]{fig/histos/L/reco/L_JetConst_EventChargedContentE_reco_leading_jet.png}%
}\hfil
\subfloat[Leading \cPqb jet.]{%
\includegraphics[width=\twidth\linewidth]{fig/histos/L/reco/L_JetConst_EventChargedContentE_reco_leading_b.png}%
}
\caption{Ratio of the energy of charged particles to the total enery of particles constituting the leading jet and the leading \cPqb jet.}
\label{fig:charged_contentE}
\end{figure}
 % \end{linenomath}

Since we study light jets from the decay of the \PW boson it is interesting to ask why we need to concentrate on the \ttbar process. The \PW production cross section is $>20\times$ larger than the \ttbar cross section. In the study we need to use  $\PW\rightarrow\cPq\cPq'$ events as the leptonic decays do not have colour flow. It is hard to trigger on resolved $\PW\rightarrow\cPq\cPq'$ events with sufficiently low \pt thresholds, so we use \ttbar events where one of the \PW bosons decays leptonically and it is used to trigger the event while the other one decays hadronically and it is used to study colour flow.

The \PW boson belongs to the colour singlet:

\begin{equation}
\frac{1}{\sqrt{3}}\left(\text{R}\overline{\text{R}}+\text{G}\overline{\text{G}}+\text{B}\overline{\text{B}}\right),
\end{equation}

\noindent where $R$, $G$ and $B$ are the three quantum states of the colour wave function.

An object belonging to the colour singlet is colourless and cannot participate in the strong interaction. We mention this feature in light of our subsequent discussion of the colour octet \PW boson.

\section{Colour octet \PW boson}

A \PW boson belonging to the colour octet is assumed. Its colour wavefunctions can take any of the 8 combinations:

\begin{align}
\text{R}\overline{\text{G}}, &&
\text{R}\overline{\text{B}}, &&
\text{G}\overline{\text{R}}, &&
\text{G}\overline{\text{B}}, &&
\text{B}\overline{\text{R}}, &&
\text{B}\overline{\text{G}}, &&
\frac{1}{\sqrt{2}}\left(\text{R}\overline{\text{R}}-\text{G}\overline{\text{G}}\right), &&
\frac{1}{\sqrt{6}}\left(\text{R}\overline{\text{R}}+\text{G}\overline{\text{G}}-2\text{B}\overline{\text{B}}\right).
\end{align}

The only known particle in nature that belongs to the colour octet is the gluon. The colourful \PW boson is a purely hypothetical particle. The mass of the colour octet \PW boson is assumed to be equal to $m_{\PW}$. This boson would couple in colour field the light quarks to the hadronic \cPqb and the hadronic \cPqt, while the light quarks would become uncoupled from each other (Fig.~\ref{fig:ttbar_cf_octet}).

Although the existence of such a particle has not been confirmed, massive coulor-octet vector bosons (colourons) are predicted in a variety of models, including axigluon models, topcolour models, technicolour models with coloured technifermions, flavour-universal and chiral colouron models, and extra-dimensional models with $KK$ gluons~\cite{Chivukula:2013xla}. These states have also recently been suspected as a potential source~(\cite{Ferrario:2009bz}, \cite{Frampton:2009rk}) of the top-quark forward-backward asymmetry observed by the CDF collaboration~(\cite{Aaltonen:2008hc}, \cite{Aaltonen:2011kc}). Searches for resonances in the dijet mass spectrum at the LHC at \sqrts=7-8~\TeV imply that the lower bound on such a boson is now 2-3~\TeV~(\cite{Han:2010rf}, \cite{Haisch:2011up}, \cite{Chatrchyan:2011ns}, \cite{Aad:2011fq}), while more recent LHC searches at \sqrts=13~\TeV with integrated luminosity 27~\fbinv place an even higher lower bound of 6.1~\TeV~(\cite{CMS:2017xrr}).
  
  \begin{figure}[h!]
  \centering
  \includegraphics[width=0.4\textwidth]{fig/ttbar_cf_flip_cropped.pdf}
  \caption{Colour flow in the decay of a top quark pair involving a hypothetical colour octet \PW boson.}
  \label{fig:ttbar_cf_octet}
\end{figure}

