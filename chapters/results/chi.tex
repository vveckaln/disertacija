\label{sec:LEP_methodology}
In order to correctly apply the LEP methodology one needs to separate the \cPqb quarks on the hadronic and the leptonic branch. The methodology to achieve this goal was described in~\ref{chap:methodology}. As a test of the validity of the methodology, one can use the invariant mass of the \cPqt quark, formed by the sum of the \cPqb quark and the \PW boson. Fig.~\ref{fig:L_jet_mass_reco} shows the resonance of the \cPqt quark. Tables~\ref{tab:mass_L_reco_MC_nominal}-\ref{tab:mass_L_reco_MC_cflip} provide measurements of the masses of the \cPqt quark and the \PW boson on the hadronic and leptonic branches using a polynomial fit.

  \begin{figure}[hbtp]
    \def\twidth{0.45}
    \subfloat[Observed mass of \PW on the hadronic branch.]{%
    \includegraphics[width=\twidth\textwidth]{fig/histos/L/reco/L_jet_mass_reco_had_w.png}%
    \label{fig:L_jet_mass_reco_had_w}
}\hfil
  \subfloat[Observed mass of \cPqt on the hadronic branch.]{%
    \includegraphics[width=\twidth\textwidth]{fig/histos/L/reco/L_jet_mass_reco_had_t.png}%
    \label{fig:L_jet_mass_reco_had_t}
}\\

 \subfloat[Observed mass of \PW on the leptonic branch.]{%
    \includegraphics[width=\twidth\textwidth]{fig/histos/L/reco/L_jet_mass_reco_lept_w.png}%
    \label{fig:L_jet_mass_reco_lept_w}
  }\hfil
 \subfloat[Observed mass of \cPqt on the leptonic branch.]{%
    \includegraphics[width=\twidth\textwidth]{fig/histos/L/reco/L_jet_mass_reco_lept_t.png}%
    \label{fig:L_jet_mass_reco_lept_t}
}
 \caption{Observed masses of \PW and \cPqt on the hadronic and the leptonic branch.}
  \label{fig:L_jet_mass_reco}
\end{figure}


\input{tables/masses/mass_L_reco_MC_nominal.txt}

\input{tables/masses/mass_L_reco_data_nominal.txt}

\input{tables/masses/mass_L_reco_MC_cflip.txt}

Three types of flows are analysed:
\begin{itemize}
\item in particle flow all particles are assigned a weight equal to 1.0.
\item in energy flow particles are assigned a weight proportional to their energy normalised to the sum of the energy of the top quarks.
\item in \pt flow particles are assigned a weight proportional to their transverse momentum normalised to the transverse momentum of the respective jet.
\end{itemize}

The results of the LEP methodology using particle flow are shown in Fig.~\ref{fig:chi_allconst_N} with all jet constituents and in Fig.~\ref{fig:chi_chconst_N} including only charged jet constituents. The flow is plotted between the leading \cPqb jet \leadingb and the 2nd leading b jet \scndleadingb, the hadronic \cPqb jet $j_{h}^{\cPqb}$ and the furthest light quark $j_{f}^{\PW}$ (jet distance is measured with the angle between the spatial components of the 4-vectors of the jets), the closest light quark $j_{c}^{\PW}$ and the hadronic \cPqb jet $j_{h}^{\cPqb}$, and the leading light jet \leadingjet and the second leading light jet \scndleadingjet.

The results using energy flow are shown in Fig.~\ref{fig:chi_allconst_E} with all jet constituents and in Fig.~\ref{fig:chi_chconst_E} including only charged jet constituents.

The results o using \pt flow are shown in Fig.~\ref{fig:chi_allconst_Pt} with all jet constituents and in Fig.~\ref{fig:chi_chconst_Pt} including only charged jet constituents.

In all cases the density drops in the middle area between jets compared to the jet centre with the central density varying between colour connect jets and jets not connected in colour.

The bin-per-bin ratios of the flow in colour-free regions (\leadingb, \scndleadingb), ($j_{h}^{\cPqb}$, $j_{f}^{\PW}$), ($j_{c}^{\PW}$, $j_{h}^{\cPqb}$) to the flow in the colour-connected region (\leadingjet, \scndleadingjet) are given in Fig.~\ref{fig:chirg_allconst_N} including all jet constituents and Fig.~\ref{fig:chirg_chconst_N} including only charged jet constituents. Significant colour reconnection is noticeable in the region ($j_{c}^{\PW}$, $j_{h}^{\cPqb}$) assuming the colour octet \PW model.

Fig.~\ref{fig:ratio_qlq2l} shows the bin-per-bin ratio of the particle flow in the region (\leadingjet, \scndleadingjet) in the the colour octet \PW model to the particle flow in the region (\leadingjet, \scndleadingjet) in the SM model. A loss of colour connection in this region in the colour octet \PW model is evident.

As a quantitative result from the LEP methodology one can use the parameter $R$ which is defined as the ratio between the integral from 0.2 to 0.8 in the colour connected region to the integral from 0.2 to 0.8 in the region not connected in colour:

\begin{equation}
R=\frac{\int_{0.2}^{0.8}f^{\text{inter \PW region}}d\chi}{\int_{0.2}^{0.8}f^{\text{intra \PW region}}d\chi},
\end{equation}

\noindent where $f(\chi)$ is the density of the flow distribution.

This parameter was used at LEP to quantify colour connection effects and their values from different experiments corresponding to 625~\pbinv of data in the range $\sqrt{s}=189-209$~\GeV are given in Table~\ref{tab:LEP_R}. We note inconsistency in the $R$ values reported by different experiments. Furthermore, $R$ should exceed 1 on theoretical basis. The range 0.2 - 0.8 is identified as sensitive to colour-connection effects. 
\begin{table}
\centering
\caption{$R$ values observed at LEP.}
\label{tab:LEP_R}
\begin{tabular}{lll}
LEP experiment & $R$ value - data                                        & reference\\
\hline
    OPAL       & 1.243                                                   & \cite{Abbiendi:2005es}\\
    Delphi     & 0.889 ($\sqrt{s}=183$~\GeV)-1.039 ($\sqrt{s}=207$~\GeV) & \cite{Abdallah:2006uq}\\
    L3         & 0.911                                                   & \cite{Achard:2003pe}\\
  \end{tabular}
\end{table} 

In our case we use 3 $R$ values for any of the regions not connected in colour with normalisation to the colour connected region (\leadingjet, \scndleadingjet).

The integral of particle flow from 0.2 to 0.8 in different regions and the inverse of $R$ values for the SM model is given Table~\ref{tab:R_L_reco_N_MC_SM}, for data in Table~\ref{tab:R_L_reco_N_data_SM} and for the \PW colour octet model in Table~\ref{tab:R_L_reco_N_MC_cflip}.

The integral of energy flow from 0.2 to 0.8 in different regions and the inverse of $R$ values for the SM model is given Table~\ref{tab:R_L_reco_E_MC_SM}, for data in Table~\ref{tab:R_L_reco_E_data_SM} and for the \PW colour octet model in Table~\ref{tab:R_L_reco_E_MC_cflip}.

The integral of \pt flow from 0.2 to 0.8 in different regions and the inverse of $R$ values for the SM model is given Table~\ref{tab:R_L_reco_Pt_MC_SM}, for data in Table~\ref{tab:R_L_reco_Pt_data_SM} and for the \PW colour octet model in Table~\ref{tab:R_L_reco_Pt_MC_cflip}.


\figureChi{allconst}{N}

\figureChi{chconst}{N}

\figureChi{allconst}{E}

\figureChi{chconst}{E}

\figureChi{allconst}{Pt}

\figureChi{chconst}{Pt}

\figureratiographs{allconst}{N}

\figureratiographs{chconst}{N}

\begin{figure}[htpb]
\def\twidth{0.45}
\centering
\subfloat[Result using all jet constituents.]{%
\includegraphics[width=\twidth\textwidth]{fig/ratiographs_merged_SM/L_qlq2l_N_allconst_reco.png}%
\label{fig:L_qlq2l_N_allconst_reco}
}\hfil
\subfloat[Result using only charged jet constituents.]{%
\includegraphics[width=\twidth\textwidth]{fig/ratiographs_merged_SM/L_qlq2l_N_chconst_reco.png}%
\label{fig:L_qlq2l_N_chconst_reco}
}
\caption{Bin-per-bin ratio of particle flow in region (\leadingjet, \scndleadingjet) in the \PW colour octet model to particle flow in region (\leadingjet, \scndleadingjet) in the SM model.}
\label{fig:ratio_qlq2l}
\end{figure}

\clearpage

\input{tables/Rvalues_SM/R_L_reco_MC_N_SM.txt}

\input{tables/Rvalues_SM/R_L_reco_data_N_SM.txt}

\input{tables/Rvalues_cflip/R_L_reco_MC_N_cflip.txt}

\input{tables/Rvalues_SM/R_L_reco_MC_E_SM.txt}

\input{tables/Rvalues_SM/R_L_reco_data_E_SM.txt}

\input{tables/Rvalues_cflip/R_L_reco_MC_E_cflip.txt}

\input{tables/Rvalues_SM/R_L_reco_MC_Pt_SM.txt}

\input{tables/Rvalues_SM/R_L_reco_data_Pt_SM.txt}

\input{tables/Rvalues_cflip/R_L_reco_MC_Pt_cflip.txt}

