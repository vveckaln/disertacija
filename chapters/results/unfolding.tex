When a detector makes an observation the end results suffers from the inefficiencies of the detector. Unfolding is a method where the observation made at the detector is corrected for detector effects. Hence we can obtain an estimate of the true distribution of the observable. However, it comes at a cost of a signficant loss of granularity of the phase-space of the observable.

We infer about the detector effects because in Monte Carlo samples each generated event is reconstructed. Therefore an observable in bin $i$ at generation level migrates to bin $k$ at reconstruction level. By accumulating a large number of events we obtain statistics of migration. In unfolding we revert the migration - given an observable at bin $k$ we assign probabilities to the true values of the observable.

Values of \pullangle at generation level that do not have a corresponding value at reconstruction level are put in the underflow bin at reconstruction level. Values of \pullangle at reconstruction level that do not have a corresponding value at generation level are put in the underflow bin at generation level. The underflow bins at generation level are treated as background and are removed. Distributions that are not filled at generation level - data and MC backgrounds are reduced by a corresponding scale factor. The underflow bin at reconstruction level is used to constrain the underflow bin for the unfolded result.

Unfolding is performed on data from which the MC backgrounds have been subtracted. We also performed the unfolding procedure in reverse obtaining the folded back output.

We are interested to have the migration matrix as diagonal as possible to reduce statistical uncertainties on the unfolding result. Two measures are used to characterise the share of statistics on the diagonal of the migration matrix - stability and purity. Stability is the ratio of the contents of the diagonal element to the total number of events at reconstruction level in the bin:

\begin{equation}
  \text{stability}\equiv\frac{\theta^{\text{diag}}_{\text{input}}}{\Sigma_{x=1}^{x=N_{x}}\theta^{x}_{\text{input}}},
\end{equation}

where $x$ is the bin index at reconstruction level, starting the numbering from 1 and $N_{x}$ is the number of bins at reconstruction level. Purity is the ratio of the contents of the diagonal element to the total number of events at generation level in the bin:

\begin{equation}
  \text{purity}\equiv\frac{\theta^{\text{diag}}_{\text{input}}}{\Sigma_{y=1}^{y=N_{y}}\theta^{y}_{\text{input}}},
\end{equation}

where we have used $y$ as the bin index at generation level. The values of purity and stability are recommended to exceed 50 \% at each bin.

An interesting measure is the amount by which the unfolded result is different from the generated result at MC (an ideal result would be 0), normalised to statistical uncertainty of the unfolded result. This measure is called the pull. A mathematical expression for the pull is

\begin{equation}
  \text{pull}\equiv\frac{\theta^{\text{gen}}_{\text{unf}}-\theta^{\text{gen}}_{\text{in}}}{\sigma^{\text{gen}}_{\text{unf}}}.
\end{equation}

We generate random toy distributions of the observable at generation level, thus obtaining a distribution of the pull.

The number of bins at generation level is reduced by a factor of 2 with regard to the number of bins at reconstruction level in order that unfolding be computationally feasible.

The class \lstinline[language=sh]|TUnfoldDensity| \cite{Schmitt:2012kp} of \ROOT is used to do the unfolding procedure. The binning scheme is managed with class \lstinline[language=sh]|TUnfoldBinning|. No regularisation is applied. The unfolding results of \pullangle from \leadingjet to \scndleadingjet including all jet constituents are shown in Fig. \ref{fig:unfolding_nominal_leading_jet_allconst_pull_angle_ORIG_MC13TeV_TTJets_ORIG}. In order to create the plots shown herein a new class \lstinline[language=sh]|CompoundHistoUnfolding| \cite{url:compoundhistounfolding} was developed which was added to \ROOT complete with input and output streamers.

The unfolding results are shown in Fig. \ref{fig:unfolding_nominal_leading_jet_allconst_pull_angle_ORIG_MC13TeV_TTJets_ORIG}. Distributions corresponding to unfolding results with migration matrices from \ttbar\ \HERWIGpp and \ttbar cflip as well as systematics \ttbar fsr dn and \ttbar fsr up (see Chap. \ref{chap:systematic_uncertainties}) are laid over the unfolding plots. In the unfolded distibution there are very large uncertainties and poor purity and stability in most of the bins. In order to mitigate these effects a bin optimisation algorithm was tried. The algorithm proceeds as follows:

\begin{itemize}
\item In each bin of the reconstructed observable, the particle-level distribution is fitted with a gaussian distribution.
\item Starting from the low edge of the reconstructed distribution, a bin $i$ is searched that fulfils $\mu_{i}-f\sigma_{i}/2 > 0$, $\mu_{i}$ being the mean and $\sigma_{i}$ the standard distribution from the fit at generation level in each bin $i$ at reconstruction level. The factor $f$ is chosen so that $f\sigma\sim\frac{\theta_{p,\max}-\theta_{p, \min}}{3}$, in order to obtain 4-3 optimised bins. In practice, this factor has to be chosen very small - 0.15 for pull angle and 0.3 for the magnitude of the pull vector, indicating that $\sigma$ is large compared to the range of the phase space of the pull angle.
\item The new optimised bin is then defined from 0 to $\text{mean}+f\text{sigma}_{i}/2$. 
\item The algorithm is iterated until the edge of the histogram is reached.
\item The obtained binning is used to present the result on particle (i.e. generation) level. To obtain the final migration matrix used in the unfolding each bin at particle level is split by two to obtain a suitable reconstruction level binning.
\end{itemize}

This algorithm is depicted in Fig. \ref{fig:gaussiancurves}. The parameter $b$ is given by $\frac{\sigma_{n}}{\sigma_{n} + \sigma_{n+1}}$.

\begin{figure}
  \centering
  \includegraphics[width = 0.8\textwidth]{fig/gaussiancurves}
  \caption{Method of optimising the binning scheme for unfolding.}
  \label{fig:gaussiancurves}
\end{figure}

The unfolded result with the optimised binning is shown in Fig. \ref{fig:unfolding_nominal_leading_jet_allconst_pull_angle_OPT_MC13TeV_TTJets_SIGMA_0p15}. The purity and stability in the central bin is still poor. Therefore a scheme using 3 regular bins as in the ATLAS analysis \cite{ATLAS:2017iaz} was tried.

The results with the regular binning scheme are shown in Fig. \ref{fig:unfolding_nominal_leading_jet_allconst_pull_angle_OPT_MC13TeV_TTJets_ATLAS3}. The stability and purity levels with this binning scheme reach acceptable levels at each bin and it was adopted for further analysis.

The unfolding results using the migration matrix from \ttbar cflip are shown in Fig. \ref{fig:unfolding_cflip_leading_jet_allconst_pull_angle_OPT_MC13TeV_TTJets_cflip_ATLAS3}. The \ttbar cflip is included as a systematic for \ttbar.

The unfolding results of the \pullangle from \leadingb to \scndleadingb with all jet constituents are shown in Fig. \ref{fig:unfolding_nominal_leading_b_allconst_pull_angle_OPT_MC13TeV_TTJets_ATLAS3}.

As an additional observable the magnitude of the pull vector \pvmag was unfolded. Fig. \ref{fig:unfolding_nominal_leading_jet_allconst_pvmag_OPT_MC13TeV_TTJets_ATLAS3} shows the unfolding results of \pvmag from \leadingjet to \scndleadingjet including all jet constituents.

The bin-per-bin significance (\%) of nuisances in the total systematical error in the unfolded result are given in Table \ref{tab:unc_table_fullpull_angle_OPT_allconst_gen_out_MC13TeV_TTJets_nominal_ATLAS3}. Nuisances that directly affect the hadronisation \ttbar Herwig++, \ttbar QCDbased and \ttbar ERDon are the most significant.

In addition to the \POWHEG+\PYTHIA 8 sample, we also investingate a \POWHEG+\PYTHIA 8 * sample in which \ttbar cflip has been added as a systematic to \ttbar. Table \ref{tab:unc_table_fullpull_angle_OPT_allconst_gen_out_MC13TeV_TTJets_cflip_ATLAS3} shows the aditional bin-per-bin $t\overline{t}$ cflip uncertainty for the \POWHEG+\PYTHIA 8 * sample.

The agreement between the unfolded result and MC prediction at generation level is quantified using a goodness-of-fit method. Given the normalised unfolded detector observation $D$, the normalised MC prediction $M$, the full covariance matrix $\Sigma$ of normalised experimental uncertainties, the $\chi^{2}$ is calculated as follows:

\begin{equation}
  \chi^{2}=(D^{T}-M^{T})\cdot\Sigma^{-1}\cdot(D-M).
  \label{eq:chi2}
\end{equation}

From the $\chi^{2}$ value the p-value can be computed using the number of degrees of freedom equal to the number of bins in the unfolded distribution subtracted by 1 to account for a loss of freedom when normalising the distributions. One row and one column is discarded from the covariance matrix $\Sigma$. $\chi^{2}$ value does not depend on the choice of the discarded elements.

Table \ref{tab:chi_table_pull_angle_OPT_allconst_nominal_ATLAS3} shows the $\chi^{2}$ values and p-values for \pullangle using all jet constituents. The results show that the pull angle distribution is poorly modelled by the MC generators. In general, the simulation predicts a more sloped distribution, i.e. a stronger colour flow effect. \HERWIGpp models better the pull angle distribution than \PYTHIA 8.2. Accuracy of \PYTHIA 8.2 is particularly poor when predicting the distribution of \pullangle from \scndleadingjet to \leadingjet.

The $\chi^{2}$ values and p-values for the \PW colour octet model are given in Table \ref{tab:chi_table_pull_angle_OPT_allconst_cflip_ATLAS3}. In the colour flip model the distribution of \pullangle from \leadingjet to \scndleadingjet is modelled less acurately than the SM prediction.
  
Table \ref{tab:chi_table_pull_angle_OPT_allconst_MC13TeV_TTJets_nominal_ATLAS3_full} shows the values of $\chi^{2}$ and if signal $M$ in Eq. \ref{eq:chi2} is replaced by the respective systematic, but leaving the covariance matrix $\Sigma$ unchanged. The agreement is better than \ttbar when the colour flow is modelled by \ttbar ERDOn, \ttbar Herwig ++ and \ttbar QCD based.

\figunfolding{nominal}{leading_jet}{allconst}{pull_angle}{ORIG}{MC13TeV_TTJets}

\figunfolding{nominal}{leading_jet}{allconst}{pull_angle}{SIGMA_0p15}{MC13TeV_TTJets}

\figunfolding{nominal}{leading_jet}{allconst}{pull_angle}{ATLAS3}{MC13TeV_TTJets}

\figunfolding{cflip}{leading_jet}{allconst}{pull_angle}{ATLAS3}{MC13TeV_TTJets_cflip}

\figunfolding{nominal}{leading_b}{allconst}{pull_angle}{ATLAS3}{MC13TeV_TTJets}

\figunfolding{nominal}{leading_jet}{allconst}{pvmag}{ATLAS3}{MC13TeV_TTJets}

\input{tables/unc_nominal_full/pull_angle/ATLAS3/unc_table_full_leading_jet_allconst_pull_angle_OPT_gen_out_MC13TeV_TTJets.txt}

\input{tables/unc_cflip_full/pull_angle/ATLAS3/unc_table_full_leading_jet_allconst_pull_angle_OPT_gen_out_MC13TeV_TTJets.txt}

\input{tables/chi_nominal/pull_angle/ATLAS3/chi_table_pull_angle_OPT_allconst.txt}

\input{tables/chi_cflip/pull_angle/ATLAS3/chi_table_pull_angle_OPT_allconst.txt}

\input{tables/chi_nominal/pvmag/ATLAS3/chi_table_pvmag_OPT_allconst.txt}

\input{tables/chi_cflip/pvmag/ATLAS3/chi_table_pvmag_OPT_allconst.txt}

\input{tables/chi_full_nominal/pull_angle/ATLAS3/chi_table_pull_angle_OPT_allconst_MC13TeV_TTJets_full.txt}
