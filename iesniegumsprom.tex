\documentclass{article}
\usepackage[utf8]{inputenc}
\usepackage{ragged2e}
\usepackage{graphicx}
\usepackage{array}
\thispagestyle{empty}
\begin{document}
\noindent RTU EEF promocijas padomes sekretāram\\\noindent Kārlim Ketneram\\

\hfill\begin{minipage}{\dimexpr\textwidth-6cm}
\begin{flushright}
  doktora studiju programmas ``Elektrotehnoloģiju datorvadība'' doktoranta
  Viestura Veckalna
\end{flushright}
\end{minipage}
\centering
\vskip 3em
\setlength{\parskip}{0.5em}
\textbf{iesniegums}
\justify
Lūdzu pieņemt doktora grāda saņemšanai sekojošus dokumentus:

\begin{enumerate}
\item promocijas darbu,

\item promocijas darba kopsavilkumu latviešu valodā,

\item augstskolas izziņu par programmas izpildi vai eksāmenu nokārtošanu izvēlētajā nozarē, apakšnozarē un svešvalodā,

\item dzīves aprakstu,

\item zinātnisko publikāciju sarakstu un to kopijas,

\item izrakstu no augstskolas sēdes protokola, kas apliecina promocijas darba apspriešanu, tā zinātnisko novitāti un manu personisko ieguldījumu.
\end{enumerate}

Promocijas darbs ir izstrādāts par tēmu ``Virsotnes kvarku pāra sabrukšanas ceļā radušos krāsu plūsmu pētījumi ar 13 TeV CERN LHP KMS eksperimentā". Promocijas darbs ir izstrādāts sadarbībā ar Eiropas kodolpētījumu organizāciju.

Kopā ir sagatavotas 368 publikācijas, kā arī ir noritējušas 3 uzstāšanās starptautiskās konferencēs, 6 uzstāšanās CERN CMS ``Top Generators and Modelling'' sanāksmēs, plakāta prezentācija CERN Zinātnes nedēļā 2017. g. un European School of High Energy Physics 2017. g.
\vskip 3em
\setlength\tabcolsep{0pt}
\begin{tabular}{m{4cm}m{4cm}}
  Viesturs Veckalns & \hspace{3cm}
\end{tabular}

Rīgā, 2018. g. 18. februārī
\end{document}
