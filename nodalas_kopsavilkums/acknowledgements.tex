Pamati šim darbam tika ielikti Pikosekunžu silīcija reizinātāju-elektronikas-kristālu pētniecības (\textit{Picosecond Siliconphotomultiplier-Electronics-Crystal research}) Marī Kirī tīkla projekta ietvaros. Izsaku pateicību Etjenetei Ofrē (\textit{oriģ.} Etiennette Auffray) (CERN, Switzerland) par šī projekta organizēšanu. Pateicos arī Mikelem Galinaro (\textit{oriģ.} Michele Gallinaro) (LIP, Portugal) par ievadu darbam KMS eksperimentā.

Pateicos arī saviem komandas biedriem Marteinam Muldersam (\textit{oriģ.} Martijn Mulders) (CERN, Switzerland), Pedru Silvam (\textit{oriģ.} Pedro Silva) (CERN, Switzerland) un Markusam Zeidelsam (\textit{oriģ.} Markus Seidel) (CERN, Switzerland) par metodoloģisko atbalstu un dalīšanos ar pieredzi.

Tāpat pateicos Rīgas Tehniskajai universitātei par nemitīgo atbalstu, kā rezultātā radās iespēja turpināt darbu KMS eksperimentā.
