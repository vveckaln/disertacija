Pamati šim darbam tika ielikti Pikosekunžu silīcija reizinātāju-elektronikas-kristālu pētniecības (\textit{Picosecond Siliconphotomultiplier-Electronics-Crystal research}) Marī Kirī tīkla projekta ietvaros. Izsaku pateicību Etjenetei Ofrē (oriģ. Etiennette Auffray, CERN, Šveice) par šī projekta organizēšanu. Pateicos arī Mikelem Galinaro (oriģ. Michele Gallinaro, LIP, Portugāle) par ievadu darbam KMS eksperimentā.

Pateicos arī saviem komandas biedriem Marteinam Muldersam (oriģ. Martijn Mulders, CERN, Šveice), Pedru Silvam (oriģ. Pedro Silva, CERN, Šveice) un Markusam Zeidelsam (oriģ. Markus Seidel, CERN, Šveice) par metodoloģisko atbalstu un dalīšanos ar pieredzi.

Tāpat pateicos Rīgas Tehniskajai universitātei par nemitīgo atbalstu, kā rezultātā radās iespēja turpināt darbu KMS eksperimentā.
