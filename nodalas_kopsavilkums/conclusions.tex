Esam guvuši pārliecību, ka vilkmes leņķa metode, kas balstīta labā treku rekonstrukcijā, ir jutīga ar krāsām saistītu strūklu identificēšanā. Vilkmes leņķa sadalījumā skaidri izšķirams paugurs, kas centrēts 0~rad ar krāsām saistītu strūklu gadījumā. Vilkmes leņķa sadalījums ir monotons gadījumā, ja strūklas nav saistītas ar krāsām.

Pārliecinosi rezultāti ir gūti arī, pielietojot ``LEP metodi''. Daļiņu blīvums ir lielāks starp strūklām, kas saistītas ar krāsām, nekā bezkrāsu reģionos. 

Rezultāti tika salīdzināti ar krāsu okteta \PW paraugiem, kuros krāsu saistība starp hadroniskajiem sabrukuma produktiem tika noņemta. Šādā gadījumā, pielietojot vilkmes leņķa metodi un ``LEP metodi'', vieglās strūklas atbilst ar krāsām nesaistītām strūklām. 

Veicām vilkmes leņķa atlocīšanas procedūru, lai iegūtu tā patieso distribūciju pirms dektorā veiktās rekonstrukcijas. Būtiskas izmaiņas secinājumos atlocīšanas procedūra neieviesa.

Pamanāms, ka \POWHEG + \PYTHIA MK simulācijas pārspīlē krāsu efektus salīdzinājumā ar detektora reālās pasaules novērojumiem. Šie pārspīlējumi redzami kā izteiktāks paugurs vilkmes leņķa sadalījumā MK simulācijā. \HERWIGpp un atsevišķi \PYTHIA uzskaņojumi krāsu saistību hadronizācijā modelē precīzāk.

Kopumā, sakritība starp datiem un MK nav īpaši laba. $\sim\frac{2}{3}$ \ttbar kombinācija ar $\sim\frac{1}{3}$ $\ttbar\ cflip$ labāk atbilst detektora novērojumiem. Šādi rezultāti tika iegūti, veicot hipotēžu pārbaudi.
