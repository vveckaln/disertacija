Mēs meklējam krāsu saistības eksperimentālo nospiedumu starp hadronu \gls{strūklām}, kuras rodas, sabrūkot \gls{virsotnes kvarku} pārim. Krāsu saistība rodas tādēļ, ka sabrūkošais \PW bozons pieder krāsu \gls{singletam}. Virsotnes kvarku pāris rodas \Pp\Pp sadursmēs ar $\sqrt{s}=13~\TeV$ momenta centra enerģiju. Novērojumi tiek veikti CERN \gls{LHP} \gls{KMS} eksperimentā. Mūs īpaši interesē vieglās strūklas, kas rodas, hadroniski sabrūkot \PW bozonam. Tās ir saistītas ar krāsām, un eksperimentāli mēs par to varam pārliecināties netieši. Tāpat pētām arī hipotētiska krāstu okteta \PW bozona sabrukšanu. Šajā gadījumā vieglās strūklas nav saistītas ar krāsām, un šos rezultātus varam izmantot, lai tos salīdzinātu ar ar krāsām saistīto gadījumu.  

Izmantojam vilkmes leņķa metodi~\cite{Gallicchio:2010sw}. Šī metode ir tikusi pielietota Fermilab Tevatrona \DZERO eksperimentā~\cite{Abazov:2011vh}, ATLAS eksperimentā I darba periodā~\cite{Aad:2015lxa}, kā arī ATLAS eksperimentā II darba periodā \cite{Aaboud:2018ibj}. Šo metodi CMS pirmoreiz pielietoja Zeidels, M. un citi~\cite{indico:Markus_cf}, taču šie rezultāti nekad nav tikuši publicēti. Salīdzinājumā ar ATLAS KMS detektora iespējama aptuveni divreiz labāka centrālā reģiona treku momenta izšķirtspēja, pateicoties tā 4~T solenoīdam (ATLAS aprīkots ar daudz mazāku 2~T solenoīdu ar lieliem toroīda magnētiem ārpusē~\cite{Aad:2008zzm}).

Tāpat tiek izmantota arī adaptācija metodei, kas tikusi izmantota LEP (turpmāk to dēvēsim par ``LEP metodi''), kur strūklu sastāvdaļas tiek projicētas uz starpstrūklu plaknēm~\cite{Abbiendi:2005es}, \cite{Abdallah:2006uq}, \cite{Achard:2003pe}. Šī metode vēl nav tikusi pielietota LHP.

Šajā disertācijā ir izklāstīti KMS eksperimenta virsotnes kvarka grupas pētījuma ietvaros gūtie rezultāti. Rezultāti dažādās stadijās ir tikuši prezentēti Virsotnes modelēšanas un ģeneratoru fizikas sanāksmēs - 2016. g. 19. janvārī, 2016. g. 29. martā, 2016. g. 7. jūnijā, 2016. g. 30. augustā, 2018. g. 13. februārī un 2018. g. 17. oktobrī.

Darbam, kas aprakstīts šajā disertācijā, noritot pilnā sparā, mēs 2018. g. maijā svinīgi atzīmējām uzņemšanu par pilntiesīgiem KMS ekperimenta biedriem. Šis darbs ir pirmais Latvijas pienesums CERN LHP eksperimentālajai programmai.
