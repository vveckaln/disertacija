Mēs meklējam eksperimentālo nospiedumu krāsu saistībai starp hadronu \gls{strūklām}, kuras rodas, sabrūkot \gls{virsotnes kvarku} pārim. Krāsu saistība rodas tādēļ, ka sabrūkošais \PW bozons pieder krāsu \gls{singletam}. Virsotnes kvarku pāris rodas \Pp\Pp sadursmēs ar $\sqrt{s}=13\TeV$ momenta centra enerģiju. Novērojumi tiek veikti CERN \gls{LHP} \gls{KMS} eksperimentā. Galvenais novērojamais lielums ir \gls{vilkmes leņķis} \cite{Gallicchio:2010sw}. Izmantojam arī \gls{LEP} izstrādātas metodologijas adaptāciju saskaņā ar kuru strūklu sastāvdaļas tiek projicētas uz starpstrūklu plaknēm \cite{Abbiendi:2005es}, \cite{Abdallah:2006uq}, \cite{Achard:2003pe}. Salīdzinot ar strūklām, kas nav saistītas ar krāsām, redzams izteiks krāsu saistības eksperimentālais nospiedums. Mēs arī eksperimentāli pētam krāsu saistību starp strūklām, kas rodas, sabrūkot hipotētiskam krāsu okteta \PW bozonam. 

Šajā disertācijā ir izklāstīti KMS eksperimenta virostnes kvarka grupas pētījuma ietvaros gūtie rezultāti. Rezultāti dažādās stadijās ir tikuši prezentēti Virsotnes modelēšenas un ģeneratoru fizikas sanāksmēs - 2016. g. 19. janvāri, 2016. g. 29. martā, 2016. g. 7. jūnija, 2016. g. 30. augustā, 2018. g. 30. augustā, 2018. g. 13. februāri un 2018. g. 17. oktorbī.

Uz šo brīdi disertācijā gūtie rezultāti nav oficiāli apstiprināti saskaņā ar KMS eksperimenta apstiprināšanas procedūru \cite{twiki:PhysicsApprovals}. Līdz ar to tie nav uzskatāmi par KMS publicējamu rezultātu un grafiki ir atzīmēti kā personīgais darbs. KMS apstiprināšana ir paredzēta kā šīs analīzes turpmākais solis.

Darbam, kas aprakstīts šajā disertācijā, noritot pilnā sparā, mēs 2018. g. maijā svinīgi atzīmējām uzņemšanu par pilntiesīgiem KMS ekperimenta biedriem. Šis darbs ir pirmais Latvijas pienesums CERN LHP eksperimentālajai programmai.
