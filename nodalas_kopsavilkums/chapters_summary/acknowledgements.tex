Pamati šim darbam tika ielikti Pikosekunžu silīcija fotoreizinātāju-elektronikas-kristālu pētniecības (angl. \textit{Picosecond Siliconphotomultiplier-Electronics-Crystal research}) Marī Kirī Tīkla projekta ietvaros. Izsaku pateicību Etjenetei Ofrē (orig. \textit{Etiennette Auffray}) (CERN, Šveice) par šī projekta organizēšanu. Izsaku pateicību arī Mikelem Galinaro (orig. \textit{Michele Gallinaro}) (LIP, Portugāle) par ievadu darbam KMS eksperimentā.

Tāpat pateicos maniem komandas biedriem Marteinam Muldersam (orig. \textit{Martijn Mulders}) (CERN, Šveice), Pedru Silvama (orig. \textit{Pedro Silva}) (CERN, Šveice) un  Markusam Zeidelam (orig. \textit{Markus Seidel})(CERN, Šveice) par metodoloģisko atbalstu un dalīšanos ar pieredzi.

Pateicos arī Rīgas Tehniskajai universitātei par nemitīgo atbalstu, kas pavēra iespēju turpināt darbu KMS eksperimentā.
