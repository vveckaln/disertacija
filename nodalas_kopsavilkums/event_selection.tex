Notikumu atlases mērķis ir nodalīt signālu no fona. Notikumu atlase ir atšķirīga detektora līmeņa MK notikumiem un ģeneratora līmeņa MK notikumiem. Dati tiek atlasīti at bilstoši detektora līmeņa notikumu atlasei.

\section{Detektora līmenis}
\label{sec:detector_level}

Notikumu atlase ir balstīta \ttbar$\to$ leptons + strūklas topoloģijā, kur viens no \PW bozoniem sabrūk par lādētu leptonu($\ell=e, \mu$) un atbilstošo neitrino, bet otrs \PW bozons sabrūk par kvarkiem, kas rada strūklas.

Lai rekonstruētu gala stadijas objektus, tiek izmantots daļiņu plūsmas algoritms \cite{Sirunyan:2017ulk}. Lai celtu rekonstrukcijas kvalitāti, šis algoritms kombinē signālus no visiem apakšdetektoriem, un ar tā palīdzību var identificēt mionus, elektronus, fotonus, lādētos hadronus un neitrālos hadronus pēc \Pp\Pp sadursmes.

Datu paraugi tiek ievākti, izmantojot \gls{Augsta līmeņa trigera} viena leptona trigera ceļus, kas apkopoti \ref{tab:triggers} tabulā.

\begin{table}[htp]
\centering
\caption{Analīzē izmantotās tiešsaistes atlases trigera ceļi.}
\label{tab:triggers}
\begin{tabularx}{\linewidth}{lllXX}\hline
Gala stadija                & Ceļš                                       & Darbības intervāls & Funkcija & L1 sākums\\\hline
e+strūklas                      & \small HLT\_Ele32\_eta2p1\_WPTight\_Gsf\_v & visi       & \small Atlasīt $e$ ar $\left|\eta\right|<,1$ un $\pt>32$ ciešajā darbības punktā, izmantojot GSF, lai rekonstruētu trekus
                                                                                         & \small L1\_SingleEG40\newline OR\newline L1\_SingleIsoEG22er\newline OR\newline L1\_SingleIsoEG24er\newline OR\newline L1\_SingleIsoEG24\newline OR\newline L1\_SingleIsoEG26\\\hline
\multirow[t]{2}{*}{$\mu$+strūklas}
                            & \small HLT\_IsoMu24\_v                     & visi       & \small Atlasīt izolētu $\mu$ ar $\pt>20$ GeV, izmantojot L3 trekera algoritmu
                                                                                         & \multirow[t]{2}{*}{\small L1\_SingleMu18}\\
                            & \small HLT\_IsoTkMu24\_v                   & visi       & \small Atlasīt izolētu $\mu$ ar $\pt>20$ GeV, izmantojot HLT trekera mionu algoritmu
                            & \\\hline
\end{tabularx}
\end{table}

Bezsaistē tiek prasīts viens ciešs elektrons/mions ar $\pt > 34/26\GeV$ un $|\eta|<2,1/2,4$. Notikumam, kurā ir otrs vaļīgs leptons ar $\pt > 15\GeV$ un $|\eta| < 2,4$, tiek uzlikts veto.

Notikumā jābūt četrām strūklām, kas sakopotas ar anti-$k_{t}$ algoritmu, strūklu atdalījumu ar $R=0,4$ un lādēto hadronu atņemšanu (AK4PFchs) un kurām $\pt>30\GeV$ un $|\eta|<2,4$. 

Vismaz divām strūklām jābūt \cPqb-atzīmētām ar Combined Secondary Vertex algoritma (CSVv2) vidējo darbības punktu. 

Notīkumā jābūt vismas divām neatzīmētām (vieglajām) strūklām, kurām jārada \PW bozona kandidāts, kura invariantā masa $\left|m_{jj}-80,4\right|<15\GeV$.

Notikumu raža dažādās atlases stadijās ir attēlota \ref{fig:_reco_selection} att. un sniegta \ref{tab:yields} tabulā. \ref{tab:yields_cflip} tabulā attēlota notikumu raža krāsu okteta \PW paraugam. Signāla īpatsvars palielinās no 0,1 \% sākotnējā stadijā līdz 94,2 \% gala stadijā - tas ir šīs atlases efektivitātes mērs.

\figureEML{/reco/}{_reco_selection}{Notikumu raža dažādās notikumu atlases stadijās: $1 \ell$, $1 \ell + \geq 4 j$, $1 \ell + \geq 4 j (2 b)$, $1 \ell + \geq 4 j (2 b, 2 lj)$.}

\input{tables/event_yields_tableslv/event_yields_table.txt}

\input{tables/event_yields_tableslv/event_yields_table_cflip.txt}

\section{Ģeneratora līmenis}
\label{sec:generator_level}

Simulācijā bezsaistes atlase daļiņu līmenī tiek imitēta, izmantojot \PSEUDOTOPPRODUCER rīku \cite{code:pseudotop}, izmantojot kopīgu leptonu atlasi gan elektroniem gan mioniem ar $\pt>26\GeV$ un $|\eta| < 2,4$, kā arī tādām pašām prasībām strūklām $\pt/\eta$ ($\pt>30\GeV$, $|\eta| < 2.4$) un \PW masai ($\left|m_{jj}-80.4\right|<15\GeV$) kā bezsaistes atlasei detektora līmenī.

Lādētie leptoni, kas izdalās cietajā procesā, tiek apzaroti ar leptoniem $R=0,1$ konusā. Strūklas tiek sakopotas ar anti-$k_T$ algoritmu ar $R=0,4$ konusu pēc apzaroto leptonu kā arī visu neitrino noņemšanas. Lai noteiktu strūklas smaržu daļiņu līmenī, strūklas sakopojumā tiek iekļauti ``spoku'' B-hadroni, pēc tam kad to moments ir reizināts ar $10^{-20}$ tā, lai tie būtiski nemainītu strūklas enerģijas mērogu daļiņu līmenī.
