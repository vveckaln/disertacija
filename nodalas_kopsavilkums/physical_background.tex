Virsotnes kvarka pāra šķērsgriezums protonu-protonu sadursmēs ar \sqrts=13\TeV saskaņā ar mērījumiem ir 803 pb \cite{Sirunyan:2018goh}. Šķērsgriezums palielinās, pieaugot masas centra enerģijai. Šķērsgriezums kā funkcija no masas centra enerģijas ir attēlots \ref{fig:tt_curve_toplhcwg_sep18} att..

\begin{figure}[hbtp]

  \centering
  \includegraphics[width=0.6\textwidth]{fig/tt_curve_toplhcwg_sep18.pdf}
  \caption{Virsotnes pāra šķērsgriezums pie dažādas masas centra enerģijas \cite{twiki:tt_curve_toplhcwg_sep18}. Grafikā ir attēlots šķērsgriezums \Pp\Pp un \Pp\Pap sadursmēs, kā arī ir attēloti KMS un ATLAS mērījumi dažādos \ttbar pāra sabrukšanas kanālos.}
  \label{fig:tt_curve_toplhcwg_sep18}
  
\end{figure}

Izmantojot sakarību

\begin{equation}
N=\sigma\int L(t)dt
\end{equation}

pie 35,9\fbinv integrētā spīduma ir sagaidāms, ka radīsies $26,7\times10^{6}$ šādi pāri. 

LHP 2 protoni saduras ar enerģiju, kas ir pietiekami liela, lai ``saspiestu'' protonus tik cieši kopā līdz kvarki vienā protonā spēj mijiedarboties ar kvarkiem otrā protonā. Tie mijiedarbojas, apmainoties ar gluonu. Šādā apmaiņā var rasties virsotnes kvarka-antikvarka pāris. \ref{fig:top_quark_productions} att. attēloti 2 šādi iespējamie procesi. Apmainītais gluons ir tik enerģētisks, ka tas spēj sašķaidīt protonu \gls{gruvešos}. Šāda sadursme tiek saukta par neelastīgu.

\begin{figure}[h!]
  \centering
  \def\twidth{0.3}
  \subfloat[Pāra radīšana.]{%
    \includegraphics[width=\twidth\textwidth]{fig/top_quark_pair_prod_gfusion}%
    \label{fig:top_quark_production}
  }\hfil
  \subfloat[Gluonu saplūšana.]{%
    \includegraphics[width=\twidth\textwidth]{fig/top_quark_pair_prod_gluon}%
    \label{fig:top_quark_production2}
  }
  %% \subfloat[Smago kvarku saplūšana. Punktotā līnija var būt jebkurš elektriski neitrāls bozons.]{
  %%   \includegraphics[width=\twidth\textwidth]{fig/top_quark_pair_prod_qqbar}
  %%   \label{fig:top_quark_production3}
  %% }

  \caption{Virsotnes kvarka pāru radīšana \Pp\Pp sadursmē.}
  \label{fig:top_quark_productions}
\end{figure}

Virsotnes kvarks sabrūk tikai vājajā procesā (\ref{fig:quark_decay} att.). Vājajā sabrukumā tiek izstarots \PW bozons un citas \gls{smaržas} kvarks ar elektriskā lādiņa lielumu  $\frac{1}{3}e$. 

\begin{figure}[H]
  \centering
  \includegraphics[width=0.3\textwidth]{fig/fig_top_quark_decay.pdf}
  \caption{Vājais virsotnes kvarka \cPqt sabrukums. $k$ un $k'$ ir fermioni, kas rodas, sabrūkot \PW bozonam.}
  \label{fig:quark_decay}
\end{figure}

Vidējie CDF, \DZERO Tevatrona eksperimentu mērījumi \cite{Aaltonen:2015cra}, kā arī ATLAS un KMS LHP eksperimentu mērījumi \cite{twiki:tt_curve_toplhcwg_sep18} novērtē Kabibo-Kobajaši-Maskavas matricas $|V_{tb}|$ komponentes vērtību vienādu ar

\begin{equation}
  |V_{tb}|=1,009\pm0,031.
\end{equation}

Tas nozīmē, ka virsotnes kvarks jaucas ar \cPqb kvarku $(0.98)^{2}$ no visiem gadījumiem. Pārēji KKM matricas 3 kolonas un 3 rindas elementi ir ļoti nelieli \cite{Patrignani:2016xqp}:

\begin{align}
  & |V_{td}|=8,4\times10^{-3}, && |V_{ts}|=40,0\times10^{-3}.
\end{align}

Virsotnes kvarka platums saskaņā ar \DZERO kolektīva mērījumu \cite{Abazov:2010tm} ar 2,3 \fbinv integrēto spīdumu ir $\Gamma=1,99^{0.69}_{-0.55}$ GeV. Tas atbilst mūža ilgumam $\tau_{t}=3,3\times10^{-25}\text{s}$.

Šāds mūža ilgums ir mazāks nekā hadronizācijas laiks ($1/\Lambda$ $\sim$ $10^{-24}\text{s}$), kur $\Lambda^{2}$ ir apmainītā gluona $Q^{2}$ enerģija, pie kura stiprā saites koeficienta $\alpha_{s}$ vērtība kļūst vienāda ar $\sim$ 1, kas ir tuvu tā asimptotiskajai vērtībai pie \gls{norobežojuma barjeras}. Līdz ar to virsotnes kvarks sabrūk, pirms tas hadronizējas, un eksperimentētājam paveras vienreizēja iespēja īsu laika sprīdi novērot ``kailu'' kvarku.

Virsotnes kvarka mūža ilgums ir mazāks arī par virsotnes kvarku pāra spina dekorelācijas laika posmu - $M/{\Lambda^{2}}=3\times 10^{-21}\text{s}$. Tātad virsotnes kvarku pāris saglabā savus spina stāvokļus pirms tas sabrūk un nodod savus spina stāvokļus sabrukuma produktiem \cite{Cristinziani:2016vif}.

Virsotnes kvarka sabrukšanas zarojuma attiecības pēc būtības ir tādas pašas kā \PW bozona sabrukuma procesam. \PW bozons sabrūk par leptonu pāriem (\Pe\Pgne, \Pgm\Pgngm, \Pgt\Pgngt) vai kvarku pāriem \cPqu, \cPqd' un \cPqc, \cPqs' (apostrofs norāda uz to, ka smaržas simetrija netiek saglabāta precīzi). Taču kvarku pāriem var būt 3 krāsas. Līdz ar to kopējais satāvokļu skaits ir $3+2\times3=9$. \PW bozona sabrukšanas zarojuma attiecību vienkāršs novērtējums un tā eksperimenta ceļā iegūtie mērījumi ir sniegti \ref{tab:W_br} tab.

\begin{table}[h!]
  \centering
  \caption{\PW bozona sabrukšanas zarojuma attiecības.}
  \label{tab:W_br}
  \begin{tabular}{l r r}
    Zars                  & $\Gamma_{j}/\Gamma$ & $\Gamma_{j}/\Gamma$\\
                          & vienkāršojums       & novērojums \cite{Patrignani:2016xqp}\\
    \hline
    $e\nu_{e}$            & $\frac{1}{9}$       & (10,71 $\pm$ 0,16) \%\\
    $\mu\nu_{\mu}$        & $\frac{1}{9}$       & (10,63 $\pm$ 0,15) \%\\
    $\tau\nu_{\tau}$      & $\frac{1}{9}$       & (11,38 $\pm$ 0,21) \%\\
    kvarku pāris          & $\frac{2}{3}$       & (67,41 $\pm$ 0,27) \%
  \end{tabular}
\end{table}

\PW bozona hadroniskajā sabrukšanas ceļā tiek izstarotas ar krāsām saistītas strūklas (\ref{fig:ttbar_cf} att.). Kvarkiem, kas rada šīs strūklas, ir pretēji vērsti momenti masas centra inerciālajā sistēmā. Kvarkiem attālinoties vienam no otra, to kinētiskā enerģija tiek atdota krāsu laukam. Krāsu laukā esošā papildu enerģija, kas līdzvērtīga apmēram $m_{\PW}$ (80,4 \GeV), tiek izlietota jaunu daļiņu radīšanai. Vienkāršots jaunu hadronu radīšanas process ir attēlots \ref{fig:combination} att., kas balstīts uz Lundas modeli \cite{Andersson:1983ia}. Alternatīvs atainojums, kas balstīts Fainmana diagramās, ir sniegts \ref{fig:colour_field} att.

\begin{figure}[htp]
  \centering
  \includegraphics[width=0.8\textwidth]{fig/combinationlv.pdf}
  \caption{Process, kurā divi enerģētiski kvarki rada jaunus hadronus.}
  \label{fig:combination}
\end{figure}

  \begin{figure}[hbtp]
    \centering
    \includegraphics[width=1.0\textwidth]{fig/colour_field_fulllv.pdf}
    \caption{Hadronu radīšana divu kvarku krāsu laukā.}
    \label{fig:colour_field}

  \end{figure}

Hadroniskajā \PW bozona sabrukšanā rodas šādas daļiņu sugas:

  \begin{table}[h!]

    \centering
    \begin{tabular}{ l l l l }
      \textbf{daļiņa}  & \textbf{masa} [GeV]  & \textbf{mūža ilgums} [s] & \textbf{novērojamais signāls}\\
      \Pgpz              & 135,0               & $8,5\times10^{-27}$  & 2\cPgg absorbēti ECAL\\
      \Pgppm             & 139,6               & $2,6\times10^{-8}$   & \gls{trekeris}, ECAL, HCAL \gls{lietus}\\
      \PKzS              & 497,6               & $8,95\times10^{-11}$ & ECAL, HCAL lietus\\
      \PKzL              & 497,6               & $5,1\times10^{-8}$   & ECAL, HCAL lietus\\
      \PKpm              & 493,7               & $1,2\times10^{-8}$   & trekeris, ECAL, HCAL lietus\\
      \Pn                & 939,6               & $881,5$              & ECAL, HCAL lietus\\
      \Pp                & 938,3               & $\infty$             & trekeris, ECAL, HCAL lietus\\
    \end{tabular}
    \caption{Jaunās daļiņas, kas rodas krāsu laukā starp enerģētiskiem ar krāsam saistītiem kvarkiem, kuri ir izstaroti, hadroniski sabrūkot \PW bozonam.}
    \label{tab:particles}

  \end{table}

  \begin{figure}[hbtp]

    \centering
     \def\twidth{0.45}
    \includegraphics[width=\twidth\textwidth]{fig/ttbar_cf_cropped.pdf}
    \caption{Krāsu plūsmas virsotnes kvarku pāra sabrukuma procesā.}
    \label{fig:ttbar_cf}
    
  \end{figure}

  Attiecīgās rezonanses ir skaidri saskatāmas ģeneratora līmenī (\ref{fig:mass_resonances} att.). Tikai neitrālais pions sabrūk pirms to var tiešā veidā novērot detektorā.

  \begin{figure}[hbtp]
    \centering
     \def\twidth{0.45}
    \includegraphics[width=\twidth\linewidth]{fig/histos/L/gen/charge/allconst/L_JetConst_M_allconst_gen_leading_jet.png}
    \caption{\protect\ref{tab:particles}. tab. norādīto daļiņu rezonanses. \\
    \footnotesize Piezīme: Šis grafiks, kā arī vairāki turpmākie grafiki atbilst KMS pieņemtajam formātam, kā attēlot novērojamo lielumu skaitīšanas eksperimentā. Šī formāta paskaidrojumi ir sniegti \protect\ref{chap:results} nod.}
    \label{fig:mass_resonances}
  \end{figure}

\ref{fig:number} att. ir attēlots daļiņu skaita sadalījums, kas veido vadošo vieglo strūklu. \ref{fig:charged_content} att. ir attēlota attiecība starp elektriski lādēto daļiņu skaitu pret kopējo daļiņu skaitu. Vadošā vieglā strūkla ir tā strūkla, kas izstarota no \PW bozona sabrukuma procesa, kurai ir vislielākais šķērsmoments.

  \begin{figure}[hbtp]
    \centering
     \def\twidth{0.45}
    \includegraphics[width=\twidth\linewidth]{fig/histos/L/reco/charge/allconst/L_JetConst_N_allconst_reco_leading_jet.png}
    \caption{Kopējais daļiņu skaits, kas veido vadošo vieglo strūklu.}
    \label{fig:number}

\end{figure}
% \begin{linenomath}
     \begin{figure}[hbtp]
     \centering
     \def\twidth{0.45}
     \includegraphics[width=\twidth\linewidth]{fig/histos/L/reco/L_JetConst_EventChargedContentN_reco_leading_jet.png}
     \caption{Lādēto daļiņu skaita attiecība pret kopējo daļiņu skaitu, kas veido vadošo vieglo strūklu.}
  \label{fig:charged_content}
   \end{figure}
 % \end{linenomath}

Tā kā mēs pētām viegās strūklas, kas radušās \PW bozona sabrukšanas rezultātā, rodas jautājums, kādēļ strādājam ar \ttbar procesa paraugiem. \PW bozona radīšanas šķērsgriezums ir $>20\times$ lielāks nekā \ttbar šķērsgriezums. Pētījumā mums jāizmanto $\PW\rightarrow \cPq\cPq'$ notikumi, jo leptoniskajos sabrukumos nav krāsu plūsmas. $\PW\rightarrow \cPq\cPq'$ notikumus ar pietiekoši zemu \pt slieksni ir sarežģīti izmantot trigera palaišanai. Tādēļ mēs izmantojam \ttbar notikumus, kur viens no \PW bozoniem sabrūk leptoniski, kas tiek izmantots trigera palaišanai, bet otrs \PW bozons sabrūk hadroniski, kas tiek izmantots krāsu plūsmas pētīšanai.

\PW bozons pieder pie krāsu singleta:

\begin{equation}
\frac{1}{\sqrt{3}}\left(\text{R}\overline{\text{R}}+\text{G}\overline{\text{G}}+\text{B}\overline{\text{B}}\right),
\end{equation}

kur $R$, $G$ un $B$ ir trīs krāsu viļna funkcijas kvantu stāvokļi.

Objets, kas pieder pie krāsu singleta, ir bezkrāsas un tas nevar piedalīties stiprajā mijiedarbībā. Mēs pieminam šo īpašību, jo tālāk aprakstīsim krāsu okteta \PW bozonu.

Tiek pieņemts pie krāsu okteta piederošs \PW bozons. Tā krāsu viļna funkcijas var pieņemt kādu no 8 kombinācjām:

\begin{align}
\text{R}\overline{\text{G}}, &&
\text{R}\overline{\text{B}}, &&
\text{G}\overline{\text{R}}, &&
\text{G}\overline{\text{B}}, &&
\text{B}\overline{\text{R}}, &&
\text{B}\overline{\text{G}}, &&
\frac{1}{\sqrt{2}}\left(\text{R}\overline{\text{R}}-\text{G}\overline{\text{G}}\right), &&
\frac{1}{\sqrt{6}}\left(\text{R}\overline{\text{R}}+\text{G}\overline{\text{G}}-2\text{B}\overline{\text{B}}\right).
\end{align}

Vienīgā dabā novērotā daļiņa, kas pieder pie krāsu okteta ir gluons. Krāsains \PW bozons ir pilnībā hipotētiska daļiņā. Tiek pieņemts, ka krāsu okteta \PW bozona masa ir vienāda ar $m_{\PW}$. Šīs bozons sasaistītu krāsu laukā vieglos kvarkus ar hadronisko \cPqb un hadronisko \cPqt. Tikmēr vieglie kvarki tiktu atsaistīti viens no otra (\ref{fig:ttbar_cf_octet} att.).
  
  \begin{figure}[h!]
  \centering
  \includegraphics[width=0.4\textwidth]{fig/ttbar_cf_flip_cropped.pdf}
  \caption{Krāsu plūsma virsotnes kvarku pāra sabrukuma procesā, pieņemot hipotētisku krāsu okteta \PW bozonu.}
  \label{fig:ttbar_cf_octet}
\end{figure}

