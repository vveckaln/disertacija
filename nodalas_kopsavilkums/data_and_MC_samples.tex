Šīs nodaļas izklāsts ir pielāgots pēc ~\cite{CMS-AN-2017-175} un \cite{CMS-AN-2017-159} parauga, jo šajos pētījumos tiek izmantoti līdzīgi datu un Montekarlo paraugi.

Dati, kas izmantoti šajā pētījumā, tika iegūti 2016{B-H} datu iegūšanas periodos ar kopīgo apstiprināto integrēto spīdumu 35,9~\fbinv visos analizētajos kanālos. Spīdums ir aprēķināts ar \textsc{brilcalc} rīku~\cite{site:brilcalc}, izmantojot šādu komandu:

\begin{lstlisting}[language=sh, breaklines=true,showstringspaces=false]
brilcalc lumi  -b "STABLE BEAMS" --normtag /afs/cern.ch/user/l/lumipro/public/Normtags/normtag_DATACERT.json -i lumiSummary.json
\end{lstlisting}

Visi dati, kas izmantoti šajā pētījumā, ir uzskaitīti \ref{tab:datasets}~tab.  

\begin{table}[htb]
\begin{center}
\caption{Primārie datu kopumi, kas izmantoti šajā analīzē. PD ir saīsinājums vienam mionam vai vienam elektronam~\cite{CMS-AN-2017-159}.}
\label{tab:datasets}
\begin{tabular}{lc}
\hline
Primārais datu kopums              & Integrētais spīdums\\
\hline
/PD/Run2016B-23Sep2016-v3/MINIAOD  & \multirow{8}{*}{35,9 \fbinv}\\
/PD/Run2016C-23Sep2016-v1/MINIAOD  & \\
/PD/Run2016D-23Sep2016-v1/MINIAOD  & \\
/PD/Run2016E-23Sep2016-v1/MINIAOD  & \\
/PD/Run2016F-23Sep2016-v1/MINIAOD  & \\
/PD/Run2016G-23Sep2016-v1/MINIAOD  & \\
/PD/Run2016H-PromptReco-v2/MINIAOD & \\
/PD/Run2016H-PromptReco-v3/MINIAOD & \\\cline{1-2}
\hline
\end{tabular}
\end{center}
\end{table}

Simulēto paraugu saraksts sniegts \ref{tab:mcdatasets}~tab. Tie ir iegūti no

\noindent RunIISummer16MiniAODv2-PUMoriond17\_80X\_mcRun2\_asymptotic\_2016\_TrancheIV\_v6

\noindent izstrādes.

Šķērsgriezumi ir teorētisks paredzējums. Tie tiek iegūti no~\cite{twiki:SingleTopRefXsec} un \cite{twiki:SM13}, izņemot \ttbar, kuram ģeneratora līmeņa šķērsgriezums ir iegūts no~\cite{site:MCM}. Sagaidāmais \ttbar šķērsgriezums otrajā vadošajā pakāpē ir $832^{+20}_{-29}~(\text{mērogs})~\pm35~(\text{PDF}+\alpha_\text{s})$~\cite{twiki:TTbarNLO}. Šo rezultātu izmantojam, lai normalizētu visus \ttbar paraugus.

\begin{longtable}{p{0.16\textwidth}ll}
\caption{Analīzē izmantotie simulāciju paraugi. Sniedzam arī šķērsgriezumu, kas izmantots, lai normalizētu paraugu~\cite{CMS-AN-2017-159}.}\\
\hline
\label{tab:mcdatasets}
Process                      & Datu kopums                                                                 & $\sigma~\text{[pb]}$\\
\hline
\multicolumn{3}{l}{\bf Signāls} \\
\hline
\ttbar                       & \small  TT\_TuneCUETP8M2T4\_13TeV-powheg-pythia8                            & \num{832.0}\\
\hline
\multicolumn{3}{l}{\bf Fons} \\
\hline
\multirow{2}{*}{\ttbar+\PW}  & \small TTWJetsToLNu\_TuneCUETP8M1\_13TeV-amcatnloFXFX-madspin-pythia8       & \num{0.20} \\
                             & \small TTWJetsToQQ\_TuneCUETP8M1\_13TeV-amcatnloFXFX-madspin-pythia8        & \num{0.41} \\\hline
\multirow{2}{*}{\ttbar+\cPZ} & \small TTZToQQ\_TuneCUETP8M1\_13TeV-amcatnlo-pythia                         & \num{0.53} \\
                             & \small TTZToLLNuNu\_M-10\_TuneCUETP8M1\_13TeV-amcatnlo-pythia8              & \num{0.25} \\\hline
\PW\cPZ                      & \small WZTo3LNu\_TuneCUETP8M1\_13TeV-amcatnloFXFX-pythia8                   & \num{5.26} \\\hline
\multirow{2}{*}{\PW\PW}      & \small WWToLNuQQ\_13TeV-powheg                                              & \num{50.0} \\
                             & \small WWTo2L2Nu\_13TeV-powheg                                              & \num{12.2} \\\hline
\multirow{2}{*}{\cPZ\cPZ}    & \small ZZTo2L2Nu\_13TeV\_powheg\_pythia8                                    & \num{0.564} \\
                             & \small ZZTo2L2Q\_13TeV\_amcatnloFXFX\_madspin\_pythia8                      & \num{3.22} \\\hline
\multirow{3}{*}{\PW + strūklas}& \small WToLNu\_0J\_13TeV-amcatnloFXFX-pythia8                               & \num{49540.0} \\
                             & \small WToLNu\_1J\_13TeV-amcatnloFXFX-pythia8                               & \num{8041.0} \\
                             & \small WToLNu\_2J\_13TeV-amcatnloFXFX-pythia8                               & \num{3052.0} \\\hline
\multirow{2}{*}{Drell-Yan}   & \small DYJetsToLL\_M-10to50\_TuneCUETP8M1\_13TeV-madgraphMLM-pythia8        & \num{18610.0} \\
                             & \small DYJetsToLL\_M-50\_TuneCUETP8M1\_13TeV-madgraphMLM-pythia8            & \num{6025.0} \\\hline
\multirow{10}{=}{Ar $\mu$ bagātinātie KHD}
                             & \small QCD\_Pt-30to50\_MuEnrichedPt5\_TuneCUETP8M1\_13TeV\_pythia8          & \num{1652471.46}\\ 
                             & \small QCD\_Pt-50to80\_MuEnrichedPt5\_TuneCUETP8M1\_13TeV\_pythia8          & \num{437504.1} \\
                             & \small QCD\_Pt-80to120\_MuEnrichedPt5\_TuneCUETP8M1\_13TeV\_pythia8         & \num{106033.66}\\
                             & \small QCD\_Pt-120to170\_MuEnrichedPt5\_TuneCUETP8M1\_13TeV\_pythia8        & \num{25190.52}\\
                             & \small QCD\_Pt-170to300\_MuEnrichedPt5\_TuneCUETP8M1\_13TeV\_pythia8        & \num{8654.49}\\
                             & \small QCD\_Pt-300to470\_MuEnrichedPt5\_TuneCUETP8M1\_13TeV\_pythia8        & \num{797.35}\\
                             & \small QCD\_Pt-470to600\_MuEnrichedPt5\_TuneCUETP8M1\_13TeV\_pythia8        & \num{45.83}\\
                             & \small QCD\_Pt-600to800\_MuEnrichedPt5\_TuneCUETP8M1\_13TeV\_pythia8        & \num{25.1}\\
                             & \small QCD\_Pt-800to1000\_MuEnrichedPt5\_TuneCUETP8M1\_13TeV\_pythia8       & \num{4.71}\\
                             & \small QCD\_Pt-1000toInf\_MuEnrichedPt5\_TuneCUETP8M1\_13TeV\_pythia8       & \num{1.62}\\\hline
\multirow{6}{=}{Ar $e$ bagātinātie KHD}
                             & \small QCD\_Pt-30to50\_EMEnriched\_TuneCUETP8M1\_13TeV\_pythia8             & \num{6493800.0}\\
                             & \small QCD\_Pt-50to80\_EMEnriched\_TuneCUETP8M1\_13TeV\_pythia8             & \num{2025400.0}\\
                             & \small QCD\_Pt-80to120\_EMEnriched\_TuneCUETP8M1\_13TeV\_pythia8            & \num{478520.0}\\
                             & \small QCD\_Pt-120to170\_EMEnriched\_TuneCUETP8M1\_13TeV\_pythia8           & \num{68592.0}\\
                             & \small QCD\_Pt-170to300\_EMEnriched\_TuneCUETP8M1\_13TeV\_pythia8           & \num{18810.0}\\
                             & \small QCD\_Pt-300toInf\_EMEnriched\_TuneCUETP8M1\_13TeV\_pythia8           & \num{1350.0}\\

\hline
\end{longtable}

Krāsu okteta paraugs ir uzskaitīts \ref{tab:mcdatasets_flip}~tab. 

\begin{table}[H]
\begin{center}
\caption{Simulācijas paraugi krāsu okteta \PW bozonam. Sniedzam šķērsgriezumu, kas izmantots, lai normalizētu paraugus šajā analīzē.}
\label{tab:mcdatasets_flip}
\hspace*{-0.5cm}
\begin{tabular}{llc}
\hline
Process & Datu kopums & $\sigma[pb]$\\
\multicolumn{3}{l}{\bf Signāls} \\
\hline
Krāsu okteta \PW bozons &  {\small TT\_TuneCUETP8M2T4\_13TeV-powheg-colourFlip-pythia8} & 832,0 \\
\hline
\end{tabular}
\end{center}
\end{table}

Balstoties uz atšķirībām starp datiem un simulētajiem notikumiem, tiek veiktas šādas korekcijas simulācijai:

\begin{enumerate}
\item sagrūduma pārsvēršana,
\item leptona identifikācijas un izolācijas efektivitātes korekcija,
\item trigeru efektivitātes korekcija,
\item ģeneratora līmeņa svara piešķiršana,
\item strūklu enerģijas mēroga un izšķirtspējas korekcijas,
\item \cPqb atzīmēšanas efektivitātes korekcija.
\end{enumerate}

