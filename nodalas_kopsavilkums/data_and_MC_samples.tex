Dati, kas izmantoti šajā pētījumā, tika iegūti 2016{B-H} datu iegūšanas periodos ar kopīgo apstiprināto integrēto spīdumu 35,9 \fbinv visos analizētajos kanālos. Spīdums ir aprēķināts ar \textsc{brilcalc} rīku \cite{site:brilcalc}, izmantojot sekojošu komandu:

\begin{lstlisting}[language=sh, breaklines=true]
brilcalc lumi  -b "STABLE BEAMS" --normtag /afs/cern.ch/user/l/lumipro/public/Normtags/normtag_DATACERT.json -i lumiSummary.json
\end{lstlisting}

Visi dati, kas izmantoti šajā pētījumā, ir uzskaitīti \ref{tab:datasets} tabulā.  

\begin{table}[htb]
\begin{center}
\caption{Primārie datu kopumi, kas izmantoti šajā analīzē. PD ir saīsinājums vienam mionam vai vienam elektronam.}
\label{tab:datasets}
\begin{tabular}{lc}
\hline
Primārais datu kopums              & Integrētais spīdums\\
\hline
/PD/Run2016B-23Sep2016-v3/MINIAOD  & \multirow{8}{*}{35,9 \fbinv}\\
/PD/Run2016C-23Sep2016-v1/MINIAOD  & \\
/PD/Run2016D-23Sep2016-v1/MINIAOD  & \\
/PD/Run2016E-23Sep2016-v1/MINIAOD  & \\
/PD/Run2016F-23Sep2016-v1/MINIAOD  & \\
/PD/Run2016G-23Sep2016-v1/MINIAOD  & \\
/PD/Run2016H-PromptReco-v2/MINIAOD & \\
/PD/Run2016H-PromptReco-v3/MINIAOD & \\\cline{1-2}
\hline
\end{tabular}
\end{center}
\end{table}

Simulēto paraugu saraksts sniegts \ref{tab:mcdatasets} tabulā. Tie ir iegūti no

RunIISummer16MiniAODv2-PUMoriond17\_80X\_mcRun2\_asymptotic\_2016\_TrancheIV\_v6

izstrādes. Šķērsgriezumi ir teorētisks paredzējums. Tie tiek iegūti no \cite{twiki:SingleTopRefXsec} un \cite{twiki:SM13}, izņemot \ttbar, kuram ģeneratora līmeņa šķērsgriezums ir iegūts no \cite{site:MCM}. Sagaidāmais \ttbar šķērsgriezums otrajā vadošajā pakāpē ir $832^{ +20}_{-29}~(\text{scale})~\pm 35~(\text{PDF}+\alpha_S)$~\cite{twiki:TTbarNLO}. Šo rezultātu izmantojam, lai normalizētu visus \ttbar paraugus.

\begin{table}
\caption{Analīzē izmantotie simulāciju paraugi. Sniedzam arī šķērsgriezumu, kas izmantots, lai normalizētu paraugu.}
\label{tab:mcdatasets}
\begin{longtable}{ p{0.16\textwidth}ll }
\hline
Process                      & Datu kopums                                                                 & $\sigma[pb]$\\
\hline
\multicolumn{3}{l}{\bf Signāls} \\
\hline
\ttbar                       & \small  TT\_TuneCUETP8M2T4\_13TeV-powheg-pythia8                            & 832,0\\
\hline
\multicolumn{3}{l}{\bf Fons} \\
\hline
\multirow{2}{*}{\ttbar+\PW}  & \small TTWJetsToLNu\_TuneCUETP8M1\_13TeV-amcatnloFXFX-madspin-pythia8       & 0,20 \\
                             & \small TTWJetsToQQ\_TuneCUETP8M1\_13TeV-amcatnloFXFX-madspin-pythia8        & 0,41 \\\hline
\multirow{2}{*}{\ttbar+\cPZ} & \small TTZToQQ\_TuneCUETP8M1\_13TeV-amcatnlo-pythia                         & 0,53 \\
                             & \small TTZToLLNuNu\_M-10\_TuneCUETP8M1\_13TeV-amcatnlo-pythia8              & 0,25 \\\hline
\PW\cPZ                      & \small WZTo3LNu\_TuneCUETP8M1\_13TeV-amcatnloFXFX-pythia8                   & 5,26 \\\hline
\multirow{2}{*}{\PW\PW}      & \small WWToLNuQQ\_13TeV-powheg                                              & 50,0 \\
                             & \small WWTo2L2Nu\_13TeV-powheg                                              & 12,2 \\\hline
\multirow{2}{*}{\cPZ\cPZ}    & \small ZZTo2L2Nu\_13TeV\_powheg\_pythia8                                    & 0,564 \\
                             & \small ZZTo2L2Q\_13TeV\_amcatnloFXFX\_madspin\_pythia8                      & 3,22 \\\hline
\multirow{3}{*}{\PW+strūklas}& \small WToLNu\_0J\_13TeV-amcatnloFXFX-pythia8                               & 49540,0 \\
                             & \small WToLNu\_1J\_13TeV-amcatnloFXFX-pythia8                               & 8041,0 \\
                             & \small WToLNu\_2J\_13TeV-amcatnloFXFX-pythia8                               & 3052,0 \\\hline
\multirow{2}{*}{Drell-Yan}   & \small DYJetsToLL\_M-10to50\_TuneCUETP8M1\_13TeV-madgraphMLM-pythia8        & 18610,0 \\
                             & \small DYJetsToLL\_M-50\_TuneCUETP8M1\_13TeV-madgraphMLM-pythia8            & 6025, \\\hline
\multirow{10}{=}{QCD ar $\mu$ bagātinātie}
                             & \small QCD\_Pt-30to50\_MuEnrichedPt5\_TuneCUETP8M1\_13TeV\_pythia8          & 1652471,46\\ 
                             & \small QCD\_Pt-50to80\_MuEnrichedPt5\_TuneCUETP8M1\_13TeV\_pythia8          & 437504,1\\
                             & \small QCD\_Pt-80to120\_MuEnrichedPt5\_TuneCUETP8M1\_13TeV\_pythia8         & 106033,66\\
                             & \small QCD\_Pt-120to170\_MuEnrichedPt5\_TuneCUETP8M1\_13TeV\_pythia8        & 25190,52\\
                             & \small QCD\_Pt-170to300\_MuEnrichedPt5\_TuneCUETP8M1\_13TeV\_pythia8        & 8654,49\\
                             & \small QCD\_Pt-300to470\_MuEnrichedPt5\_TuneCUETP8M1\_13TeV\_pythia8        & 797,35\\
                             & \small QCD\_Pt-470to600\_MuEnrichedPt5\_TuneCUETP8M1\_13TeV\_pythia8        & 45,83\\
                             & \small QCD\_Pt-600to800\_MuEnrichedPt5\_TuneCUETP8M1\_13TeV\_pythia8        & 25,1\\
                             & \small QCD\_Pt-800to1000\_MuEnrichedPt5\_TuneCUETP8M1\_13TeV\_pythia8       & 4,71\\
                             & \small QCD\_Pt-1000toInf\_MuEnrichedPt5\_TuneCUETP8M1\_13TeV\_pythia8       & 1,62\\\hline
\multirow{6}{=}{QCD ar $e$ bagātinātie}
                             & \small QCD\_Pt-30to50\_EMEnriched\_TuneCUETP8M1\_13TeV\_pythia8             & 6493800,0\\
                             & \small QCD\_Pt-50to80\_EMEnriched\_TuneCUETP8M1\_13TeV\_pythia8             & 2025400,0\\
                             & \small QCD\_Pt-80to120\_EMEnriched\_TuneCUETP8M1\_13TeV\_pythia8            & 478520,0\\
                             & \small QCD\_Pt-120to170\_EMEnriched\_TuneCUETP8M1\_13TeV\_pythia8           & 68592,0\\
                             & \small QCD\_Pt-170to300\_EMEnriched\_TuneCUETP8M1\_13TeV\_pythia8           & 18810,0\\
                             & \small QCD\_Pt-300toInf\_EMEnriched\_TuneCUETP8M1\_13TeV\_pythia8           & 1350,0\\

\hline
\end{longtable}
\end{table}

Krāsu okteta paraugs ir uzskaitīts \ref{tab:mcdatasets_flip} tab. 

\begin{table}[htb]
\begin{center}
\caption{Simulācijas paraugi krāsu okteta \PW bozonam. Sniedzam šķērsgriezumu, kas izmantots, lai normalizētu paraugus šajā analīzē.}
\label{tab:mcdatasets_flip}
\hspace*{-0.5cm}
\begin{tabular}{ llc }
\hline
Process & Datu kopums & $\sigma[pb]$\\
\multicolumn{3}{l}{\bf Signāls} \\
\hline
Krāsu okteta \PW bozons &  {\small TT\_TuneCUETP8M2T4\_13TeV-powheg-colourFlip-pythia8} & 832,0 \\
\hline
\end{tabular}
\end{center}
\end{table}

Balstoties uz atšķirībām starp datiem un simulētajiem notikumiem, tiek veiktas šādas korekcijas simulācijai.

\begin{enumerate}
\item Sagrūduma pārsvēršana
\item Leptona identifikācijas un izolācijas efektivitātes korekcija
\item Trigeru efektivitātes korekcija
\item Ģeneratora līmeņa svara piešķiršana
\item Strūklu enerģijas mēroga un izšķirtspējas korekcijas
\item \cPqb-atzīmēšanas efektivitātes korekcija
\end{enumerate}

