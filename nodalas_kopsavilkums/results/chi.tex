\label{sec:LEP_methodology}

Tiek analizētas trīs veidu plūsmas:
\begin{itemize}
\item daļiņu plūsmā visiem notikumiem tiek piešķirsts svars, vienāds ar 1,0.
\item enerģijas plūsmā daļiņām tiek piešķirts svars, kas vienāds ar to enerģijas attiecību pret virsotnes kvarku enerģijas summu.
\item \pt plūsmā daļiņām tiek piešķirts svars, kas vienāds ar to šķērsmomentu attiecību pret attiecīgās strūklas šķērsmomentu.
\end{itemize}

Rezultāti, izmantojot LEP metodoloģiju, iekļaujot visas strūklas sastāvdaļās, ir attēloti \ref{fig:chi_allconst_N} att. Tiek attēlotas plūsmas starp vadošo \cPqb strūklu \leadingb un otru vadošo \cPqb strūklu \scndleadingb, hadronisko \cPqb strūklu $j_{h}^{\cPqb}$ un tālāko vieglo strūklu $j_{f}^{\PW}$ (attālums starp strūklām ir mērīts kā leņķis starp strūklu 4-vektoru telpiskajām komponentēm), tuvāko vieglo strūklu $j_{c}^{\PW}$ un hadroniskā \cPqb strūklu $j_{h}^{\cPqb}$, un vadošo vieglo strūklu \leadingjet un otro vadošo vieglo strūklu \scndleadingjet.

Visos gadījumos blīvums samazinās centrālajā reģionā starp strūklām. Blīvums centrālajā regionā atkarīgs no tā, vai strūklas ir vai nav saistītas ar krāsām.

Plūsmas attiecība no krāsām brīvajos reģionos (\leadingb, \scndleadingb), ($j_{h}^{\cPqb}$, $j_{f}^{\PW}$), ($j_{c}^{\PW}$, $j_{h}^{\cPqb}$) pret plūsmu ar krāsam saistītajā reģionā (\leadingjet, \scndleadingjet), iekļaujot visas strūklu sastāvdaļas, ir attēlota \ref{fig:chirg_allconst_N} att. Krāsu okteta \PW modeļa gadījumā, novērojama atkārtota krāsus saistība reģionā ($j_{c}^{\PW}$, $j_{h}^{\cPqb}$).

\ref{fig:ratio_hbqc} att. attēlota daļiņu plūsmas attiecība (\leadingjet, \scndleadingjet) reģionā atbilstoši krāsu okteta \PW modelim pret daļiņu plūsmu (\leadingjet, \scndleadingjet) reģionā atbilstoši Standarta modelim. Redzams, ka \PW okteta modeļa gadījumā novērojama krāsu saistības izzušana.

Kā kvantitatīvu LEP metodoloģijas rezultātu var izmantot parametru $R$, kas tiek definēts kā attiecība starp integrāli no 0,2 līdz 0,8 ar krāsām saistītajā reģionā pret integrāli no 0,2 līdz 0,8 ar krāsām nesaistītā regionā:

\begin{equation}
R=\frac{\int_{0.2}^{0.8}f^{\text{inter \PW reģions}}d\chi}{\int_{0.2}^{0.8}f^{\text{intra \PW reģions}}d\chi},
\end{equation}

kur $f(\chi)$ ir plūsmas sadalījuma blīvums.

Šis parametrs LEP tika izmantots, lai kvantificētu krāsu saistību. Tā vērtības, kas iegūtas dažādos eksperimentos, izmantojot 625 \pbinv integrēto spīdumu intervālā \sqrts=189-209 \GeV, ir sniegtas \ref{tab:LEP_R} tab. Vērojama dažādos eksperimentos novēroto $R$ vērtību nesakritība. Turklāt, balstoties uz teorētiskiem apsvērumiem, $R$ būtu jāpārsniedz 1.

\begin{table}
\centering
\caption{LEP novērotās $R$ vērtības.}
\label{tab:LEP_R}
\begin{tabular}{lll}
LEP eksperiments & $R$ vērtība                                             & atsauce\\
\hline
    OPAL         & 1,243                                                   & \cite{Abbiendi:2005es}\\
    Delphi       & 0,889 ($\sqrt{s}=183$ GeV) - 1,039 ($\sqrt{s}=207$ GeV) & \cite{Abdallah:2006uq}\\
    L3           & 0,911                                                   & \cite{Achard:2003pe}\\
  \end{tabular}
\end{table} 

Mūsu gadījumā lietojam 3 $R$ vērtības, kas atbilst 3 ar krāsām nesaistītajiem reģioniem.

Integrālis no 0,2 līdz 0,8 dažādos reģionos un inversas $R$ vērtības Standarta modelim ir sniegtas \ref{tab:R_L_reco_N_MC_SM} tab., datiem \ref{tab:R_L_reco_N_data_SM} tab. un krāsu okteta \PW modelim \ref{tab:R_L_reco_N_MC_cflip} tab.

\figureChilv{allconst}{N}{}{Daļiņu plūsmas histogramas, iekļaujot visas strūklu daļiņas.}

\figureratiographslv{allconst}{N}{Daļiņu plūsma, iekļaujot visas strūklu daļiņas, attiecībā pret daļiņu plūsmu $\leadingjet, \scndleadingjet$ apgabalā.}

\begin{figure}[htpb]
\def\twidth{0.45}
\centering
\includegraphics[width=\twidth\textwidth]{fig/ratiographs_merged_SM/L_hbqc_N_allconst_reco.png}
\caption{Attiecība starp daļiņu plūsmu (\leadingjet, \scndleadingjet) reģionā krāsu okteta \PW modelim pret daļiņu plūsmu (\leadingjet, \scndleadingjet) reģionā atbilstoši Standarta modelim.}
\label{fig:ratio_hbqc}OA
\end{figure}

\input{tables/Rvalues_SMlv/R_L_reco_MC_N_SM.txt}

\input{tables/Rvalues_SMlv/R_L_reco_data_N_SM.txt}

\input{tables/Rvalues_cfliplv/R_L_reco_MC_N_cflip.txt}


