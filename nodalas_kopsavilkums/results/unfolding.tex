Kad eksperimentētājs veic novērojumu ar detektoru, jāsamierinās ar paša detektora ietekmi uz rezultātu \textendash\ kropļojumiem. \gls{Atlocīšana} ir metode, ar kuras palīdzību detektora novērojami tiek koriģēti, ievērojot detektora radīto ietekmi. Tādējādi mēs iegūstam patieso novērojamā lieluma sadalījumu. Tomēr jārēķinās, ka atlocītā novērojamā lieluma fāzes telpa kļūst ievērojami graudaināka. 

Detektora ietekmi mēs varam novērtēt tādēļ, ka, strādājot ar Montekarlo paraugiem, katrs ģenerētais notikums tiek rekonstruēts, izmantojot detektora simulāciju. Rekonstrukcijas procesā novērojamais lielums vērtību intervālā $i$ ģenerācijas līmenī migrē uz vertību intervālu $k$ rekonstrukcijas līmenī. Apskatot lielu skaitu notikumu, iegūstam migrācijas statistiku. Atlocīšanas ceļā veicam šai migrācijai pretēju procesu - ja dots novērojamais lielums vērtību intervālā $k$, mēs piešķiram varbūtības dažādām novērojamā lieluma patiesajām vērtībām.

Tās \pullangle vērtības ģenerēšanas līmenī, kurām nav atbilstošas vērtības rekonstrukcijas līmenī, tiek ievietotas \gls{pirmsintervālā} rekonstrukcijas līmenī. Tās \pullangle vērtības rekonstrukcijas līmenī, kurām nav atbilstošas vērtības ģenerācijas līmenī, tiek ievietotas pirmsintervālā ģenerācijas līmenī. Pirmsintervāls ģenerācijas līmenī tiek uzskatīts par fonu, un tas tiek iztukšots. Sadalījumi, kas netiek pildīti ģenerācijas līmenī \textendash\ dati un MK fons \testendash\ tiek samazināti ar atbilstošu proporcijas koeficientu. Pirmsintervāls rekonstrukcijas līmenī tiek izmantots, lai ierobežotu atlocītā rezultāta pirmsintervālu. 

Atlocīšana tiek veikta ar datiem, no kuriem ir atņemts MK fons. Atlocīšana tika veikta arī prerēji, iegūstot atpakaļatlocīto rezultātu. 

Lai samazinātu atlocītā rezultāta satistisko nenoteiktību, esam ieinteresēti, lai migrācijas matrica būtu pēc iespējas diagonāla. Lai raksturotu notikumu skaitu, kas sakopoti uz migrācijas matricas diagonāles, tiek lietoti divi lielumi: stabilitāte un tīrība. Stabilitāte ir attiecība starp diagonāles saturu un kopējo notikumu skaitu rekonstrukcijas līmenī vērtību intervālā:

\begin{equation}
  \text{stabilitāte}\equiv\frac{\theta^{\text{diag}}_{\text{ievade}}}{\Sigma_{x=1}^{x=N_{x}}\theta^{x}_{\text{ievade}}},
\end{equation}

\noindent kur $x$ ir vērtību intervāla indekss rekonstrukcijas līmenī, numerāciju sākot no 1, un $N_{x}$ ir vērtību intervālu skaits rekonstrukcijas līmenī. Tīrība ir attiecība starp diagonāles saturu un kopējo notikumu skaitu ģenerācijas līmenī:

\begin{equation}
  \text{tīrība}\equiv\frac{\theta^{\text{diag}}_{\text{ievade}}}{\Sigma_{y=1}^{y=N_{y}}\theta^{y}_{\text{ievade}}},
\end{equation}

\noindent kur $y$ ir vērtību intervāla indekss ģenerācijas līmenī. Ir ieteicams, lai tīrības un stabilitātes vertības pārsniegtu 50~\% katrā vērtību intervālā.

Apskatām arī interesantu rādītāju, kas raksturo, par cik atlocītais rezultāts ir atšķirīgs no ģenerētā MK rezultāta (ideāls rādītajs būtu 0), normalizētu pret atlocīta rezultāta statistisko nenoteiktību. Šo rādītāju sauc par vilkmi:

\begin{equation}
  \text{vilkme}\equiv\frac{\theta^{\text{ģen}}_{\text{atl}}-\theta^{\text{ģen}}_{\text{ievade}}}{\sigma^{\text{ģen}}_{\text{atl}}}.
\end{equation}

Lai iegūtu vilkmes sadalījumu, ģenerējam novērojamā lieluma gadījuma spēļu sadalījumus ģenerācijas līmenī.

Vērtību intervālu skaits ģenerāciju līmenī tiek samazināts divas reizes, salīdzinot ar intervālu skaitu rekonstrukcijas līmenī, lai atlocīšana būtu skaitliski iespējama.

Atlocīšanas procesa īstenošanai tiek izmantota \ROOT klase \lstinline[language=sh]|TUnfoldDensity|~\cite{Schmitt:2012kp}. Intervālu shēma tiek pārvaldīta ar \lstinline[language=sh]|TUnfoldBinning| klasi. Netiek lietota regularizācija. \pullangle no \leadingjet uz \scndleadingjet atlocīšanas rezultāti, iekļaujot visas strūklas sastāvdaļas, ir atainoti \ref{fig:unfolding_nominal_leading_jet_allconst_pull_angle_OPT_MC13TeV_TTJets_ATLAS3}~att. Lai radītu šos attēlus, tika izveidota jauna klase \lstinline[language=sh]|CompoundHistoUnfolding|~\cite{url:compoundhistounfolding}, kas tika pievienota \ROOT, t.sk. ar ievades un izvades straumētājiem.

Atlocīšanas rezultātiem ir uzklāti sadalījumi, kas atbilst atlocīšas rezultātiem, kas iegūti ar migrācijas matricām, kas savukārt iegūtas no $\ttbar\ Herwig++$ un $\ttbar\ cflip$ paraugiem, kā arī sistemātikas $\ttbar\ fsr\ dn$ un $\ttbar\ fsr\ up$ (skat. \ref{chap:systematic_uncertainties}~nod.).

Statistisko \gls{traucējumu} nozīmība katrā vērtību intervālā kopējā atlocītā rezultāta sistemātiskajā kļūdā  (\%) ir norādīta \ref{tab:unc_table_fullpull_angle_OPT_allconst_gen_out_MC13TeV_TTJets_nominal_ATLAS3}~tab. Traucējumi, kas tieši ietekmē hadronizāciju - $\ttbar\ Herwig++$, $\ttbar\ QCDbased$ un $\ttbar\ ERDon$ ir visnozīmīgākie.

Tiek novērtēta arī sakritība starp atlocīto rezultātu un MK paredzējumu ģeneratora līmenī. Ja dots atlocīts novērojamais lielums $D$, normalizētais MK paredzējums $M$, pilnā eksperimentālo nenoteiktību kovariances matrica $\Sigma$, $\chi^{2}$ tiek aprēķināts šādi:

\begin{equation}
  \chi^{2}=(D^{T}-M^{T})\cdot\Sigma^{-1}\cdot(D-M).
  \label{eq:chi2}
\end{equation}

No $\chi^{2}$ vērtības iespējams aprēķināt $p$-vērtību, zinot, ka brīvību skaits ir vienāds ar atlocītā sadalījuma vertības intervālu skaitu, no kura atņemts 1, ņemot vēro brīvības zaudējumu, veicot sadalījuma normalizāciju. No kovariances matricas tiek atmesta viena rinda un viena kolonna $\Sigma$. Atmetamo elementu izvēle neietekmē $\chi^{2}$ vērtību.

\ref{tab:chi_table_pull_angle_OPT_allconst_nominal_ATLAS3}~tab. sniegtas \pullangle $\chi^{2}$ vērtības un $p$-vērtības, ja ietvertas visas strūklas sastāvdaļas. No rezultātiem redzams, ka MK ģeneratori diezgan neprecīzi modelē vilkmes leņķa sadalījumu. Simulācijas, vispārīgi skatot, paredz stāvāku vilkmes leņķa sadalījumu, t. i. izteiktāku krāsu plūsmas efektu. \HERWIGpp vilkmes leņķa sadalījumu modelē labāk nekā \PYTHIA 8.2. \PYTHIA 8.2 precizitāte ir jo īpaši vāja, paredzot vilkmes leņķa sadalījumu no \scndleadingjet uz \leadingjet.

$\chi^{2}$ vērtības un $p$-vērtības krāsu okteta \PW bozona modelim ir sniegtas \ref{tab:chi_table_pull_angle_OPT_allconst_cflip_ATLAS3}~tab. \POWHEG+\PYTHIA 8 * ailē $\ttbar\ cflip$ paragus ir ticis pievienots kā \ttbar sistemātiskā nenoteiktīb. Krāsu apmaiņas modelī \pullangle no \leadingjet uz \scndleadingjet sadalījums ir modelēts neprecīzāk nekā SM paredzējums. 
  
%% \ref{tab:chi_table_pull_angle_OPT_allconst_MC13TeV_TTJets_nominal_ATLAS3_full} tab. sniegtas $\chi^{2}$ vērtības, ja signāls $M$ \ref{eq:chi2} izteiksmē tiek aizvietots ar attiecīgo sistemātiku, bet saglabājot kovariances matricu $\Sigma$ nemainīgu. Sakritība ir labāka nekā \ttbar, ja krāsu plūsma tiek modelēta ar \ttbar ERDOn, \ttbar Herwig++ vai \ttbar QCD based.
\figunfoldinglv{nominal}{leading_jet}{allconst}{pull_angle}{ATLAS3}{MC13TeV_TTJets}{Vilkmes leņķa no \leadingjet uz \scndleadingjet atlocīšanas rezultāti \ttbar metodei, iekļaujot visas strūklas sastāvdaļas un izmantojot trīs vienāda izmēra vērtību intervālus}

%\Needspace{6\baselineskip}
%\newpage
\input{tables/unc_nominal_fulllv/pull_angle/ATLAS3/unc_table_full_leading_jet_allconst_pull_angle_OPT_gen_out_MC13TeV_TTJets.txt}
\input{tables/unc_cflip_fulllv/pull_angle/ATLAS3/unc_table_full_leading_jet_allconst_pull_angle_OPT_gen_out_MC13TeV_TTJets.txt}
\input{tables/chi_nominallv/pull_angle/ATLAS3/chi_table_pull_angle_OPT_allconst.txt}
\input{tables/chi_cfliplv/pull_angle/ATLAS3/chi_table_pull_angle_OPT_allconst.txt}

%\input{tables/chi_nominallv/pvmag/ATLAS3/chi_table_pvmag_OPT_allconst.txt}
