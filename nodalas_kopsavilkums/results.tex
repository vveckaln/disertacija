\label{chap:results}
\section{Notikumu attēls}
\ref{fig:event_display}~att. redzams notikums, kuru veido vieglās strūklas, \cPqb kvarku strūklas un leptons $\eta-\phi$ plaknē. Attēlots arī vilkmes vektors. Attēls veidots līdzīgi kā \ref{fig:pull_angle}~att.

\begin{figure}[hbtp]
  \centering
  \includegraphics[width=1.0\textwidth]{fig/individual_plots/reco_allconst_total_1111_DeltaR_2p846131_pull_angle_1p964620.png}
  \caption{Vadošās strūklas vilmes vektors (pārtraukta līnija ar punktiem), kas veido 1,96~rad vilkmes leņķi ar vektoriālo starpību (pārtraukta līnija) starp otro vadošo vieglo strūklu un vadošo vieglo strūklu. Vadošās strūklas sastāvdaļas ir attēlotas zilā krāsā, bet otrās vadošās strūklas sastāvdaļas ir attēlotas sarkanā krāsā. Vadošā strūkla ir attēlota ar nepārtrauktu līniju, bet otrā vadošā strūkla ir attēlota ar pārtrauktu līniju. Hadroniskā \cPqb~kvarka strūkla un tās sastāvdaļas attēlotas zaļā krāsā, bet leptoniskā \cPqb~kvarka strūkla un tās sastāvdaļas attēlotas violetā krāsā. Ar ``$\times$'' atzīmēts lādētais leptons. Vilkmes vektors ir palielināts~200 reizes, bet apļu, kas attēlo strūklas, rādiuss ir vienāds ar $\nicefrac{\pt~[\GeV]}{75,0}$, savukārta apļu, kas attēlo strūklu sastāvdaļas, rādiuss ir vienāds ar $\nicefrac{p^{\text{constituent}}_{\text{T}}}{p^{\text{jet}}_{\text{T}}}$.}
  \label{fig:event_display}
\end{figure}

\section{Vilkmes vektors}

Pētījuma veikšanai tika izstrādāts rīku kopums \textsc{CFAT}~\cite{url:cfat}. Galvenā pētījuma daļa tika veikta, izmantojot \CMSSW versiju \lstinline[language=sh]|CMSSW_8_0_26_patch1|. Grafiki ir veidoti, izmantojot \ROOT~\cite{Brun}. Vilkmes vektori tiek noteikti visām novērojamajām strūklām \textendash\ vadošajai vieglajai strūklai \leadingjet (ar vislielāko \pt), otrajai vadošajai vieglajai strūklai \scndleadingjet, vadošajai hadroniskā \cPqb strūklai \leadingb un otrajai vadošajai hadroniskā \cPqb strūklai \scndleadingb. Visos gadījumos, tiek izdalīti apakšgadījumi atkarībā no tā, vai vilkmes vektora aprēķinā iekļautas visas strūklas sastāvdaļas vai tikai elektriski lādētās sastāvdaļas. Rezultāti ir sadalīti kanālos: $e$ + strūklas, $\mu$ + strūklas un kopīgajā $\ell$ + strūklas kanālā.

Vadošās vieglās strūklas \leadingjet $\eta$ dimensijas sadalījums vilkmes vektoram ar visām strūklas sastāvdaļām ir redzams \ref{fig:_eta_PV_allconst_reco_leading_jet}~att.

Vietā ir paskaidrojums par KMS grafiku formātu. Augšējais grafiks \ref{fig:_eta_PV_allconst_reco_leading_jet}~att. vizualizē datus un Montekarlo simulācijas. Ja nav norādīts citādi, Montekarlo simulācijas ir attēlotas rekonstrukcijas līmenī.

Ja dota sistemātskā nenoteiktība ar indeksu $k$ mēs to identificējam kā pozitīvu sistemātiku $U^{k}_{i}$, ja vērtības intervālā $i$ sistemātika $S^{k}_i$ pārsniedz nominālo vērtību $N_{i}$. Pretējā gadījuma sistemātiskā nenoteiktība tiek noteikta kā negatīva sistemātikā nenoteiktība $D^{k}_{i}$. Kopējā pozitīvā un negatīvā sistemātiskā nenoteiktība tiek noteikta kā kvadrātu summa:

\begin{align}
U_{i}=\sqrt{\sum_{k}\left(U^{k}_{i}-N_{i}\right)^{2}}, && D_{i}=\sqrt{\sum_{k}\left(D^{k}_{i}-N_{i}\right)^{2}}.
\end{align}

Zilās joslas platums atbilst sistemātiskajai kļūdai, kas noteikta kā $\nicefrac{(U_{i}+D_{i})}{2}$. Joslas centrs atbilst $N_{i}+\nicefrac{(U_{i}-D_{i})}{2}$. Teiktais attiecas arī uz rozā joslu ar atšķirību, ka sistemātikas ir tikušas normalizētas ar signāla integrāli (šādas normalizētas histogramas tiek sauktas par \gls{veidoliem}). Apakšējā ielikumā attēlota datu attiecība pret Montekarlo, kā arī sistemātikas, kas normalizētas attiecībā pret Montekarlo.

\figureEMLlv{/reco/PV/charge/allconst/}
          {_eta_PV_allconst_reco_leading_jet}
          {\leadingjet vilkmes vektora $\eta$ dimensijas sadalījums, iekļaujot visas strūklas sastāvdaļas.}

\leadingjet vilkmes vektora $\phi$ dimensijas sadalījums, iekļaujot visas strūklas sastāvdaļas ir, attēlots \ref{fig:_phi_PV_allconst_reco_leading_jet}~att. 

\figureEMLlv{/reco/PV/charge/allconst/}
          {_phi_PV_allconst_reco_leading_jet}
          {\leadingjet vilkmes vektora $\phi$ dimensijas sadalījums, iekļaujot visas strūklas sastāvdaļas.}

Vadošās vieglās strūklas vilkmes vektora lieluma sadalījums, iekļaujot visas strūklu komponentes, ir aplūkojams \ref{fig:_mag_PV_allconst_reco_leading_jet}~att. Vilkmes vektora lielums parasti nepārsniedz \num{0.02}~[bez mērvienības].

\figureEMLlv{/reco/PV/charge/allconst/}
          {_mag_PV_allconst_reco_leading_jet}
          {\leadingjet vilkmes vektora lieluma sadalījums, iekļaujot visas strūklas sastāvdaļas.}

\section{Vilkmes leņķis}

Vilkmes leņķa sadalījums starp ar krāsām saistītām strūklām  no \leadingjet uz \scndleadingjet ar visām strūklu komponentēm un ar jebkuru \DeltaR\ ir redzams \ref{fig:_pull_angle_allconst_reco_leading_jet_scnd_leading_jet_DeltaRTotal}~att.

\figureEMLlv{/reco/pull_angle/DeltaRTotal/charge/allconst/}
          {_pull_angle_allconst_reco_leading_jet_scnd_leading_jet_DeltaRTotal}
          {Vilkmes leņķa sadalījums no \leadingjet uz \scndleadingjet ar jebkuru \DeltaR, iekļaujot visas strūklu sastāvdaļas.}


\ref{fig:_pull_angle_allconst_reco_leading_b_scnd_leading_b_DeltaRTotal}~att. ir vilkmes leņķa sadalījums gadījumā, kad nav sagaidāma krāsu saistība starp strūklām \textendash\ no \leadingb uz \scndleadingb, iekļaujot visas strūklu sastāvdaļas un pie visām \DeltaR vērtībām.

\figureEMLlv{/reco/pull_angle/DeltaRTotal/charge/allconst/}
          {_pull_angle_allconst_reco_leading_b_scnd_leading_b_DeltaRTotal}
          {Vilkmes leņķa no \leadingb uz \scndleadingb sadalījums pie visām \DeltaR vērtībām, iekļaujot visas strūklu sastāvdaļas.}

Papildu iespēja novērot vilkmes leņķa sadalījumu starp objektiem, starp kuriem nav krāsu saistība, ir izvēlēties strūklu un leptonu. \ref{fig:_pull_angle_allconst_reco_leading_jet_lepton_DeltaRTotal}~att. attēlota vilkmes leņķa sadalījumu starp \leadingjet un lādēto leptonu. 

\figureEMLlv{/reco/pull_angle/DeltaRTotal/charge/allconst/}
          {_pull_angle_allconst_reco_leading_jet_lepton_DeltaRTotal}
          {Vilkmes leņķa sadalījums no \leadingjet uz lādēto leptonu pie visām \DeltaR vērtībām, iekļaujot visas strūklu sastāvdaļas.}

Attēlos \ref{_pull_angle_allconst_reco_leading_jet_scnd_leading_jet_DeltaRTotal}\textendash\ref{_pull_angle_allconst_reco_leading_jet_lepton_DeltaRTotal} redzams, ka centrālais paugurs vilkmes leņķa sadalījumā ir izteikts, ja iesaistītas ar krāsām saistītas strūklas, kā arī sadalījums izlīdzinās, ja ir iesaistīti ar krāsām nesaistīti objekti. 

Centrālais paugurs var būt redzams arī gadījumos, ja starp fizikālo objektu vektoriem pastāv kolinearitāte, kaut arī paši fizikas objekti nav saistīti ar krāsām. Šāds gadījums redzams, apskatot vilkmes leņķa sadalījumu no \leadingjet uz hadronisko \PW - \ref{fig:_pull_angle_allconst_reco_leading_jet_had_w_DeltaRTotal}~att. 

\figureEMLlv{/reco/pull_angle/DeltaRTotal/charge/allconst/}
          {_pull_angle_allconst_reco_leading_jet_had_w_DeltaRTotal}
          {Vilkmes leņķa sadalījums no \leadingjet uz hadronisko \PW pie visām \DeltaR vērtībām, iekļaujot visas strūklu sastāvdaļas.}

Interesants gadījums ir kūlis. \ref{fig:_pull_angle_allconst_reco_leading_jet_beam_DeltaRTotal}~att. aplūkojams vilkmes leņķa sadalījums no \leadingjet uz kūļa pozitīvo virzienu. Redzams neliels paugurs taisnā leņķī.

\figureEMLlv{/reco/pull_angle/DeltaRTotal/charge/allconst/}
          {_pull_angle_allconst_reco_leading_jet_beam_DeltaRTotal}
          {Vilkmes leņķa sadalījums no \leadingjet uz kūļa pozitīvo virzienu, iekļaujot visas strūklas sastāvdaļas.}

\section{\DeltaR ietekme}

Gadījumos, kad divas strūklas atrodas cieši viena pie otras $\eta-\phi$ telpā, strūklu sakopošanas algoritms mēdz pievienot vienas strūklas (ar mazāko \pt) sastāvdaļas otrai strūklai (ar lielāko \pt). Šis process ietekmē analīzi ar vilkmes leņķi, jo vilkmes vektoram šādi rodas nosliece rādīt uz strūklu, no kuras tika atņemtas sastāvdaļas. \ref{fig:_pull_angle_allconst_reco_leading_jet_scnd_leading_jet_DeltaRle1p0}~att. un \ref{fig:_pull_angle_allconst_reco_leading_jet_scnd_leading_jet_DeltaRgt1p0}~att. redzami divi gadījumi: cieši klātesošas strūklas ar $\DeltaR\leq1,0$ un attālas strūklas ar $\DeltaR>1,0$.

\figureEMLlv{/reco/pull_angle/DeltaRle1p0/charge/allconst/}
          {_pull_angle_allconst_reco_leading_jet_scnd_leading_jet_DeltaRle1p0}
          {Vilkmes leņķa sadalījums no \leadingjet uz \scndleadingjet pie \DeltaR$\leq1,0$, iekļaujot visas strūklu sastāvdaļas.}

\figureEMLlv{/reco/pull_angle/DeltaRgt1p0/charge/allconst/}
          {_pull_angle_allconst_reco_leading_jet_scnd_leading_jet_DeltaRgt1p0}
          {Vilkmes leņķa sadalījums no \leadingjet uz \scndleadingjet pie \DeltaR$>1,0$, iekļaujot visas strūklu sastāvdaļas.}


\section{Jutīguma analīze}

Vilkmes leņķa metodoloģijas jutīgums tika pētīts, lietojot turpmāk aprakstītos \gls{sliekšņus}:
\begin{description}
\item[Hadroniskā \PW bozona \pt] Tika izvēlēts 50~\GeV slieksnis, un tika iegūti vilkmes leņķa sadalījumi, hadroniskā \PW bozona \pt esot zemākam vai pārsniedzot šo slieksni. Rezultāti ir atainoti \ref{fig:_pull_angle_hadWPtgt50p0GeV_reco_leading_jet_scnd_leading_jet_DeltaRTotal}\textendash\ref{fig:_pull_angle_hadWPtle50p0GeV_reco_leading_jet_scnd_leading_jet_DeltaRTotal}~att.

\item[Strūklas sastāvdaļu skaits] Tika izvēlēts strūklas sastāvdaļu skaita $N$ slieksnis vienāds ar 20, un tika iegūti vilkmes leņķa sadalījumi, strūklu sastāvdaļu skaitam $N$ pārsniedzot vai esot zemākam par šo slieksni. Rezultāti ir atainoti \ref{fig:_pull_angle_PFNgt20_reco_leading_jet_scnd_leading_jet_DeltaRTotal}\textendash\ref{fig:_pull_angle_PFNle20_reco_leading_jet_scnd_leading_jet_DeltaRTotal}~att.
                                        
\item[Strūklas sastāvdaļu \pt] Tika izvēlēts strūklas sastāvdaļu \pt slieksnis vienāds ar 0,5~\GeV, un tika iegūti vilkmes leņķa sadalījumi, strūklu sastāvdaļu \pt esot lielākam vai mazākam par šo slieksni. Rezultāti ir atainoti \ref{fig:_pull_angle_PFPtgt0p5GeV_reco_leading_jet_scnd_leading_jet_DeltaRTotal}\textendash\ref{fig:_pull_angle_PFPtle0p5GeV_reco_leading_jet_scnd_leading_jet_DeltaRTotal}~att.

\item[Vilkmes vektora lielums] Tika izvēlēts vilkmes vektora lieluma slieksnis vienāds ar 0,005~[bez mērvienībām], un tika iegūti vilkmes leņķa sadalījumi, vilkmes vektora lielumam pārsniedzot vai esot zemākam par šo slieksni. Rezultāti ir attēloti \ref{fig:_pull_angle_PVMaggt0p005_reco_leading_jet_scnd_leading_jet_DeltaRTotal}\textendash\ref{fig:_pull_angle_PVMagle0p005_reco_leading_jet_scnd_leading_jet_DeltaRTotal}~att.
\end{description}

\figureEMLlv{/reco/pull_angle/DeltaRTotal/hadronic_W_Pt/hadWPtgt50p0GeV/}
          {_pull_angle_hadWPtgt50p0GeV_reco_leading_jet_scnd_leading_jet_DeltaRTotal}
          {Vilkmes leņķa sadalījums no \leadingjet uz \scndleadingjet ar \PW bozona \pt$>$50~\GeV, pie visiem \DeltaR, iekļaujot visas strūklu sastāvdaļas.}

\figureEMLlv{/reco/pull_angle/DeltaRTotal/hadronic_W_Pt/hadWPtle50p0GeV/}
          {_pull_angle_hadWPtle50p0GeV_reco_leading_jet_scnd_leading_jet_DeltaRTotal}
          {Vilkmes leņķa sadalījums no \leadingjet uz \scndleadingjet ar \PW bozona \pt$\leq$50~\GeV, pie visiem \DeltaR, iekļaujot visas strūklu sastāvdaļas.}
          

\figureEMLlv{/reco/pull_angle/DeltaRTotal/PF_number/PFNgt20/}
          {_pull_angle_PFNgt20_reco_leading_jet_scnd_leading_jet_DeltaRTotal}
          {Vilkmes leņķa sadalījums no \leadingjet uz \scndleadingjet ar strūklas sastāvdaļu skaitu $N>20$, pie visiem \DeltaR, iekļaujot visas strūklu sastāvdaļas .}

\figureEMLlv{/reco/pull_angle/DeltaRTotal/PF_number/PFNle20/}
          {_pull_angle_PFNle20_reco_leading_jet_scnd_leading_jet_DeltaRTotal}
          {Vilkmes leņķa sadalījums no \leadingjet uz \scndleadingjet ar strūklas sastāvdaļu skaitu $N\leq20$, pie visiem \DeltaR, iekļaujot visas strūklu sastāvdaļas.}

\figureEMLlv{/reco/pull_angle/DeltaRTotal/PF_Pt/PFPtgt0p5GeV/}
          {_pull_angle_PFPtgt0p5GeV_reco_leading_jet_scnd_leading_jet_DeltaRTotal}
          {Vilkmes leņķa sadalījums no \leadingjet uz \scndleadingjet, ja strūklas sastāvdaļu \pt$>$0.5~\GeV, pie visiem \DeltaR, iekļaujot visas strūklu sastāvdaļas.}

\figureEMLlv{/reco/pull_angle/DeltaRTotal/PF_Pt/PFPtle0p5GeV/}
          {_pull_angle_PFPtle0p5GeV_reco_leading_jet_scnd_leading_jet_DeltaRTotal}
          {Vilkmes leņķa sadalījums  no \leadingjet uz \scndleadingjet, ja strūklas sastāvdaļu \pt$\leq$0,5~\GeV pie visiem \DeltaR, iekļaujot visas strūklu sastāvdaļas.}

\figureEMLlv{/reco/pull_angle/DeltaRTotal/PV_magnitude/PVMaggt0p005}
          {_pull_angle_PVMaggt0p005_reco_leading_jet_scnd_leading_jet_DeltaRTotal}
          {Vilkmes leņķa sadalījums no \leadingjet uz \scndleadingjet ja vilkmes vektora lielums $>$0,005~[bez mērvienībām], pie visiem \DeltaR, iekļaujot visas strūklu sastāvdaļas.}

\figureEMLlv{/reco/pull_angle/DeltaRTotal/PV_magnitude/PVMagle0p005}
          {_pull_angle_PVMagle0p005_reco_leading_jet_scnd_leading_jet_DeltaRTotal}
          {Vilkmes leņķa sadalījums  no \leadingjet uz \scndleadingjet, ja vilkmes vektora lielums $\leq$0,005~[bez mērvienībām], pie visiem \DeltaR, iekļaujot visas strūklu sastāvdaļas.}

Vilkmes leņķa metodoloģija ir jutīga pret hadroniskā \PW bozona \pt, strūklas sastāvdaļu skaitu, strūklas sastāvdaļu \pt, bet ne īpaši jutīga pret vilkmes vektora lielumu. 

