\label{sec:experimental_setup}

Šis pētījums ir veikts ar iespējams vienu no vissarežģītākajiem un vislielākajiem zinātnes aprīkojumiem cilvēces vēsturē, kuru izmanto un apkalpo vērienīgs un globāls pētniecības kolektīvs. LHP un tā eksperimenti tika iecerēti un izveidoti, lai atbildētu uz fundamentāliem jautājumiem fizikā:

\begin{itemize}
\item Pētīt elektrovājo simetrijas laušanu un meklēt Higsa bozonu. Higsa bozona pastāvēšana tika teorētiski paredzēta 1964.~gadā~\cite{Higgs:1964ia}, \cite{Englert:1964et}, un ilgstoši tas bija iztrūkstošais akmentiņš standartmodeļa mozaīkā. Ja to atklātu, tad tiktu apstiprināti mūsu priekšstati par subatomāro pasauli. Par attiecīgo atklājumu vienlaicīgi paziņoja KMS un ATLAS 2012.~gadā~(\cite{Chatrchyan:2012xdj}, \cite{Aad:2012tfa}) pēc gandrīz 50 gadus ilgiem meklējumiem.
\item Pētīt standartmodeļa fiziku ar līdz šim nepieredzētu precizitāti, izmantojot vismodernākos detektorus, lielu integrēto spīdumu un lielu masas centra enerģiju. Viena no visintersantākajām tēmām ir nesen atklātā virsotnes kvarka pētniecība. Savas lielās masas dēļ tas labi saistās ar Higsa bozonu.
\item Izveidot apstākļus, kādos pastāvēja senatnīgā kvarku-gluonu plazma, šādi rodot atbildes uz fundamentāliem jautājumiem par mūsu Visuma evolūciju.
\item Meklēt tumšo matēriju, eksotiskās daļiņas, supersimetriskos partnerus, \gls{papildus dimensijas}, kā arī risināt citas mīklas un hipotēzes ārpus standartmodeļa. Uz šiem jautājumiem joprojām nav rastas pārliecinošas atbildes, un tādēļ ir iecerēts radīt \gls{Augsta spīduma LHP}, \gls{Nākotnes apļveida kolaideri} un citas lielizmēra eksperimentālas iekārtas. 
\end {itemize}

KMS eksperiments ir viens no galvenajiem eksperimentiem LHP. Šajā aprakstā LHP tiks minēts pirmais, pēc tam \textendash\ KMS aprīkojuma apraksts.

\section{LHP}

LHP ir divapļu supravadošs hadronu paātrinātājs un kolaiders, kas ierīkot 26,7~km garā tunelī 45\textendash170~m zem zemes, šķērsojot Francijas un Šveices robežu Ženēvas apkārtnē (\ref{fig:LHC_underground}~att). Hadroni tajā riņķo ar nemainīgu rādiusu bet mainīgu frekvenci. Tātad LHP ir sinhrotrons. LHP atkārtoti izmanto Lielā elektronu-pozitronu paātrinātāja tuneli un \gls{ievades ķēdi}.

Sākotnēji LHP projekts sastapās ar sīvu konkurenci no daudz jaudīgākā Supravadošā superkolaidera (SSK) ASV. K. Rubia (oriģ. C. Rubbia) apgalvoja, ka 10 reizes lielāks spīdums LHP kompensētu tā mazāko enerģiju salīdzinājumā ar SSK. Galu galā 1993.~gadā SSK projekts tika atcelts. Būtisku lomu spēlēja ievērojamais budžeta pieaugums. CERN padome LHP projektu apstiprināja 1994.~gadā. Tas sāka iegūt datus 2008.~gadā.

\begin{figure}[htpb]
  \centering
  \includegraphics[width=0.8\textwidth]{fig/LHC_underground.png}
  \caption{Lielais hadronu paātrinātājs, kas atrodas pazemē uz Francijas\textendash Šveices robežas, Ženēvas apkārtnē~\cite{cds:LHCunderground}.}
  \label{fig:LHC_underground}
\end{figure}

Protoni LHP riņķo praktiski ar gaismas ātrumu. Enerģija uz protonu ir 7~\TeV, $\gamma$ faktors ir 7461. Paātrināt protonu no miera stāvokļa līdz šādai enerģijai vienā paātrinātājā nav praktiski. Tādēļ līdz šādas enerģijas sasniegšanai protoni tiek paātrināti vairākās stadijas CERN paātrinātāju kompleksā (\ref{fig:CERN_accelerator_complex}~att.):

\begin{itemize}
\item līdz 50~\MeV Linac2,
\item līdz 1,4~\GeV PS \gls{pastiprinātājā},
\item līdz 26~\GeV Protonu sinhrotronā (PS),
\item līdz 450~\GeV Superprotonu sinhrotronā (SPS).
\end{itemize}

\begin{figure}[htpb]
  \centering
  \includegraphics[width=1\textwidth]{fig/CERNacceleratorcomplex.jpg}
  \caption{CERN paātrinātāju komplekss~\cite{espace:CERNacceleratorcomplex}.}
  \label{fig:CERN_accelerator_complex}
\end{figure}

Pēc tam, kad protoni ir tikuši pilnībā paātrināti, to riņķveida kustība tiek uzturēta \textendash LHP ir \gls{uzkrāšanas aplis}. Vienā \gls{pūlī} ir sakopoti $1,15\times10^{11}$~protoni, kopā riņķo 2808~pūļi. Apgrieziena frekvence ir 11,245~kHz \cite{Bruning:2004ej}. Katra pūļa šķērsojums ilgst 25~ns. Kūļa caurulēs tiek uzturēts ārkārtīgi augsts vakuums.

LHP izmanto supravadošu magnētu sistēmas. Īpaši izceļams ir dipola magnēts, kas kūli ieloka apļveida arkā, un kvadrupola magnēts, kas kūli sašaurina pirms sadursmes punktiem. Augstākas pakāpes magnēti precizē kūļa kustību un koriģē to. Magnētu sistēmas balstās uz NbTi Rezerforda kabeli, kas ir atdzesēts ar hēliju zem 2~K \textendash\ zem hēlija lambdas punkta. Līdz ar to atšķirībā no citiem lieliem paātrinātājiem, kas arī izmanto NbTi, bet strādā virs hēlija lambdas punkta (Tevatron-FNAL, HERA-DESY un RHIC-BNL), LHP dipolu magnētos tiek iegūts daudz stiprāks 8~T lauks. LHP vajadzībām tika izstrādāts īpašs divi-vienā dipola magnēts, kurā tiek izmantots viens jūgs divu polaritāšu laukiem, kas atbilst abiem pretējos virzienos riņķojošiem protonu kūļiem. Lai atdzesētu magnētus, tiek izmantota vislielākā kriogēnā sistēma uz Zemes~\cite{MYERS:2013hra} \cite{Evans:2008zzb}.

LHP nominālā masas centra enerģija ir 14~\TeV. Pirmajā datu gūšanas periodā no 2010.\textendash2013. gadam. tas strādāja ar \sqrts=7\textendash8~\TeV. Šo periodu sauc par I~\gls{darba periodu}. Otrajā datu gūšanas periodā no 2015.\textendash2018.~gada, ko sauc par II~darba periodu, tas strādāja ar \sqrts=13\textendash14~\TeV. Šis pētījums ir veikts ar II~darba perioda datiem.

LHP ietilpst divi liela spīduma eksperimentālie ievietojumi \textendash KMS un ATLAS, katram no kuriem nominālais spīdums ir virs $10~{\nicefrac{1}{\text{pb}\cdot\text{s}}}$, viens \cPqb fizikas eksperiments, kura nominālais spīdums ir $0,1~\nicefrac{1}{\text{pb}\cdot\text{s}}$ un viens speciāls jonu sadursmju eksperiments - ALICE. 

\section{KMS detektors}

KMS detektors atrodas LHP piektajā punktā netālu no franču ciemata Sesī (oriģ. Cessy) starp Ženēvas ezeru un Žurā kalniem (oriģ. Jura). Tas atrodas pazemes bunkuros 100~m dziļumā. 

KMS detektors ir projektēts dažādām fizikas programmām \TeV skalā. Tas ir sīpola tipa detektors, aptverot $4\pi$ telpiskā leņķa ap sadursmes punktu. KMS detektoru veido šādi slāņi, sākot no kūļa ass \textendash\ silīcija pikseļu un \gls{strēmeļu} trekeris, vara volframāta elektromagnētiskais kalorimetrs, misiņa un plastikāta scintilators, supravadošs magnēts, kas rada 3,8\textendash4,0~T magnētisko lauku, un gāzu jonizācijas mionu spektrometrs \cite{Chatrchyan:2008aa}. KMS detektoram ir cilindra forma. Tā abos galos ir \gls{gala segumi}, bet tā centrālā daļa tiek saukta par \gls{mucu}. KMS detektora garums ir 21,6~m, diametrs 14,6~m un kopējais svars 12~500~t. KMS detektora atvērums redzams \ref{fig:CMS_detector}~att.

\begin{figure}[hbtp]
\centering
\def\twidth{1}
\includegraphics[width=\twidth\textwidth]{fig/cms_160312_06lv}
\caption{KMS detektora atvērums \cite{Sakuma:2013jqa}.}
\label{fig:CMS_detector}
\end{figure}

Sākot no kūļu mijiedarbības reģiona, daļiņas vispirms nonāk trekerī, kurš no signāliem (\gls{ierosinājumiem}) jutīgajos slāņos rekonstruē lādēto daļiņu trajektorijas un sākuma punktus (\gls{virsotnes}). Trekeris atrodas magnētiskajā laukā, kas noliec lādēto daļiņu trajektorijas un ļauj izmērīt to elektrisko lādiņu un momentus. Elektroni un fotoni pēc tam tiek absorbēti elektromagnētiskajā kalorimetrā (ECAL). Attiecīgie elektromagnētiskie lieti tiek detektēti kā enerģijas \gls{puduri} vienkopus esošās šūnās, no kuriem var izmērīt daļiņu enerģiju un virzienu. Lādēti un neitrāli hadroni ECAL var ierosināt hadronu lietu, kuru pēc tam pilnībā absorbē hadroniskais kalorimetrs (HCAL). Attiecīgie puduri tiek izmantoti, lai novērtētu to enerģiju un virzienu.  Mioni un pioni šķērso kalirometrus ar vāju vai nekādu mijiedarbību. Kamēr neitrino izbēg nedetektēti, mioni rada ierosinājumus aiz kalorimetriem esošos papildu trekēšanas slāņos, kurus sauc par mionu detektoriem. Šis vienkāršotais apskats ir atspoguļots \ref{fig:CMSpf}~att., kurā ir redzama KMS detektora loksne.

Ievērojami uzlabots notikumu apraksts tiek iegūts, atrodot sakarības starp novērojumiem dažādos detektora slāņos (trekiem un puduriem), lai identificētu katru \gls{gala stāvokļa} daļiņu, un apkopojot to mērījumus, lai rekostruētu daļiņas īpašības, balstoties uz šo identifikāciju. Šo visaptverošo pieeju sauc par \gls{daļiņu plūsmas} (PF) rekonstrukciju \cite{Sirunyan:2017ulk}.

\begin{figure}[h]
  \centering
  \includegraphics[width=1\textwidth]{fig/CMSpf.png}
  \caption{Daļiņu mijiedarbību skice KMS šķērsgriezumā no kūļa mijiedarbības punkta līdz mionu detektoram~\cite{Sirunyan:2017ulk}.}
  \label{fig:CMSpf}
\end{figure}

Smalkās \gls{graudainības} un ātrās atbildes trekeris~\cite{Karimaki:368412}, \cite{tracker_addendum} ir svarīgs segments smalko kūļa sastāvdaļu izšķiršanā. Tas ir precīzi novietots līdz ar kūļa asi, un tā kopējais garums ir 5,8~m, rādiuss 2,5~m. KMS soleonīds rada homogēnu un koaksiālu 3,8\textendash4,0~T magnētisko lauka visā trekera tilpumā. Kad rādiuss ir zem 10~cmm ierosinājuma biežuma pakāpe ir $100~\nicefrac{\text{kHz}}{\text{mm}^2}$. Lai sasniegtu vēlamo izšķirtspēju \SI{100}{\um}~$\times$~\SI{150}{\um}, tiek izmantoti pikseļu detektori. Pie lielāka rādiusa samazinātā daļiņu plūsma pieļauj silīcija mikrostrēmeļu detektoru izmantošanu. To tipiskais izmērs ir \SI{10}{cm}~$\times$~\SI{80}{\um} līdz \SI{25}{cm}~$\times$~\SI{150}{\um}, izmēram pieaugot, pieaugot rādiusam. Kopā ir 66~milj. pikseļu ar 1~$\text{m}^2$ aktīvo virsmu pikseļu detektorā un 9,3~milj. strēmeļu ar 193~m${}^2$ aktīvo virsmu strēmeļu detektorā.

KMS elektromagnētiskais kalorimetrs (ECAL) ir hermētisks homogens kalorimetrs, kurš izveidots no 61~200 svina volframāta ($\text{PbWO}_{4}$) kristāliem, kas uzstādīti mucā, kā arī \num{7324} kristāliem gala segumos. Muca aptver pseido\gls{straujumu} intervālā $\left|\eta\right|<1,479$, gala segumi aptver pseidostraujumu intervālā $1,479<\left|\eta\right|<3,0$. Pirms gala segumiem ir uzstādīts pirmslietus detektors. Mucā tiek izmantotas lavīnas fotodiodes (APDs), bet gala segumos tiek izmantotas vakuuma fototriodes (VPTs). $\text{PbWO}_{4}$ kristāliem piemīt tādas īpašības, kas tos padara piemērotus LHP elektromagnētiskajam kalorimetram. To lielais blīvums 8,28~$\nicefrac{\text{g}}{\text{cm}^3}$, mazais radiācijas garums (0,89~cm) un mazais Moljēra rādiuss (2,2~cm) ļauj radīt smalkas graudainības un kompaktu kalorimetru. Scintilācijas rimšanas laiks $\text{PbWO}_{4}$ ir tādas pašas pakāpes lielums, kā LHP pūļu šķērsojuma laiks: apmēram 80~\% gaismas tiek izstaroti~25 ns. Mucas kristālu šķērsgriezuma laukums atbilst aptuveni 0,0174~$\times$~0,0174 $\eta-\phi$ mērvienībās, kam atbilst priekšējā šķērsgriezuma laukums 22~$\text{mm}^2$~$\times$~22~$\text{mm}^2$ un aizmugurējā šķērsgriezuma laukums 26~$\text{mm}^2$~$\times$~26~$\text{mm}^2$. Kristālu garums ir 230~mm, kas atbilst 25,8~$X_{0}$. Mucā ir 61~200 kristāli. Gala segumos kristālu aizmugurējā šķērsgriezuma laukums ir 30~$\times$~30~$\text{mm}^2$, un priekšējā šķērsgriezuma laukums ir 28,62~$\text{mm}^2$~$\times$~28,62~$\text{mm}^2$, kā arī to garums ir 220~mm, kas atbilst 24,7~$X_{0}$. Turklāt \gls{uzticamajā reģionā} $1,653<\left|\eta\right|<2,6$ atrodas pirmslietus detektors, kura galvenais mērķis ir identificēt neitrālos pionus gala segumos. Pirmslietus detektors sastāv no svina izstarotāja, kurā ienākošie elektroni/fotoni izraisa elektromagnētiskos lietus. Aiz svina izstarotāja atrodas silīcija strēmeles, lai izmērītu atstāto enerģiju un lietu šķērsprofilus. Enerģijas izšķirtspēja mucas elektromagnētiskajā kalorimetrā ir atkarīga no ienākošās enerģijas, un tā ir novērtēta no 0,94~\% ($\nicefrac{sigma}/{E}$) pie 20~\GeV līdz 0,34~\% pie 250~\GeV~\cite{Adzic:2007mi}. 

Hadroniskais kalorimetrs~\cite{HCAL_report} sastāv no mucas ($\left|\eta\right|<1,3$) un diviem gala seguma diskiem ($1,3<\left|\eta\right|<3,0$). Hadronu kalorimetra tilpums centrālajā pseidostraujuma intervālā ir ierobežots. Tādēļ tiek izmantots ārējs \gls{atblāzmas novērotājs} aiz solenoīda. Solenoīds tiek izmantots kā papildu absorbējošais materiāls atblāzmas novērotājam. Absorbētājs sastāv no 40~mm biezas priekšējās tērauda plāksnes, kurai seko astoņas 50,5~mm biezas misiņa plāksnes, sešas 56,5~mm biezas misiņa plāksnes un 75~mm bieza tērauda aizmugurējā plāksne. Kopējais absorbētāja biezums 90~$^{\circ}$ leņķī ir 5,82~mijiedarbības garumi ($\lambda_{\text{I}}$). Kā aktīvais materiāls tiek izmantots \gls{dakstiņos} izkārtots plastikāta scintilātors. Gaismas izvadīšanai tiek izmantotas viļņa garuma pārbīdes šķiedras. Nolasījumi hadroniskajā kalorimetrā tiek veikti individuālos \gls{torņos} ar kopējo šķērsgriezuma laukumu $\Delta\eta\times\Delta\phi=0,087\times0,087$ intervālā $\left|\eta\right|<1,6$ un $0,17~\times~0,17$ pie liekākiem pseidostraujumiem. Hadronisko kalorimeteru papildina šaurleņka hadronu kalorimetri $\left|\eta\right|$ intervālā līdz $\simeq5,0$, kur daļiņu plūsma un radiācijas bojājumi ir vislielākie. Hadronu šaurleņķa kalorimetrs sastāv no tērauda absorbētāja, kuru veido rievotas plāksnes. Rievās gareniski kūļa virzienam ir ievietotas radiācijas noturīgas kvarca šķiedras, un tās nolasa fotoreizinātāji. Signāli tiek sagrupēti tā, lai varētu izveidot kalorimetra torņus ar šķērsgriezumu $\Delta\eta\times\Delta\phi=0,175\times0,175$ lielākajā daļā pseidostraujuma intervāla. 

Magnēts ir novietots aiz kalorimetriem un trekera, lai nodrošinātu, ka pēc iespējas mazāks materiāla daudzums atrodas starp minētajiem apakšdetektoriem un mijiedarbības punktu. Magnēta garums ir 12,5~m, un tā brīvurbuma rādiuss ir 3,15~m. Tinums rada 3,8\textendash4,0~T vienmērīgu aksiālu magnētisko lauku trekerī un kalorimetros. Magnēts strādā 4,45~K temperatūrā, un tajā tiek izmantoti NbTi supravadoši tinumi. Magnētu raksturo augsta uzkrātās enerģijas/masas attiecība 11,6~$\nicefrac{\text{kJ}}{\text{kg}}$.

Mionu kanāls ir ļoti efektīvs instruments interesējošu \gls{AEF} procesu pētīšanai, un tas ir bijis ļoti nozīmīgs jau pirmajās KMS eksperimenta iecerēs. Tas izkaidrojams ar to, ka mionus ir ļoti viegli novērot un tiem raksturīgi nelieli radiācijas zudumi trekera materiālā. Četras mionu detektora plāksnes atrodas ārpus solenoīda tinuma, kuras caurvij trīs tērauda jūga slāņi~\cite{muon_tech_rep}. Mucā, kas atbilst pseidostraujuma intervālam $\left|\eta\right|<1,2$, kur mionu biežums ir zems un 4~T magnētiskais lauks ir vienmērīgs un pārsvarā ierobežots tērauda jūgā, tiek izmantotas dreifa kameras. Gala segumos $0,9<\left|\eta\right|<2,4$, kur mionu biežums un trokšņu līmenis ir augsts, kā arī magnētiskais lauks ir liels un nevienmērīgs, mionu sistēmā tiek izmantotas katoda strēmeļu kameras (KSK). Trokšņu līmeņa nenoteiktības dēļ un mionu sistēmas nenoteiktības dēļ tās spējā nomērīt pareizo kūļa šķērsošanas laiku, LHP, sasniedzot nominālo spīdumu, papildus gan mucā, gan gala segumos tiek izmantota speciāla trigeru sistēma, kas sastāv no pretestības plākšņu kamerām (PPK). PPK nodrošina ātru, neatkarīgu un ļoti segmentētu trigeri ar asu \pt slieksni lielā mionu sistēmas pseidostraujuma intervālā ($\left|\eta\right|<1,6$). Daļiņu plūsmas rekonstrukcija izmanto globālu trajektorijas saderināšanu mionu detektorā un iekšējā trekerī. 

Strūklas tiek rekonstruētas, izmantojot anti-$k_{\text{T}}$ algoritmu~\cite{Cacciari:2008gp} ar rādiusa parametru $R=0,4$ un \FASTJET pakotnes izpildījumu~\cite{Cacciari:2011ma}. Attālums starp strūklām $d_{ij}$ tiek noteikts atbilstoši šādai formulai, izmantojot $p=-1$:

\begin{equation}
d_{ij}=\text{min}(k_{\text{T}i}^{2p}, k_{\text{T}j}^{2p})\frac{\Delta_{ij}^{2}}{R^{2}},
\end{equation}

\noindent kur $\Delta_{ij}^{2}=(y_{i}-y_{j})^{2}+(\phi_{i}-\phi_{j})^{2}$ un $k_{\text{T}i}$, $y_{i}$, $\phi_{i}$ ir attiecīgi daļiņas $i$ šķērsmoments, straujums, un azimuts. 

Svarīga šī algoritma priekšrocība ir tāda, ka \gls{mīkstās} daļiņās neizmaina strūklas formu. Ja attālums starp strūklām $\Delta_{ij}\leq2R$, tad tās ieņem konisku formu. 
