Nenoteiktības tiek iedalītas eksperimentālajās un teorētiskajās nenoteiktībās. Iekļaujot kādu nenoteiktību no pirmās grupas, tiek variēts kāds parametrs notikumu atlasē, piemēram, datu/MK koeficients. Teorētiskās nenoteiktības atspoguļo mūsu zināšanu trūkumu par reālo pasauli, piemēram, hadronizācijas procesa norises niansēm. 

\section{Eksperimentālās nenoteiktības}
\begin{description}
\item[Sagrūdums] Kaut gan sagrūdums ir iekļauts simulācijā, pastāv nenoteiktībā tā modelēšanā. Lai novērtētu modelēšanas kļūdu, tiek varēts minimālās noslieces šķērsgriezuma parametrs par 5 \% attiecībā pret tā sākotnējo novērtējumu. 
\item[Trigera un atlases efektivitāte] Trigera efektivitātes nenoteiktība un leptona identifikācijas un izolācijas koeficienta nenoteiktība tiek iestrādāta, mainot nominālos parametrus augšup un lejup. Mionu trekera efektivitātes nenoteiktība arī ir iekļauta šajā kategorijā un ir pievienota otrajā pakāpē, kaut gan tās sagaidāmā ietekme ir nenozīmīga. Koeficientu noteikšana ir aprakstīta \ref{chap:data_and_mc_samples} nod.
\item[Strūklu enerģijas izšķirtspēja] Izmantojam ieteiktos strūklas enerģijas mērījumus \cite{twiki:JER}. Katra strūkla tiek tālāk izsmērēta augšup un lejup atkarībā no tās \pt un $\eta$ attiecībā pret centrālo vērtību, kas izmērīta datos. Galvenā šīs nenoteiktības ietekme ir notikumu izslēgšana/iekļaušana, ja to strūklu parametri ir tuvu atlases slieksnim. 
\item[Strūklu enerģijas mērogs] Lai kalibrētu strūklas simulācijā, izmantojam strūklas enerģijas mēroga parametrizāciju, kas atkarīga no \pt - $\eta$. Šī parametrizācija ir atkarīga no Spring16 V3 korekcijām, kuras sizstrādājusi JetMET Fizikas objektu grupa \cite{twiki:JER}.  Variējot strūklu enerģijas skalu, tiek pārrēķināts arī iztrūkstošās šķērsenerģijas $E^{\text{miss}}_{T}$ novērtējums. Galvenā šīs nenoteiktības ietekme ir notikumu izslēgšana/iekļaušana, ja to strūklu parametri ir tuvu bezsaistes atlases slieksnim. 
\item[\cPqb-atzīmēšana] \cPqb - atzīmēšanas nominālā efektivitāte tiek koriģēta ar no \pt-atkarīgiem koeficientiem, kurus piedāvā BTV Fizikas objektu gruba \cite{twiki:BTV}. Atkarībā no strūklas garšas, \cPqb-atzīmēšanas lēmums tiek atjaunināts saskaņā ar koeficientu. Arī koeficients tiek variēts saskaņā ar tā nenoteiktību. Galvenā šīs nenoteiktības ietekme ir uz lēmumiem par strūklu \cPqb-atzīmēšanu apstiprināšanu/noraidīšanu.
\item[Trekēšanas efektivitāte] TRK un MUO Fizikas objektu grupas ir izstrādājušas trekēšanas efektivitātes koeficientus kā funkciju no treka $\eta$ va rekonustruēto virsotņu daudzumu. Visi šie koeficienti ir atkarīgi no datu gūšanas periodiem.
\end{description}

\section{Teorētiskās nenoteiktības}
\begin{description}
\item[KHD koeficientu izvēle] Apskatām antikorelētas faktorizācijas un renormalizācijas koeficientu variācijas ar kārtu 0,5 un 2 ($\mu_R/\mu_F$) \ttbar paraugos. Šīs variācijas tiek noglabātas simulētajos notikumus kā alternatīva svaru kopa, kas tiek izmantota šīs nenoteiktības novērtēšanā. Tiek izmantota 7 variāciju kopums (izņemot op. variāciju $\mu_R/\mu_F$).
\item[\EVTGEN] Sistemātiskā nenoteiktība rodas, ja smago garšu daļiņu, pamatā $B$ un $D$ mezonu sabrukums tiek simulēts ar Monte Karlo notikumu ģeneratoru \EVTGEN.
\item[Hadronizatora izvēle] Sistemātiskā nenoteiktība rodas, ja hadronizācijas procesu modelē ar \HERWIGpp. 
\item[Virsotnes kvarka masa] Visprecīzākie virsotnes kvarka masas mērījumi norāda uz nenoteiktību $\pm 0,49 \text{GeV}$ \cite{Khachatryan:2015hba}. Tomēr mēs piesardzīgi pieņemam nenoteiktību vienādu ar $\pm 1 \text{GeV}$. 
\item[\PYTHIA uzskaņojumi] Tiek izmantoti šādi \PYTHIA uzskaņojumi:
  \begin{enumerate}
  \item Matricas elementu + partonu lietus savietošanas shēma 
  \item Partonu lietus mērogs 
  \item Krāsu otrreizējās savienošanas modelis 
  \item Pamata notikuma variācijas 
  \end{enumerate}
\end{description}

\ref{tab:mcsystdatasets} tabulā ir apkopotas teorētisko nenoteiktību simulāciju paraugi un to šķērsgriezums. Tie ir iegūti no

RunIISummer16MiniAODv2-PUMoriond17\_80X\_mcRun2\_asymptotic\_2016\_TrancheIV\_v6

izstrādes.

\begin{table}[!htp]
\begin{center}
\caption{Teorētisko nenoteiktību simulāciju paraugi.}
\label{tab:mcsystdatasets}
\begin{tabular}{llr}
\hline
Signāla variācija & Datu kopa & $\sigma[pb]$\\
\hline
\multirow{4}{*}{Partonu lietus mērogs}
& {\small TT\_TuneCUETP8M2T4\_13TeV-powheg-isrup-pythia8}     & 832\\
& {\small TT\_TuneCUETP8M2T4\_13TeV-powheg-isrdown-pythia8}   & 832\\
& {\small TT\_TuneCUETP8M2T4\_13TeV-powheg-fsrup-pythia8}     & 832\\
& {\small TT\_TuneCUETP8M2T4\_13TeV-powheg-fsrup-pythia8}     & 832\\\hline
\multirow{2}{*}{Pamata notikums}
& {\small TT\_TuneCUETP8M2T4up\_13TeV-powheg-pythia8 }        & 832\\
& {\small TT\_TuneCUETP8M2T4down\_13TeV-powheg-pythia8}       & 832\\\hline
\multirow{2}{*}{ME-PS savietošanas skala (hdamp)}
& {\small TT\_hdampUP\_TuneCUETP8M2T4\_13TeV-powheg-pythia8}  & 832\\
& {\small TT\_hdampDOWN\_TuneCUETP8M2T4\_13TeV-powheg-pythia8}& 832 \\\hline
\multirow{3}{*}{Krāsu otrreizējā savienošanās}
& {\small TT\_TuneCUETP8M2T4\_erdON\_13TeV-powheg-pythia8 }   & 832\\
& {\small TT\_TuneCUETP8M2T4\_QCDbasedCRTune\_erdON\_13TeV-powheg-pythia8} & 832\\
& {\small TT\_TuneCUETP8M2T4\_GluonMoveCRTune\_13TeV-powheg-pythia8} & 832\\\hline
\multirow{2}{*}{Virsotnes masa}
& {\small TT\_TuneCUETP8M2T4\_mtop1715\_13TeV-powheg-pythia8 }& 832\\
& {\small TT\_TuneCUETP8M2T4\_mtop1735\_13TeV-powheg-pythia8} & 832\\\hline
\HERWIGpp & {\small TT\_TuneEE5C\_13TeV-powheg-herwigpp}      & 832\\
\hline
\end{tabular}
\end{center}
\end{table}

