LHP, kas strādā ar 13~TeV masas centra enerģiju, ir virsotnes kvarku fabrika. Virsotnes kvarku ražošanas šķērsgriezums LHP ir 803~pb. Virsotnes kvarka mūža ilgums ir $3,3\times10^{-25}$~s, un tas ir tik īss, ka atšķirībā no citiem kvarkiem virsotnes kvarks sabrūk, pirms tas hadronizējas. Virsotnes kvarks sabrūk vājajā ceļā, izstarojot \PW bozonu. Ja \PW bozons sabrūk hadroniskajā veidā, stiprajā kodola mijiedarbībā tiek radītas daļiņu strūklas. Šo procesu apraksta kvantu hromodinamika, un varam modelēt virsotnes kvarka sabrukšanas procesu ar krāsu lādiņu un krāsu saitēm. Strūklas, kas radušās, sabrūkot \PW bozonam, mijiedarbojas krāsu laukā (tās ir saistītas ar krāsām). Saistība ar krāsām atstāj pamanāmus eksperimentālus nospiedumus, ko mēs varam novērot KMS detektorā, īpaši izmantojot tā trekeri, 4~T solenoīdu un kalorimetrus. KMS eksperimentā šāds pētījums tiek veikts pirmoreiz. Krāsu saistību starp strūklām, kas radušās, sabrūkot virsotnes kvarku pārim, pētām, izmantojot gala stāvokli, ko veido viens lādēts leptons, divas vieglās strūklas un divas \cPqb-atzīmētās strūklas. Izmantojam vilkmes leņķi un daļiņu projicēšanu uz plaknes, ko veido divas strūklas. Tiek izmantots arī krāsu okteta \PW spēļu modelis, lai novērtētu dažādo metožu sniegumu.
