\documentclass[a4paper]{article}
\usepackage[utf8]{inputenc}
\usepackage{ragged2e}
\usepackage{graphicx}
\usepackage{array}
\thispagestyle{empty}
\renewcommand{\normalsize}{\fontsize{12pt}{13.4pt}\selectfont}

\begin{document}
\noindent RTU EEF promocijas padomes priekšsēdētājam\\\noindent prof. Leonīdam Ribickim\\

\hfill\begin{minipage}{\dimexpr\textwidth-6cm}
\begin{flushright}
  doktora studiju programmas ``Elektrotehnoloģiju datorvadība'' doktoranta
  Viestura Veckalna
\end{flushright}
\end{minipage}
\centering
\vskip 3em
\setlength{\parskip}{0.5em}
\textbf{iesniegums}
\justify
Lūdzu atļaut aizstāvēt promocijas darbu par tēmu ``Virsotnes kvarku pāra sabrukšanas ceļā radušos krāsu plūsmu pētījumi ar 13 TeV CERN LHP KMS eksperimentā". Promocijas darbs ir izstrādāts sadarbībā ar Eiropas Kodolpētījumu organizāciju.

Kopā ir sagatavotas 368 publikācijas, kā arī ir noritējušas 3 uzstāšanās starptautiskās konferencēs, 6 uzstāšanās CERN CMS ``Top Generators and Modelling'' sanāksmēs, plakāta prezentācija CERN Zinātnes nedēļā 2017. g. un European School of High Energy Physics 2017. g.
\vskip 3em
\setlength\tabcolsep{0pt}
\begin{tabular}{m{4cm}m{4cm}}
  Viesturs Veckalns & \hspace{3cm}
\end{tabular}

Rīgā, 2019. g. 28. februārī
\end{document}
